% PREÁMBULO
\documentclass[letterpaper]{article}
\usepackage[utf8]{inputenc}
\usepackage[spanish]{babel}

\usepackage{enumitem}
\usepackage{titling}

% Símbolos
	\usepackage{amsmath}
	\usepackage{amssymb}

% Márgenes
	\usepackage
	[
		margin = 1.4in
	]
	{geometry}

% Imágenes
	\usepackage{float}
	\usepackage{graphicx}
	\graphicspath{{imagenes/}}
	\usepackage{subcaption}

% Ambientes
	\usepackage{amsthm}

	\theoremstyle{definition}
	\newtheorem{ejercicio}{Ejercicio}

	\newtheoremstyle{lemathm}{4pt}{0pt}{\itshape}{0pt}{\bfseries}{ --}{ }{\thmname{#1}\thmnumber{ #2}\thmnote{ (#3)}}
	\theoremstyle{lemathm}
	\newtheorem{lema}{Lema}

	\newtheoremstyle{lemademthm}{0pt}{10pt}{\itshape}{ }{\mdseries}{ --}{ }{\thmname{#1}\thmnumber{ #2}\thmnote{ (#3)}}
	\theoremstyle{lemademthm}
	\newtheorem*{lemadem}{Demostración}

% Ajustes
	\allowdisplaybreaks	% Los align pueden cambiar de página

% Macros
	\newcommand{\sumi}[2]{\sum_{i=#1}^{#2}}
	\newcommand{\dint}[2]{\displaystyle\int_{#1}^{#2}}
	\newcommand{\inte}[2]{\int_{#1}^{#2}}
	\newcommand{\dlim}{\displaystyle\lim}
	\newcommand{\limxinf}{\lim_{x\to\infty}}
	\newcommand{\limninf}{\lim_{n\to\infty}}
	\newcommand{\dlimninf}{\displaystyle\lim_{n\to\infty}}
	\newcommand{\limh}{\lim_{h\to0}}
	\newcommand{\ddx}{\dfrac{d}{dx}}
	\newcommand{\txty}{\text{ y }}
	\newcommand{\txto}{\text{ o }}
	\newcommand{\Txty}{\quad\text{y}\quad}
	\newcommand{\Txto}{\quad\text{o}\quad}
	\newcommand{\si}{\text{si}\quad}

	\newcommand{\etiqueta}{\stepcounter{equation}\tag{\theequation}}
	\newcommand{\tq}{:}
	\renewcommand{\o}{\circ}
	% \newcommand*{\QES}{\hfill\ensuremath{\boxplus}}
	% \newcommand*{\qes}{\hfill\ensuremath{\boxminus}}
	% \newcommand*{\qeshere}{\tag*{$\boxminus$}}
	% \newcommand*{\QESHERE}{\tag*{$\boxplus$}}
	\newcommand*{\QES}{\hfill\ensuremath{\blacksquare}}
	\newcommand*{\qes}{\hfill\ensuremath{\square}}
	\newcommand*{\QESHERE}{\tag*{$\blacksquare$}}
	\newcommand*{\qeshere}{\tag*{$\square$}}
	\newcommand*{\QED}{\hfill\ensuremath{\blacksquare}}
	\newcommand*{\QEDHERE}{\tag*{$\blacksquare$}}
	\newcommand*{\qel}{\hfill\ensuremath{\boxdot}}
	\newcommand*{\qelhere}{\tag*{$\boxdot$}}
	\renewcommand*{\qedhere}{\tag*{$\square$}}

	\newcommand{\abs}[1]{\left\vert#1\right\vert}
	\newcommand{\suc}[1]{\left(#1_n\right)_{n\in\N}}
	\newcommand{\en}[2]{\binom{#1}{#2}}
	\newcommand{\upsum}[2]{U(#1,#2)}
	\newcommand{\lowsum}[2]{L(#1,#2)}

	\newcommand{\N}{\mathbb{N}}
	\newcommand{\Q}{\mathbb{Q}}
	\newcommand{\R}{\mathbb{R}}
	\newcommand{\Z}{\mathbb{Z}}
	\newcommand{\eps}{\varepsilon}
	\newcommand{\ttF}{\mathtt{F}}
	\newcommand{\bfF}{\mathbf{F}}

	\newcommand{\To}{\longrightarrow}
	\newcommand{\mTo}{\longmapsto}
	\newcommand{\ssi}{\Longleftrightarrow}
	\newcommand{\sii}{\Leftrightarrow}
	\newcommand{\then}{\Rightarrow}

	\newcommand{\pTFC}{{\itshape 1er TFC\/}}
	\newcommand{\sTFC}{{\itshape 2do TFC\/}}

% Membrete
	\usepackage{fancyhdr}
	\setlength{\headheight}{14pt}
	\pagestyle{fancy}
	\fancyhf{}
	\rhead{\theauthor}
	\lhead{\thetitle}
	\cfoot{\thepage}

% Datos
    \title{Cálculo II\\Tarea 08}
    \author{Rubén Pérez Palacios\\Profesor: Dr. Fernando Nuñez Medina}
    \date{30 Abril 2020}

% DOCUMENTO
\begin{document}
	\maketitle
	
	\section*{Teoremas}

	\begin{enumerate}
		\item Sea ${a_n}$ una suceción con $a_n \in \R$ convergente, entonces la suceción $a_{n+k}$ también es convergente.
		
		
	\end{enumerate}

    \section*{Problemas}

    \begin{enumerate}
        
        \item Comprueba cada una de los siguientes límites
		
		\begin{enumerate}
			\item $\lim_{n\to\infty} \frac{n}{n+1} = 1$.
			
			Veamos lo siguiente

			\[\frac{n}{n+1} = 1 - \frac{1}{n + 1}.\]

			Al ser $lim_{n\to\infty}\frac{1}{n + 1} = 0$ y $lim_{n\to\infty} 1 = 1$, entonces concluimos que

			\[\lim_{n\to\infty}\frac{n}{n+1} = 1.\]

			\item $\lim_{n\to\infty} \frac{n+3}{n^3+1} = 1|$.
			
			Veamos lo siguiente

			\[\frac{n+3}{n^3+1} = \frac{\frac{1}{n^2}+\frac{3}{n^3}}{1 + \frac{1}{n^3}}.\]

			Donde

			\[0 = \lim_{n\to\infty} \frac{0}{1} = \lim_{n\to\infty} \frac{\frac{1}{n^2}+\frac{3}{n^3}}{1 + \frac{1}{n^3}}.\]

			Por lo tanto conlcuimos que

			\[\lim_{n\to\infty}\frac{n}{n+1} = 1.\]

			\item $\lim_{n\to\infty} \sqrt[8]{n^2+1} - \sqrt[4]{n + 1} = 0$.
			
			Veamos primero que

			\[\lim_{n\to\infty} \sqrt[4]{n + 1} - \sqrt[4]{n} = 0.\]

			x|

			Ahora veamos que

			\[\lim_{n\to\infty} \sqrt[8]{n^2 + 1} = \lim_{n\to\infty} \sqrt[8]{n^2},\]

			esto pues 

		\end{enumerate}

		\item Halla los siguientes límites.
		
		\begin{enumerate}
			\item $\lim_{n\to\infty} \frac{n}{n+1} - \frac{n+1}{n}$.

			\[0 = \lim_{n\to\infty} 1 - 1 = \lim_{n\to\infty} \frac{1}{1 + \frac{1}{n}} - \lim_{n\to\infty} \frac{1 + \frac{1}{n}}{1} = \lim_{n\to\infty} \frac{n}{n+1} - \frac{n+1}{n}.\]

			\item $\lim_{n\to\infty} n - \sqrt{n+a}\sqrt{n+b}$.
			
			\begin{align*}
				-\frac{a+b}{2} &= \lim_{n\to\infty} -\frac{(a+b)}{1 + \sqrt{1+\frac{a}{n}} \sqrt{1+\frac{b}{n}}}\\
				&= \lim_{n\to\infty} -\frac{(a+b)}{1 + \frac{\sqrt{n+a}}{\sqrt{n}}\frac{\sqrt{n+b}}{\sqrt{n}}}\\
				&= \lim_{n\to\infty} \frac{ab}{n + \sqrt{n+a}\sqrt{n+b}} - \frac{n(a+b)}{n + \sqrt{n+a}\sqrt{n+b}}\\
				&= \lim_{n\to\infty} \frac{n^2 - (n+a)(n+b)}{n + \sqrt{n+a}\sqrt{n+b}}\\
				&= \lim_{n\to\infty} n - \sqrt{n+a}\sqrt{n+b}	
			\end{align*}

			\item $\lim_{n\to\infty} \frac{2^n+(-1)^n}{2^{n+1}+(-1)^{n+1}}$.
			
			\begin{align*}
				\frac{1}{2} &= \lim_{n\to\infty} \frac{1+\frac{(-1)^n}{2^n}}{2+\frac{(-1)^{n+1}}{2^n}}\\
				&= \lim_{n\to\infty} \frac{2^n+(-1)^n}{2^{n+1}+(-1)^{n+1}}
			\end{align*}

		\end{enumerate}

		\item Realiza lo siguiente.
		
		\begin{enumerate}
			\item Demuestra que si $0 < a < 2$, entonces $a < \sqrt{2a} < 2$.
			
			\begin{proof}
				Veamos la primera desigualdad,

				\begin{align*}
					& a < 2\\
					\Rightarrow & \text{al ser $a > 0$.}\\
					& a^2 < 2a
					\Rightarrow & \text{al ser $a > 0$.}\\
					& a < \sqrt{2a}.
				\end{align*}


				Ahora la segunda

				\begin{align*}
					& a < 2\\
					\Rightarrow\\
					& 2a < 4
					\Rightarrow & \text{al ser $a > 0$.}\\
					& \sqrt{2a} < 2.
				\end{align*}

			\end{proof}

			\item Demuestra que la suceción
			
			\[\sqrt{2},\sqrt{\sqrt{2}},\sqrt{\sqrt{\sqrt{2}}},\cdots\]

			converge.

			\begin{proof}
				Sea $b_1 = \sqrt{2}, b_{n+1}  =  \sqrt{2\cdot b_n}$ la cual es la descrita anteriormente.
    
				Primero veamos que la sucesión esta acotada por $2$. Procederemos por inducción
				
				\begin{itemize}
					\item \textbf{Caso base: n = 1}
					\[b_1 = \sqrt{2}\]
					Por orden de los naturales sabemos
					\[2 < 4\]
					Después por $b > a > 0 \implies b^q > a^q > 0, q \in \mathbb{Q}$ entonces tenemos
					\[\sqrt{2} < 2\]
					\item \textbf{Hipótesis de inducción}
					\[b_n < 2\]
					\item \textbf{Paso Inductivo}
					Por hipótesis de inducción sabemos
					\[b_n < 2\]
					Luego multiplicando por $2$ obtenemos
					\[2b_n < 4\]
					Después por $b > a > 0 \implies b^q > a^q > 0, q \in \mathbb{Q}$ entonces tenemos
					\[\sqrt{2b_n} < 2\]
					Por definición de la sucesión ${b_n}$ esto es
					\[b_{n+1} < 2\]
				\end{itemize}
				
				Con lo que concluimos
				\[b_n < 2, \forall n \in \mathbb{N}\]
				
				Luego veamos que la suseción es creciente
				
				Por lo demostrado anterior sabemos
				\[b_n < 2\]
				Luego si multiplicamos por $b_n$ obtenemos
				\[b_n^2 < 2b_n\]
				Después por $b > a > 1 \implies b^r > a^r > 1, r \in \R$ entonces tenemos
				\[b_n < \sqrt{2b_n}\]
				Por definición de la sucesión ${b_n}$ concluimos
				\[b_n < b_{n+1}\]
				
				Luego como la sucesión ${b_n}$ es monótona y acotada entonces
				\[\exists \lim_{n\to\infty} b_n\]

			\end{proof}

			\item Calcule el límite. 
				
				Ahora por definición de la sucesión ${b_n}$
				\[\lim_{n\to\infty} b_n = \lim_{n\to\infty} \sqrt{2b_{n-1}}\]
				
				Después por $\lim a_n = \lim a_{n-1}$ entonces tenemos
				\[\lim_{n\to\infty} b_n = \lim_{n\to\infty} \sqrt{2b_n}\]
				
				Elevando al cuadrado obtenemos
				\[(\lim_{n\to\infty} b_n)^2 = \lim_{n\to\infty} 2b_n\]
				
				Como $\lim_{n\to\infty} c(a_n) = c \lim_{n\to\infty} a_n$ esto es
				\[(\lim_{n\to\infty} b_n)^2 = 2 \lim_{n\to\infty} b_n\]
				
				Dividiendo por $\lim_{n\to\infty} b_n$ (ya que $b_n > 0$, esto por que es creciente y $b_1 = \sqrt{2} > 0$) concluimos
				\[\lim_{n\to\infty} b_n = 2\].

		\end{enumerate}

		\item Integración por sustitución.
        
        \begin{enumerate}
            \item $\int \frac{dx}{\sqrt{1+e^x}}$
			
			Sea $u = e^x$ entonces $du = e^xdx$, sustituyendo obtenemos

			\[\int \frac{dx}{\sqrt{1+e^x}} = \int \frac{du}{u\sqrt{1+u}}.\]

			Sea $v = \sqrt{1+u}$ entonces $dv = \frac{du}{2\sqrt{1+u}}$, sustituyendo obtenemos

			\[\int \frac{dx}{\sqrt{1+e^x}} = \int \frac{2vdv}{(v^2-1)v} = \int \int \frac{2dv}{v^2-1} = \int \frac{2dv}{(v+1)(v-1)}.\]

			Por el metodo de fracciones parciales obtenemos

			\[\int \frac{dx}{\sqrt{1+e^x}} = \int \left(\frac{1}{(v-1)} - \frac{1}{(v+1)}\right)dv = \log(v-1) - \log(v+1).\]

			Por lo tanto concluimos

			\[\int \frac{dx}{\sqrt{1+e^x}} = \log(\sqrt{1+e^x}-1) - \log(\sqrt{1+e^x}+1).\]

            \item $\int \frac{dx}{2+\tan(x)}$
			
			Sea $u = \tan(x)$ entonces $x = \tan^{-1}(u)$ y $dx = \frac{du}{1+u^2}$, por lo que sustituyendo obtenemos

			\[\int \frac{dx}{2+\tan(x)} = \int \frac{du}{(1+u^2)(2+u)}\]

			Luego por fracciones parciales vemos que

			\[\int \frac{dx}{1+e^x} = \frac{1}{5}\int \left(\frac{1}{2+u} - \frac{u-2}{1+u^2}\right)du = \frac{log(2+u)}{5} - \frac{log(u^2+1)}{10} + \frac{2\tan^{-1}(u)}{5}.\]

			Por lo tanto concluimos

			\[\int \frac{dx}{1+e^x} = \frac{log(2+\tan(x))}{5} - \frac{log(\sec(u)^2)}{10} + \frac{2x}{5}.\]

            \item $\int \frac{dx}{\sqrt{\sqrt{x+1}}}$

			Sea $u = \sqrt{\sqrt{x}+1}$ entonces $x = (u^2 - 1)^2$ y por regla de la cadena $dx = 4u(u^2-1)du$, sustituyendo en la integral original obtenemos

			\[\int \frac{dx}{\sqrt{\sqrt{x}}+1} = \int \frac{4u(u^2-1)du}{u} = \int 4u^2du - \int 4du = \frac{4u^3}{3} - 4u.\]

			Por lo tanto concluimos que

			\[\int \frac{dx}{\sqrt{\sqrt{x}}+1} = \frac{4\left(\sqrt{\sqrt{x}+1}\right)^3}{3} - 4\left(\sqrt{\sqrt{x}+1}\right).\]

        \end{enumerate}
		
		\item Una sustancia radiactiva disminuye a un ritmo proporcional a la cantidad que de ella queda (puesto que todos los átomos tienen la misma probabilidad de desintegrarse, la desintegración total es proporcional al número de átomos remanentes). Si $A(t)$ es la cantidad en el tiempo t, esto significa que $A'(t) = cA(t)$ para algún c (el cual representa la probabilidad de que se desintegre	un átomo).
		
		\begin{enumerate}
			\item Halla $A(t)$ en términos de la cantidad $A_0 = A(0)$ presente
			en el tiempo 0.

			Por la tare pasada sabemos que existe un $k$ tal que

			\[A(t) = ke^{ct},\]

			evualando en $0$ obtenemos

			\[A(0) = ke^{c0} = k,\]

			por lo tanto

			\[A_0 = k\]

			y

			\[A(t) = A_0e^{ct}.\]

			\item  Demuestra que existe un número $\tau$ (la "vida media" del
			elemento radiactivo) con la propiedad de que $A(t + \tau) = A(t)/2$.

			\begin{proof}

				Sea $\tau = \frac{\log(\frac{1}{2})}{c}$, entonces

				\[e^{c\tau} = \frac{1}{2},\]

				por lo que

				\[A_0e^{c(t + \tau)} = \frac{A_0e^{ct}}{2},\]

				por lo tanto

				\[A(t + \tau) = \frac{A(t)}{2}.\]

			\end{proof}

		\end{enumerate}

	\end{enumerate}
	
\end{document}
