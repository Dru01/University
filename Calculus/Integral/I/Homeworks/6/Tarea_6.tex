\documentclass[letterpaper]{article}
\usepackage[utf8]{inputenc}

\usepackage[margin=1.3in]{geometry}

\usepackage{float}
\usepackage{bbold}
\usepackage{eufrak}
\usepackage{calligra}
\usepackage{aurical}
\usepackage{frcursive}
\usepackage{mathrsfs}
\usepackage{amssymb}
\usepackage{latexsym}

% Formato {
    \usepackage{framed}
    \usepackage{titlesec}
    \usepackage{eufrak}
% }
    
% Ambientes {
    \usepackage{amsthm,amsmath, amssymb}
    \usepackage{etoolbox}
    
    % Indentación {
        %puse estas llaves para que puedan comprimir esto y no les estorbe
        \makeatletter
        \patchcmd{\@thm}
          {\trivlist}
          {\list{}{\leftmargin=\thm@leftmargin\rightmargin=\thm@rightmargin}}
          {}{}
        \patchcmd{\@endtheorem}
          {\endtrivlist}
          {\endlist}
          {}{}
        \newlength{\thm@leftmargin}
        \newlength{\thm@rightmargin}
        
        \newcommand{\xnewtheorem}[3]{%
          \newenvironment{#3}
            {\thm@leftmargin=#1\relax\thm@rightmargin=#2\relax\begin{#3INNER}}
            {\end{#3INNER}}%
            
          \theoremstyle{definition}%
          \newtheorem{#3INNER}%
        }
        \makeatother
    % }
    
    % Ejercicio {
        % Uso: \ejercicio[Título opcional]{Enunciado del teorema}
        \theoremstyle{definition}
        \newtheorem{ejerciciothm}{Ejercicio}
        \newcommand{\ejercicio}[2][]{
            \begin{ejerciciothm}[#1]
                 #2
            \end{ejerciciothm}
        }
    % }
    
    % Inciso {
        \makeatletter
            \newcounter{subproblem}[ejerciciothm]
            \newenvironment{subproblem}
                {\hfill\refstepcounter{subproblem}\par\noindent
                \thesubproblem\normalsize}
                {\vspace{0.5mm}}
            \renewcommand\thesubproblem{\alph{subproblem})}
        \makeatother
        \newcommand{\inciso}[1]{
            \begin{subproblem}
                #1
            \end{subproblem}
        }
    % }

    % Lema {
        % Uso: \lema[Título opcional]
        \theoremstyle{definition}
        \xnewtheorem{2.5em}{2.5em}{lemathm}{Lema}
        \newcommand{\lema}[2][]{
            \begin{lemathm}[#1]
                 #2
            \end{lemathm}
        }
    % }
% }

%Definir comandos{
    \newcommand{\N}{ \mathbb{N} }
    \newcommand{\R}{ \mathbb{R} }
    \newcommand{\Pn}{ \mathbb{P} }
    \newcommand{\Aes}{ \mathcal{A} }
    \newcommand{\ToInf}{ \rightarrow \infty}
    %\QEDA para hacer cuadrito de QED.
    \newcommand*{\QEDA}{\hfill\ensuremath{\blacksquare}}
    \newcommand{\ci}{\left[}
    \newcommand{\cd}{\right]}
    \newcommand{\pareni}{\left(}
    \newcommand{\parend}{\right)}
    \newcommand{\comillas}{``}
    %\newcommand{\ls}{\limsup\limits_{n\rightarrow \infty}}
    %\newcommand{\li}{\liminf\limits_{n\rightarrow \infty}}
	\makeatletter
        \newcommand{\titulo}{\@title}
    \makeatother
	
	% Para Santiago
	\usepackage{etoolbox}

    \newcommand\ls[1][]{%
      \ifstrempty{#1}{%
        \limsup_{n\to\infty}
      }{%
        \limsup_{#1\to\infty}
      }%
    }

    \newcommand\li[1][]{%
      \ifstrempty{#1}{%
        \liminf_{n\to\infty}
      }{%
        \liminf_{#1\to\infty}
      }%
    }
    
    % Nuria, aquí están los comandos que pusiste. Atte. Nuria
    \renewcommand*{\qedhere}{\tag*{$\square$}}
    \newcommand*{\QED}{\hfill\ensuremath{\blacksquare}}
    \newcommand*{\QEDHERE}{\tag*{$\blacksquare$}}
    
    \renewcommand*{\P}{ \mathbb{P} }
    
	\newcommand{\To}{\longrightarrow}
	\newcommand{\mTo}{\longmapsto}
	
% 	\newcommand{\txty}{\quad\text{y}\quad}
% 	\newcommand{\Txty}{\qquad\text{y}\qquad}
% }

\title{Cálculo II\\Tarea 06}
\author{Rubén Pérez Palacios\\Profesor: Dr. Fernando Nuñez Medina}
\date{25 Febrero 2020}

\begin{document}

\maketitle

\section*{Problemas}

\begin{enumerate}
    
  \item Halla
  
  \[\int_{a}^{b} \frac{f'(t)}{f(t)}dt\]

  (para $f > 0$ en $[a,b]$).

  Fijemonos en la derivadad de la composición $log \circ f$, vista en la tarea pasada,

  \[(log \circ f)'(x) = \frac{f'(x)}{f(x)}.\]

  Luego $(log \circ f)(x)$ es primitiva de $\frac{f'(x)}{f(x)}$, por el segundo teorema fundamental del cálcuo conlcuimos que

  \[\int_{a}^{b} \frac{f'(x)}{f(x)} = (log \circ f)(b) - (log \circ f)(a) = \log\left(\frac{f(b)}{f(a)}\right).\]

  \item Calcula las siguientes integrales

  \begin{enumerate}
    \item $\int \frac{dx}{\sqrt{1-x^2}}$
    
    Sea $\sin(u) = x$ entonces $dx = cos(u) du$, por lo tanto

    \[\int \frac{dx}{\sqrt{1-x^2}} = \int \frac{cos(u)du}{\sqrt{1-\sin^2(x)^2}} = \int du = u = sin^{-1}(x).\]

    \item $\int \frac{dx}{\sqrt{1+x^2}}$
    
    Sea $\tan(u) = x$ entonces $dx = sec(u)^2 du$, por lo tanto

    \[\int \frac{dx}{\sqrt{1+x^2}} = \int \frac{sec(u)^2du}{\sqrt{1+\tan(x)^2}} =\int sec(u)du = log(sec(u) + tan(u))\]\[ = log(\sqrt{1+x^2} + x^2).\]

    \item $\int \frac{dx}{\sqrt{x^2-1}}$
    
    Sea $sec(u) = x$ entonces $dx = sec(u)tan(u)du$, por lo tanto

    \[\int \frac{dx}{\sqrt{x^2-1}} = \int \frac{sec(u)tan(u)du}{tan(u)} = \int sec(u)du = log(sec(u) + tan(u))\]\[ = log(x + \sqrt{x^2-1}).\]

  \end{enumerate}

  \item Calcule las siguientes integrales
  
  \begin{enumerate}
    \item $\int \frac{2x^2+7x-1}{x^3+x^2+x-1}dx$
    
    Expresaremos lo anterior por fracciones parciales

    \begin{align*}
      \frac{2x^2+7x-1}{x^3+x^2+x-1} &= \frac{2x^2+7x-1}{(x+1)^2(x-1)}\\
      &= \frac{A}{(x+1)^2 }+ \frac{B}{x+1} + \frac{C}{x-1}\\
      &= \frac{A(x-1) + B(x^2-1) + C(x+1)^2}{x^3+x^2+x-1}\\
      &= \frac{(B+C)x^2 + (A+2C)x + (C-B-A)}{x^3+x^2+x-1}.
    \end{align*}

    De lo que obtenemos el siguiente sistema de ecuaciones

    \begin{align*}
      B + C &= 2\\
      A + 2C &= 7\\
      A + B - C &= 1,
    \end{align*}

    resolviendo el sistema de ecuaciones obtenemos

    \[A = 3 \quad B = 0 \quad C = 2\].

    Sustituyendo en la integral tenemos

    \begin{align*}
      \int \frac{2x^2+7x-1}{x^3+x^2+x-1}dx &= \int \left(\frac{3}{(x+1)^2}+ \frac{0}{x+1} + \frac{2}{x-1}\right)dx\\
      &= 3 \int \frac{2dx}{(x+1)^2} + 2\int \frac{dx}{x-1}\\
      &= -\frac{3}{(x+1)} + 2log(x-1)
    \end{align*}

    \item $\int \frac{2x+1}{x^3-3x^2+3x-1}dx$
    
    Expresaremos lo anterior por fracciones parciales

    \begin{align*}
      \frac{2x+1}{x^3-3x^2+3x-1} &= \frac{2x^2+7x-1}{(x-1)^3}\\
      &= \frac{A}{(x-1)^3}+ \frac{B}{(x-1)^2} + \frac{C}{x-1}\\
      &= \frac{A + B(x-1) +  C(x-1)^2}{(x-1)^3}\\
      &= \frac{Cx^2 + (B-2C)x + (A-B+C)}{(x-1)^3}.
    \end{align*}

    De lo que obtenemos el siguiente sistema de ecuaciones

    \begin{align*}
      C &= 0\\
      B - 2C &= 2\\
      A - B + C &= 1,
    \end{align*}

    resolviendo el sistema de ecuaciones obtenemos

    \[A = 3 \quad B = 2 \quad C = 0\].

    Sustituyendo en la integral tenemos

    \begin{align*}
      \int \frac{2x+1}{x^3-3x^2+3x-1}dx &= \int \left(\frac{3}{(x-1)^3}+ \frac{2}{(x-1)^2} + \frac{0}{x-1}\right)dx\\
      &= 3\int \frac{dx}{(x-1)^3} + 2\int \frac{dx}{(x-1)^2}\\
      &= -\frac{3}{2(x-1)^2} - \frac{2}{x-1}\\
      &= \frac{1-4x}{2(x-1)^2}
    \end{align*}
    
    \item $\int \frac{x^3+7x^2-5x+5}{(x-1)^2(x+1)^3}dx$
    
    Expresaremos lo anterior por fracciones parciales

    \begin{align*}
      &\frac{x^3+7x^2-5x+5}{(x-1)^2(x+1)^3}\\
      &= \frac{A}{(x-1)^2}+ \frac{B}{x-1} + \frac{C}{(x+1)^3} + \frac{D}{(x+1)^2} + \frac{E}{x+1}\\
      &= \frac{A(x+1)^3 + B(x-1)(x+1)^3 + C(x-1)^2 + D(x-1)^2(x+1) + E(x-1)^2(x+1)^2}{(x-1)^2(x+1)^3}\\
      &= \frac{ (B+E)x^4 + (A+2B+D)x^3 + (3A+C-D-2E)x^2}{(x-1)^2(x+1)^3}\\
      &+ \frac{(3A-2B-2C-D)x + (A-B+C+D+E) }{(x-1)^2(x+1)^3}.
    \end{align*}

    De lo que obtenemos el siguiente sistema de ecuaciones

    \begin{align*}
      B+E&=0\\
      A+2B+D&=1\\
      3A+C-D-2E&=7\\
      3A-2B-2C-D&=-5\\
      A-B+C+D+E&=5\\
    \end{align*}

    resolviendo el sistema de ecuaciones obtenemos
    
    \begin{align*}
      A&=1, & B&=0, & C&=4, & D&=0, & &\text{y}& E&=0.
    \end{align*}
    
    Sustituyendo en la integral tenemos

    \begin{align*}
      &\int \frac{x^3+7x^2-5x+5}{(x-1)^2(x+1)^3}dx\\
      &= \int \left(\frac{1}{(x-1)^2}+ \frac{0}{(x-1)^2} + \frac{4}{(x+1)^3} + \frac{0}{(x+1)^2} + \frac{0}{x+1}\right)dx\\
      &= \int\frac{dx}{(x-1)^2} + 4\int\frac{dx}{(x+1)^3}\\
      &= -\frac{1}{x-1} - \frac{2}{(x+1)^2}
    \end{align*}

  \end{enumerate}

  \item Calcula las siguientes integrales
  
  \begin{enumerate}
    \item $\int \frac{a^x}{b^x}dx$
    
    Veamos que $\frac{a^x}{b^x}dx = \left(\frac{a}{b}\right)^xdx$, por lo tanto

    \[\int \frac{a^x}{b^x}dx = \int \left(\frac{a}{b}\right)^xdx = \frac{\left(\frac{a}{b}\right)^x}{log(\frac{a}{b})}\]

    \item $\int tan(x)^2dx$
    
    Veamos lo siguiente

    \[\int tan(x)^2dx = \int sex(x)^2dx - \int dx = tan(x) - x\]

    \item $\int frac{dx}{a+x^2}dx$
    
    Si $a = 0$ entonces $\int frac{dx}{a+x^2}dx = -\frac{1}{x}$, si $a \neq 0$ entonces

    \[\int frac{dx}{a+x^2}dx = \frac{1}{a} \int \frac{1}{a} \frac{dx}{1 + \left(\frac{x}{a}\right)^2} = -\frac{arctan(x/a)}{a}\]
  \end{enumerate}

  \item Calcula las siguientes integrales
  
  \begin{enumerate}
    \item $\int \frac{e^xdx}{e^{2x} + 2e^x + 1}$
    
    Sea $u = e^x$, entonces $du = e^xdx$, por lo tanto

    \[\int \frac{e^xdx}{e^{2x} + 2e^x + 1} = \int \frac{du}{(u + 1)^2} = -\frac{1}{u+1} = -\frac{1}{e^x+1}\]

    \item $\int e^{e^x} e^x dx$
    
    Sea $u = e^x$, entonces $du = e^x dx$, por lo tanto

    \[\int e^{e^x} e^x dx = \int e^u du = e^u = e^{e^x}\]

    \item $\int \frac{xdx}{\sqrt{1-x^4}}$
    
    Sea $u = x^2$, entonces $du = 2xdx$, y sea $sin(v) = u$ entonces $du = cos(v)dv$. Veamos lo siguiente

    \[\frac{1}{2} \int \frac{2xdx}{\sqrt{1-x^4}} = \frac{v}{2}\]

    Por lo tanto

    \[\int \frac{xdx}{\sqrt{1-x^4}} = \frac{\sin^{-1}(x^2)}{2}\]

  \end{enumerate}

  \item Integración por partes
  
  \begin{enumerate}
    \item $\int x^2sin(x)dx$
    
    Sea $u = x^2$ y $v = -cos(x)$, entonces $du = 2xdx$ y $dv = sen(x)dx$, por lo tanto

    \[\int x^2sin(x)dx = -x^2cos(x) - 2\int cos(x)xdx\]

    Sea $u = x$ y $v = sen(x)$, entonces $du = dx$ y $dv = -cos(x)dx$, por lo tanto

    \[\int x^2sin(x)dx = -x^2cos(x) + 2\left(xsen(x) - \int sin(x)dx\right) = cos(x)(2-x^2) + 2xsen(x)\]

    \item $log(x)^3dx$
    
    Sea $u = log(x)^3$ y $v = x$, entonces $du = 3log(x)^2(1/x)dx$ y $dv = dx$, por lo tanto

    \[\int log(x)^3dx = log(x)^3x - 3\int log(x)^2dx\]

    Sea $u = log(x)^2$ y $v = x$, entonces $du = 2log(x)(1/x)dx$ y $dv = dx$, por lo tanto

    \[\int log(x)^3dx = log(x)^3x - 3\left(log(x)^2x - 2\int log(x)dx\right)\]

    Sea $u = log(x)$ y $v = x$, entonces $du = (1/x)dx$ y $dv = dx$, por lo tanto

    \[\int log(x)^3dx = log(x)^3x - 3\left(log(x)^2x - 2\left(log(x)x - \int dx\right)\right)\]\[ = log(x)^3x -3log(x)^2x + 6log(x)x - 6x\]

    \item $\int \frac{log(log(x))}{x}dx$
    
    Sea $u = log(x)$ entonces $du = dx/x$ por lo tanto

    \[\int \frac{log(log(x))}{x}dx = \int log(u)du\]

    Sea $v = log(x)$ y $w = x$, entonces $du = (1/x)dx$ y $dv = du$, por lo tanto

    \[\int \frac{log(log(x))}{x}dx = log(u)u - \int du = (log(u)-1)u\]

    Con lo que concluimos que

    \[\int \frac{log(log(x))}{x}dx = log(u)u - \int du = (log(log(x))-1)log(x)\]

  \end{enumerate}
\end{enumerate}

\end{document}