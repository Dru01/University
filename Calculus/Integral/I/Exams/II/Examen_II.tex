% PREÁMBULO
\documentclass[letterpaper]{article}
\usepackage[utf8]{inputenc}
\usepackage[spanish]{babel}

\usepackage{enumitem}
\usepackage{titling}

% Símbolos
	\usepackage{amsmath}
	\usepackage{amssymb}
	\usepackage[utf8]{inputenc}
	\usepackage[T1]{fontenc}
	\usepackage{mathtools}
	\usepackage[thinc]{esdiff}

% Márgenes
	\usepackage
	[
		margin = 1.4in
	]
	{geometry}

% Imágenes
	\usepackage{float}
	\usepackage{graphicx}
	\graphicspath{{imagenes/}}
	\usepackage{subcaption}

% Ambientes
	\usepackage{amsthm}

	\theoremstyle{definition}
	\newtheorem{ejercicio}{Ejercicio}

	\newtheoremstyle{lemathm}{4pt}{0pt}{\itshape}{0pt}{\bfseries}{ --}{ }{\thmname{#1}\thmnumber{ #2}\thmnote{ (#3)}}
	\theoremstyle{lemathm}
	\newtheorem{lema}{Lema}

	\newtheoremstyle{lemademthm}{0pt}{10pt}{\itshape}{ }{\mdseries}{ --}{ }{\thmname{#1}\thmnumber{ #2}\thmnote{ (#3)}}
	\theoremstyle{lemademthm}
	\newtheorem*{lemadem}{Demostración}

% Ajustes
	\allowdisplaybreaks	% Los align pueden cambiar de página

% Macros
	\newcommand{\sumi}[2]{\sum_{i=#1}^{#2}}
	\newcommand{\dint}[2]{\displaystyle\int_{#1}^{#2}}
	\newcommand{\inte}[2]{\int_{#1}^{#2}}
	\newcommand{\dlim}{\displaystyle\lim}
	\newcommand{\limxinf}{\lim_{x\to\infty}}
	\newcommand{\limninf}{\lim_{n\to\infty}}
	\newcommand{\dlimninf}{\displaystyle\lim_{n\to\infty}}
	\newcommand{\limh}{\lim_{h\to0}}
	\newcommand{\ddx}{\dfrac{d}{dx}}
	\newcommand{\txty}{\text{ y }}
	\newcommand{\txto}{\text{ o }}
	\newcommand{\Txty}{\quad\text{y}\quad}
	\newcommand{\Txto}{\quad\text{o}\quad}
	\newcommand{\si}{\text{si}\quad}

	\newcommand{\etiqueta}{\stepcounter{equation}\tag{\theequation}}
	\newcommand{\tq}{:}
	\renewcommand{\o}{\circ}
	% \newcommand*{\QES}{\hfill\ensuremath{\boxplus}}
	% \newcommand*{\qes}{\hfill\ensuremath{\boxminus}}
	% \newcommand*{\qeshere}{\tag*{$\boxminus$}}
	% \newcommand*{\QESHERE}{\tag*{$\boxplus$}}
	\newcommand*{\QES}{\hfill\ensuremath{\blacksquare}}
	\newcommand*{\qes}{\hfill\ensuremath{\square}}
	\newcommand*{\QESHERE}{\tag*{$\blacksquare$}}
	\newcommand*{\qeshere}{\tag*{$\square$}}
	\newcommand*{\QED}{\hfill\ensuremath{\blacksquare}}
	\newcommand*{\QEDHERE}{\tag*{$\blacksquare$}}
	\newcommand*{\qel}{\hfill\ensuremath{\boxdot}}
	\newcommand*{\qelhere}{\tag*{$\boxdot$}}
	\renewcommand*{\qedhere}{\tag*{$\square$}}

	\newcommand{\abs}[1]{\left\vert#1\right\vert}
	\newcommand{\suc}[1]{\left(#1_n\right)_{n\in\N}}
	\newcommand{\en}[2]{\binom{#1}{#2}}
	\newcommand{\upsum}[2]{U(#1,#2)}
	\newcommand{\lowsum}[2]{L(#1,#2)}

	\newcommand{\N}{\mathbb{N}}
	\newcommand{\Q}{\mathbb{Q}}
	\newcommand{\R}{\mathbb{R}}
	\newcommand{\Z}{\mathbb{Z}}
	\newcommand{\eps}{\varepsilon}
	\newcommand{\ttF}{\mathtt{F}}
	\newcommand{\bfF}{\mathbf{F}}

	\newcommand{\To}{\longrightarrow}
	\newcommand{\mTo}{\longmapsto}
	\newcommand{\ssi}{\Longleftrightarrow}
	\newcommand{\sii}{\Leftrightarrow}
	\newcommand{\then}{\Rightarrow}

	\newcommand{\pTFC}{{\itshape 1er TFC\/}}
	\newcommand{\sTFC}{{\itshape 2do TFC\/}}

% Membrete
	\usepackage{fancyhdr}
	\setlength{\headheight}{14pt}
	\pagestyle{fancy}
	\fancyhf{}
	\rhead{\theauthor}
	\lhead{\thetitle}
	\cfoot{\thepage}

% Datos
    \title{Cálculo II\\Examen II}
    \author{Rubén Pérez Palacios\\Profesor: Dr. Fernando Nuñez Medina}
    \date{30 Abril 2020}

% DOCUMENTO
\begin{document}
	\maketitle

	\section*{Teoremas}

	\begin{enumerate}
		\item Sea $\{a_n\}$ una sucesión con $a_n \in \R$ y $f: \R \rightarrow \R$ tal que $f(n) = a_n$, luego sí  existe el $\limxinf f(x)$ entonces
		
		\[\limxinf f(x) = \limninf a_n.\]

		\begin{proof}
			Al existir el límite de $f$ entonces $\limxinf f(x) = L$ y por definición de limite tenemos que

			\[\forall \varepsilon > 0 \exists c \in \R : \forall x > c \text{ se cumple que } |f(x) - L| < \varepsilon.\]

			Luego como $f(n) = a_n$ y por propiedad arquimediana $\exists N \in N$ tal que $N > c$ entonces

			\[\forall \varepsilon > 0 \exists N \in \N : \forall n > N \text{ se cumple que } |f(n) - L| < \varepsilon.\]

			por definición de límite concluimos que existe el límite de $\{a_n\}$ y que

			\[\limxinf f(x) = \limninf a_n.\]

		\end{proof}

		\newpage

		\item El polinomio de taylor en $a$ de grado $n$ de $f$ es el polinomio en $(x-a)$ de $f$.
		
		\begin{proof}
			
			Por deifnición sabemos que
			
			\[f(x) = P_{n,a}(x) + R_{n,a}(x).\]
			
			Luego si $f$ es una función polinomial de grado $n$ entonces $f^{(n+1)} = 0$, esto lo demostraremos con inducción
			
			\begin{itemize}
				\item \textbf{Caso base: n = 0}
				Entonces $f$ es de la forma $f(x) = c$ y por lo tanto $f'(x) = 0$.
				\item \textbf{Hipótesis:}
				Si $f$ es una función polinomial de grado $n$ entonces $f^{(n+1)} = 0$	
				\item \textbf{Paso inductivo}
				Sea $f$ una función polinomial de grado $n+1$ entonces $f$ es de la forma 
				
				\[f(x) = \sum_{i=0}^{n+1} a_ix^i\]
				
				entonces 
				
				\[f(x) = \sum_{i=0}^{n+1} a_ix^i + a_{n+1}x^{n+1},\]

				por lo que 
				
				\[f(x) = g(x) + a_{n+1}x^{n+1}\]
				
				donde $g(x)$ es una función polinomial de grado $n$.
				
				Ahora por linealidad de la derivada y por hipotesis de inducción
				
				\[f^{(n+1)} = \frac{a_{n+1}}{n+1}.\]
				
				por lo tanto
				
				\[f^{(n+2)} = 0.\]
				
			\end{itemize}

			Por inducción matemática concluimos que si $f$ es una función polinomial de grado $n$ entonces $f^{(n+1)} = 0$.
				
			Por el teorema de Taylor concluimos que si $f$ es un función polinomial de grado $n$ entonces (ya que $R_{n,a}(x) = 0$)

			\[f(x) = P_{n,a}(x).\]
			
			Es decir el polinomio de taylor en $a$ de grado $n$ de $f$ es el polinomio en $(x-a)$ de $f$.

		\end{proof}

	\end{enumerate}

	\newpage

    \section*{Problemas}

	\begin{enumerate}
		
		\item Realiza lo siguiente
		
		\begin{enumerate}
			\item Calcula $\int \frac{x}{1+x^2}dx$.

			Sea $u = x^2 + 1$ entonces $du = 2xdx$, sustituyendo en la integral obtenemos

			\[\int \frac{x}{1+x^2}dx = \int \frac{du}{2u}.\]

			Al ser $\int \frac{du}{u} = \log(u)$ entonces

			\[\int \frac{x}{1+x^2}dx = \frac{\log(u)}{2},\]

			por lo tanto concluimos que

			\[\int \frac{x}{1+x^2}dx = \frac{\log(x^2+1)}{2}.\]

			\item Calcula $\int \tan^{-1}(x)dx$.
			
			Integraremos por partes, entonces sea $u = \tan{-1}(x)$ y $dv = dx$ entonces $du = \frac{1}{x^2+1}dx$ y $v = x$; por lo tanto

			\[\int \tan^{-1}(x)dx = \tan^{-1}(x)x - \int \frac{x}{x^2+1} dx\]

			por el inciso anterior concluimos

			\[\int \tan^{-1}(x)dx = \tan^{-1}(x)x - \frac{\log(x^2+1)}{2}\]

		\end{enumerate}

		\item Escribe el polinomio $f(x) = x^5$ como polinomio en $(x-3)$.
		
		Por el teorema 2 es suficiente calcular el polinomio de Taylor en 3 de grado 5.Empezaremos calculando las derivadas necesarias para el polinomio de Taylor:

		\begin{align*}
			f^{(0)}(x) &= x^5\\
			f^{(1)}(x) &= 5x^4\\
			f^{(2)}(x) &= 20x^3\\
			f^{(3)}(x) &= 60x^2\\
			f^{(4)}(x) &= 120x\\
			f^{(5)}(x) &= 120\\
		\end{align*}

		Las evaluamos en $3$ y obtenemos
		
		\begin{align*}
			f^{(0)}(3) &= 243\\
			f^{(1)}(3) &= 405\\
			f^{(2)}(3) &= 540\\
			f^{(3)}(3) &= 540\\
			f^{(4)}(3) &= 360\\
			f^{(5)}(3) &= 120\\
		\end{align*}

		Por lo tanto

		\[f(x) = P_{5,3}(x) = \frac{243}{0!} + \frac{405}{1!}(x-3) + \frac{540}{2!}(x-3)^2 + \frac{540}{3!}(x-3)^3 + \frac{360}{4!}(x-3)^4 + \frac{120}{5!}(x-3)!\]
		\[= 243 + 405(x-3) + 270(x-3)^2 + 90(x-3)^3 + 15(x-3)^4 + (x-3)^5.\]

		\item Comprueba que $\limninf \sqrt[n]{n^2+n} = 1$.
		
		\begin{proof}
			Primero veamos que por definición

			\[\sqrt[n]{n^2+n} = \exp(\log(\sqrt[n]{n^2+n})),\]
			
			entonces

			\[\sqrt[n]{n^2+n} = \exp\left(\frac{\log(n^2+n)}{n}\right),\]

			y como $n^2 + n = n (n+1)$ obtenemos

			\[\sqrt[n]{n^2+n} = \exp\left(\frac{\log(n+1) + \log(n)}{n}\right).\]

			Ahora veamos el siguiente límite

			\[\limxinf \frac{log(x)}{x}\]

			por L'Hopital tenemos que

			\[\limxinf \frac{log(x)}{x} = \limxinf \frac{\frac{1}{x}}{1} = \limxinf \frac{1}{x} = 0,\]

			por el teorema 1 se cumple que

			\[\limninf \frac{\log(n)}{n} = 0,\]

			analogamente obtenemos que
			
			\[\limninf \frac{\log(n+1)}{n} = 0.\]

			Por último veamos lo siguiente

			\begin{align*}
				1 &= \exp(0)\\
				&= \exp\left(\limninf \frac{\log(n+1) + \log(n)}{n}\right) & \text{ por lo demostrado anteriormente.}\\
				&= \limninf \exp\left(\frac{\log(n^2+n)}{n}\right) & \text{ al ser exp continua}\\
				&= \limninf \exp(\log(\sqrt[n]{n^2+n}))\\
				&= \limninf \sqrt[n]{n^2+n}
			\end{align*}

		\end{proof}

		\item Demuestra que cualquier subseción de una sucesión convergente es convergente.
		
		\begin{proof}
			Sea $\{a_n\}$ una sucesión convergente con $a_n \in \R$ y $\limninf a_n = l$. Sea $\{a_{n_k}\}$ una subseción de $s_n$. Primero veamos que $n_k \geq k$, proseguiremos por inducción

			\begin{itemize}
				\item \textbf{Caso base: k = 1}
				
				$n_1 \geq 1$, esto pues $n_k \in \N, \forall k$.

				\item \textbf{Hipótesis:}
				
				\[n_k \geq k\]

				\item \textbf{Paso Inductivo:}
				
				Ya que $n_{k+1} > n_k$ y por hipótesis tenemos que

				\[n_{k+1} > k,\]

				por lo tanto

				\[n_{k+1} \geq k + 1.\]

			\end{itemize}

			Por inducción matemática obtenemos que $n_k \geq k$.

			Ahora por definición de convergencia en $\{a_n\}$ tenemos que

			\[\forall \varepsilon > 0, \exists N \in \N : \forall k \in \N, k \geq N \text{ se cumple que } |a_k - l| \leq \varepsilon,\]

			como $n_k \geq k$ entonces

			\[\forall \varepsilon > 0, \exists N \in \N : \forall n_k \in \N, n_k \geq N \text{ se cumple que } |a_{n_k} - l| \leq \varepsilon,\]

			por definición de convergencia concluimos que $a_{n_k}$ converge y que

			\[\limninf a_{n_k} = l.\]

		\end{proof}

		\item Decir si son convergentes o divergentes cada una de las siguien-
		tes series innitas. Sugerencia: Aplica la prueba de la integral para c).

		\begin{enumerate}
			\item $\sum_{n=2}^{\infty} \frac{1}{\log(n)^n}$
			
			Primero veamos que $e < 3$ entonces $e^2 < 9$ y por lo tanto $2 < \log(9)$ al ser una función creciente y además para todo $n > 8$ se cumple que $2 < \log(n)$. Ahora entonces tenemos que para $n > 8$

			\[\frac{1}{\log(n)} < \frac{1}{2},\]

			al ser ambos positivos obtenemos que

			\[\frac{1}{\log(n)^n} < \frac{1}{2^n},\]

			al ser $\limninf \sum_{n=0}^{\infty} \frac{1}{2^n}$ convergente es más igual a $1$ entonces por el test de comparación concluimos que

			\[\sum_{n=2}^{\infty} \frac{1}{\log(n)^n}\]

			converge.

			\item $\sum_{n=1}^{\infty} \frac{n^2}{n^3+1}$
			
			Primero veamos que $n^3 \geq 1, n \in \N$ por lo que

			\[2n^3 \geq n^3 + 1,\]

			al ser $n > 0$ entonces

			\[\frac{n^3}{n^3+1} \geq \frac{1}{2},\]
			
			por lo tanto

			\[\frac{n^2}{n^3+1} \geq \frac{1}{2n}.\]

			Luego como $\limninf \frac{1}{2n}$ diverge por lo que por la condición de la desaparición ("vanish condition") concluimos que

			\[\sum_{n=1}^{\infty} \frac{n^2}{n^3+1}\]

			diverge.

			\newpage

			\item $\sum_{n=2}^{\infty} \frac{1}{n\log(n)}$
			
			Veamos lo siguiente, sea $u = \log(x)$ entonces $du = \frac{dx}{x}$ por lo que

			\[\int \frac{1}{x\log(x)}dx = \int \frac{du}{u} = \log(u) = \log(\log(x)).\]

			Por lo tanto

			\[\limninf \int_{2}^{n} \frac{1}{x\log(x)}dx = \log(\log(N)) - \log(\log(2)).\]

			Podemos ver que $\forall a \in \R$ se cumple que

			\[a = \log(e^a) = \log(\log(e^{e^a}))\]

			Por propiedad arquimediana $\exists n \in \N$ tal que $n > e^{e^a}$ y como $\log$ es creciente entonces $a < \log(\log(n))$. Es decir, $\forall a \in \R$ existe $n\in\N$ tal que

			\[a < \log(\log(n))\]

			y por lo tanto $\log(\log(N))$ diverge y también $\log(\log(N)) - \log(\log(2))$.

			Por criterio de la integral

			\[\sum_{n=2}^{\infty} \frac{1}{n\log(n)}\]

			diverge.

		\end{enumerate}

		\item Demuestra que si $\sum_{n=1}^\infty a_n$ converge absolutamente, entonces
		
		\[\left|\sum_{n=1}^{\infty} a_n\right| \leq \sum_{n=1}^{\infty} |a_n|\]

		Por inducción demostraremos que esto se cumple hasta $N \in \N$ es decir $\left|\sum_{n=1}^{N} a_n\right| \leq \sum_{n=1}^{N} |a_n|$. Procederemos a demostrar por indducción

		\begin{itemize}
			\item \textbf{Caso base: N = 1}
			
			$|a_n| = |a_n|$ lo cual es cierto

			\item \textbf{Hipótesis:}
			
			\[\left|\sum_{n=1}^{N} a_n\right| \leq \sum_{n=1}^{N} |a_n|.\]

			\item \textbf{Paso Inductivo:}
			
			Por la desigualdad del triángulo tenemos que

			\[\left|\sum_{n=1}^{N+1} a_n\right| \leq \left|\sum_{n=1}^{N} a_n\right| + |a_{N+1}|\]

			Por hipótesis de inducción obtenemos

			\[\left|\sum_{n=1}^{N+1} a_n\right| \leq \sum_{n=1}^{N+1} |a_n|.\]

		\end{itemize}

		Por inducción matemática obtenemos que

		\[\left|\sum_{n=1}^{N} a_n\right| \leq \sum_{n=1}^{N} |a_n|,\]

		además por definición $|a_n| \geq 0$ por lo que

		\[\left|\sum_{n=1}^{N} a_n\right| \leq \sum_{n=1}^{\infty} |a_n|.\]

		Ahora como $\sum_{0}^{\infty} a_n$ es absolutamente convergente, $\sum_{0}^{\infty} a_n$ y por la desigualdad concluimos

		\[\left|\sum_{n=1}^{\infty} a_n\right| \leq \sum_{n=1}^{\infty} |a_n|.\]

	\end{enumerate}
	
\end{document}
