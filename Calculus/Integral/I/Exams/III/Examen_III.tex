% PREÁMBULO
\documentclass[letterpaper]{article}
\usepackage[utf8]{inputenc}
\usepackage[spanish]{babel}

\usepackage{enumitem}
\usepackage{titling}

% Símbolos
	\usepackage{amsmath}
	\usepackage{amssymb}
	\usepackage[utf8]{inputenc}
	\usepackage[T1]{fontenc}
	\usepackage{mathtools}
	\usepackage[thinc]{esdiff}

% Márgenes
	\usepackage
	[
		margin = 1.4in
	]
	{geometry}

% Imágenes
	\usepackage{float}
	\usepackage{graphicx}
	\graphicspath{{imagenes/}}
	\usepackage{subcaption}

% Ambientes
	\usepackage{amsthm}

	\theoremstyle{definition}
	\newtheorem{ejercicio}{Ejercicio}

	\newtheoremstyle{lemathm}{4pt}{0pt}{\itshape}{0pt}{\bfseries}{ --}{ }{\thmname{#1}\thmnumber{ #2}\thmnote{ (#3)}}
	\theoremstyle{lemathm}
	\newtheorem{lema}{Lema}

	\newtheoremstyle{lemademthm}{0pt}{10pt}{\itshape}{ }{\mdseries}{ --}{ }{\thmname{#1}\thmnumber{ #2}\thmnote{ (#3)}}
	\theoremstyle{lemademthm}
	\newtheorem*{lemadem}{Demostración}

% Ajustes
	\allowdisplaybreaks	% Los align pueden cambiar de página

% Macros
	\newcommand{\sumi}[2]{\sum_{i=#1}^{#2}}
	\newcommand{\dint}[2]{\displaystyle\int_{#1}^{#2}}
	\newcommand{\inte}[2]{\int_{#1}^{#2}}
	\newcommand{\dlim}{\displaystyle\lim}
	\newcommand{\limxinf}{\lim_{x\to\infty}}
	\newcommand{\limninf}{\lim_{n\to\infty}}
	\newcommand{\dlimninf}{\displaystyle\lim_{n\to\infty}}
	\newcommand{\limh}{\lim_{h\to0}}
	\newcommand{\ddx}{\dfrac{d}{dx}}
	\newcommand{\txty}{\text{ y }}
	\newcommand{\txto}{\text{ o }}
	\newcommand{\Txty}{\quad\text{y}\quad}
	\newcommand{\Txto}{\quad\text{o}\quad}
	\newcommand{\si}{\text{si}\quad}

	\newcommand{\etiqueta}{\stepcounter{equation}\tag{\theequation}}
	\newcommand{\tq}{:}
	\renewcommand{\o}{\circ}
	% \newcommand*{\QES}{\hfill\ensuremath{\boxplus}}
	% \newcommand*{\qes}{\hfill\ensuremath{\boxminus}}
	% \newcommand*{\qeshere}{\tag*{$\boxminus$}}
	% \newcommand*{\QESHERE}{\tag*{$\boxplus$}}
	\newcommand*{\QES}{\hfill\ensuremath{\blacksquare}}
	\newcommand*{\qes}{\hfill\ensuremath{\square}}
	\newcommand*{\QESHERE}{\tag*{$\blacksquare$}}
	\newcommand*{\qeshere}{\tag*{$\square$}}
	\newcommand*{\QED}{\hfill\ensuremath{\blacksquare}}
	\newcommand*{\QEDHERE}{\tag*{$\blacksquare$}}
	\newcommand*{\qel}{\hfill\ensuremath{\boxdot}}
	\newcommand*{\qelhere}{\tag*{$\boxdot$}}
	\renewcommand*{\qedhere}{\tag*{$\square$}}

	\newcommand{\abs}[1]{\left\vert#1\right\vert}
	\newcommand{\suc}[1]{\left(#1_n\right)_{n\in\N}}
	\newcommand{\en}[2]{\binom{#1}{#2}}
	\newcommand{\upsum}[2]{U(#1,#2)}
	\newcommand{\lowsum}[2]{L(#1,#2)}

	\newcommand{\N}{\mathbb{N}}
	\newcommand{\Q}{\mathbb{Q}}
	\newcommand{\R}{\mathbb{R}}
	\newcommand{\Z}{\mathbb{Z}}
	\newcommand{\eps}{\varepsilon}
	\newcommand{\ttF}{\mathtt{F}}
	\newcommand{\bfF}{\mathbf{F}}

	\newcommand{\To}{\longrightarrow}
	\newcommand{\mTo}{\longmapsto}
	\newcommand{\ssi}{\Longleftrightarrow}
	\newcommand{\sii}{\Leftrightarrow}
	\newcommand{\then}{\Rightarrow}

	\newcommand{\pTFC}{{\itshape 1er TFC\/}}
	\newcommand{\sTFC}{{\itshape 2do TFC\/}}

% Membrete
	\usepackage{fancyhdr}
	\setlength{\headheight}{14pt}
	\pagestyle{fancy}
	\fancyhf{}
	\rhead{\theauthor}
	\lhead{\thetitle}
	\cfoot{\thepage}

% Datos
    \title{Cálculo II\\Examen III}
    \author{Rubén Pérez Palacios\\Profesor: Dr. Fernando Nuñez Medina}
    \date{6 Mayo 2020}

% DOCUMENTO
\begin{document}
	\maketitle

    \section*{Problemas}

	\begin{enumerate}
		
		\item Demuestra que una sucesión convergente es siempre acotada.
		\begin{proof}
		
			Sea $a_n$ una sucesión convergente a $l$, por definición tenemos que

			\[\forall \varepsilon > 0, \exists N \in \N : \forall n \in \N, n \geq N \text{ se cumple que } |a_n - l| < \varepsilon.\]

			Sea $\varepsilon = 1$ entonces existe un $N$ tal que para todo $n \in \N, n \geq N$ se cumple que

			\[|a_n - l| < 1,\]

			por la desigualdad del triangulo obtenemos

			\[|a_n| = |a_n - l + l| \leq |a_n - l| + |l| < 1 + |l|.\]

			Ahora sea $M = \sup(|a_1|, \cdots, |a_N|, 1+|l|)$, entonces por definición concluimos que

			\[|a_n| \leq M \quad \forall n,\]

			es decir la sucesión $a_n$ es acotada.

		\end{proof}

		\item Muestra que si $\sum_{n=1}^{\infty} a_n$ converge, entonces sus sumas parciales también.
		
		\begin{proof}
			Sea $a_n$ una sucesión tal que $\sum_{n=1}^{\infty} a_n$ converge, entonces por definición la sumas parciales $s_n$ convergen digamos a $l$. Por el problema anterior concluimos que $s_n$ es acotada.
		\end{proof}

		\newpage

		\item Sea
		
		\[f_n(x) = \frac{e^{-x^2}}{n}, x\in\R\]

		\begin{enumerate}
			\item Determina el límite puntual de $f_n(x)$.
			
			El límite puntal de $f_n(x)$ es
			
			\[\limninf f_n(x) = 0,\]
			
			Esto pues $e^{-x^2}$ con respecto a $n$ es constante y $\limninf \frac{c}{n} = 0$.

			\item Determina si $f_n$ converge uniformemente a la función encontrada en $a)$.
			
			Veamos lo siguiente

			\begin{align*}
				0 &\leq x^2\\
				1 &\leq e^{x^2}\\
				e^{-x^2} &\leq 1\\
				\frac{e^{-x^2}}{n} &\leq \frac{1}{n}\\
			\end{align*}

			y al ser 
			
			\[\frac{e^{-x^2}}{n} > 0\]

			conlcuimos que

			\[|f_n(x)| = \left|\frac{e^{-x^2}}{n}\right| \leq \frac{1}{n}\]

		\end{enumerate}

		\item Sea
		
		\[f(x) = \sum_{n=1}^{\infty} \frac{1}{1+n^2x}, x \in [1,2].\]

		\begin{enumerate}
			\item Determina si $f$ es continua.
			
			Por el problema 5 $f$ es derivable en $[1,2]$, por lo que es continua en $[1,2]$

			\item Determina si $f$ es integrable.
			
			Al ser $f$ continua en $[1,2]$ por el teorema 3 del capítulo 13 del spivak $f$ es integrable en $[1,2]$

		\end{enumerate}

		\newpage

		\item Determina si la función $f$ del ejercicio anterior es derivable.
		
		Sea $f_n(x) = \frac{1}{1+n^2x}$, por regla de la cadena obtenemos

		\[f_n'(x) = -\frac{n^2}{(1+n^2x)^2}.\]

		Puesto que, tanto $f_n$ y $f'$ son composición de funciones derivables entonces estas dos funciones son derivables y por tanto continuas.

		Sea 
		
		\[M_n = \frac{n^2}{(1+n^2)^2},\]

		como $1+n^2\leq1+n^2x$ ya que $x\in[1,2]$ entonces

		\[|f_n'| \leq M_n.\]

		Veamos que

		\[\frac{n^2}{(1+n^2)^2} \leq \frac{n^2}{(n^2)^2} = \frac{1}{n^2}.\]

		Puesto que

		\[\sum_{n=1}^{\infty} \frac{1}{n^2}\]

		converge entonces

		\[\sum_{n=1}^{\infty} \frac{n^2}{(1+n^2)^2}\]

		converge por la prueba de comparación, por lo tanto

		\[\sum_{n = 1}^{\infty} M_n\]

		converge.
		
		Por la prueba m de Weierstrass concluimos que

		\[\sum_{n=1}^{\infty} f_n'\]

		converge uniformemente, por el corolario de la definición de convergencía uniforme y que $f_n'$ son continuas concluimos que

		\[f'(x) = \sum_{n=1}^{\infty} f_n'(x), \forall x \in [1,2].\]

		\item Supon que $\{f_n\}$ es una sucesión de funciones acotadas (no necesariamente continuas) en $[a,b]$ que converge uniformemente hacia $f$ en $[a,b]$. Demuestra que $f$ es acotada en $[a,b]$.
		
		\begin{proof}
			Sea $\{f_n\}$ una sucesión de funciones acotadas (no necesariamente continuas) en $[a,b]$ que converge uniformemente hacia $f$ en $[a,b]$. Por definición de convergencia uniforme tenemos que

			\[\forall \varepsilon > 0, \exists N \in \N : \forall n \in \N, n \geq N \text{ se cumple que } |f(x) - f_n(x)| < \varepsilon.\]

			Sea $\varepsilon = 1$ entonces existe un $N$ tal que para todo $n \in \N, n \geq N$ se cumple que

			\[|f(x) - f_n(x)| < 1,\]

			por la desigualdad del triangulo obtenemos

			\[|f(x)| \leq |f(x) - f_n(x) + f_n(x)| \leq |f(x) - f_n(x)| + |f_n(x)| < 1 + |f_n(x)|,\]
			
			al ser $f_n$ acotada entonces existe un $M_n$ tal que $|f_n| < M_n$ por lo que

			\[|f(x)| \leq 1 + M_n,\]

			en particular

			\[|f(x)| \leq 1 + M_N,\]

			Por lo tanto $f$ es acotada.
		\end{proof}


	\end{enumerate}
	
\end{document}
