% Preámbulo
\documentclass[letterpaper]{article}
\usepackage[utf8]{inputenc}
\usepackage[spanish]{babel}

\usepackage{enumitem}
\usepackage{titling}

% Símbolos
	\usepackage{amsmath}
	\usepackage{amssymb}
	\usepackage{amsthm}
	\usepackage{amsfonts}
	\usepackage{mathtools}
	\usepackage{bbm}
	\usepackage[thinc]{esdiff}
	\allowdisplaybreaks

% Márgenes
	\usepackage
	[
		margin = 1.2in
	]
	{geometry}

% Imágenes
	\usepackage{float}
	\usepackage{graphicx}
	\graphicspath{{imagenes/}}
	\usepackage{subcaption}

% Ambientes
	\usepackage{amsthm}

	\theoremstyle{definition}
	\newtheorem{ejercicio}{Ejercicio}

	\newtheoremstyle{lemathm}{4pt}{0pt}{\itshape}{0pt}{\bfseries}{ --}{ }{\thmname{#1}\thmnumber{ #2}\thmnote{ (#3)}}
	\theoremstyle{lemathm}
	\newtheorem{lema}{Lema}

	\newtheoremstyle{lemathm}{4pt}{0pt}{\itshape}{0pt}{\bfseries}{ --}{ }{\thmname{#1}\thmnumber{ #2}\thmnote{ (#3)}}
	\theoremstyle{lemathm}
	\newtheorem{sol}{Solución}
	
	\newtheoremstyle{lemathm}{4pt}{0pt}{\itshape}{0pt}{\bfseries}{ --}{ }{\thmname{#1}\thmnumber{ #2}\thmnote{ (#3)}}
	\theoremstyle{lemathm}
	\newtheorem{theo}{Teorema}

	\newtheoremstyle{lemademthm}{0pt}{10pt}{\itshape}{ }{\mdseries}{ --}{ }{\thmname{#1}\thmnumber{ #2}\thmnote{ (#3)}}
	\theoremstyle{lemademthm}
	\newtheorem*{lemadem}{Demostración}

% Macros
	\newcommand{\sumi}[2]{\sum_{i=#1}^{#2}}
	\newcommand{\dint}[2]{\displaystyle\int_{#1}^{#2}}
	\newcommand{\inte}[2]{\int_{#1}^{#2}}
	\newcommand{\dlim}{\displaystyle\lim}
	\newcommand{\limxinf}{\lim_{x\to\infty}}
	\newcommand{\limninf}{\lim_{n\to\infty}}
	\newcommand{\dlimninf}{\displaystyle\lim_{n\to\infty}}
	\newcommand{\limh}{\lim_{h\to0}}
	\newcommand{\ddx}{\dfrac{d}{dx}}
	\newcommand{\txty}{\text{ y }}
	\newcommand{\txto}{\text{ o }}
	\newcommand{\Txty}{\quad\text{y}\quad}
	\newcommand{\Txto}{\quad\text{o}\quad}
	\newcommand{\si}{\text{si}\quad}

	\newcommand{\etiqueta}{\stepcounter{equation}\tag{\theequation}}
	\newcommand{\tq}{:}
	\renewcommand{\o}{\circ}
	\newcommand*{\QES}{\hfill\ensuremath{\blacksquare}}
	\newcommand*{\qes}{\hfill\ensuremath{\square}}
	\newcommand*{\QESHERE}{\tag*{$\blacksquare$}}
	\newcommand*{\qeshere}{\tag*{$\square$}}
	\newcommand*{\QED}{\hfill\ensuremath{\blacksquare}}
	\newcommand*{\QEDHERE}{\tag*{$\blacksquare$}}
	\newcommand*{\qel}{\hfill\ensuremath{\boxdot}}
	\newcommand*{\qelhere}{\tag*{$\boxdot$}}
	\renewcommand*{\qedhere}{\tag*{$\square$}}

	\newcommand{\suc}[1]{\left(#1_n\right)_{n\in\N}}
	\newcommand{\en}[2]{\binom{#1}{#2}}
	\newcommand{\upsum}[2]{U(#1,#2)}
	\newcommand{\lowsum}[2]{L(#1,#2)}
	\newcommand{\abs}[1]{\left| #1 \right| }
	\newcommand{\bars}[1]{\left \| #1 \right \| }
	\newcommand{\pars}[1]{\left( #1 \right) }
	\newcommand{\bracs}[1]{\left[ #1 \right] }
	\newcommand{\inprod}[1]{\left\langle #1 \right\rangle }
    \newcommand{\norm}[1]{\left\lVert#1\right\rVert}
    \newcommand{\floor}[1]{\left \lfloor #1 \right\rfloor }
	\newcommand{\ceil}[1]{\left \lceil #1 \right\rceil }
	\newcommand{\angles}[1]{\left \langle #1 \right\rangle }
	\newcommand{\set}[1]{\left \{ #1 \right\} }
	\newcommand{\norma}[2]{\left\| #1 \right\|_{#2} }


	\newcommand{\NN}{\mathbb{N}}
	\newcommand{\QQ}{\mathbb{Q}}
	\newcommand{\RR}{\mathbb{R}}
	\newcommand{\ZZ}{\mathbb{Z}}
	\newcommand{\PP}{\mathbb{P}}
    \newcommand{\EE}{\mathbb{E}}
	\newcommand{\1}{\mathbbm{1}}
	\newcommand{\eps}{\varepsilon}
	\newcommand{\ttF}{\mathtt{F}}
	\newcommand{\bfF}{\mathbf{F}}

	\newcommand{\To}{\longrightarrow}
	\newcommand{\mTo}{\longmapsto}
	\newcommand{\ssi}{\Longleftrightarrow}
	\newcommand{\sii}{\Leftrightarrow}
	\newcommand{\then}{\Rightarrow}

	\newcommand{\pTFC}{{\itshape 1er TFC\/}}
	\newcommand{\sTFC}{{\itshape 2do TFC\/}}


% Datos
    \title{Inteligencia Artificial y Teoría de la Computación \\ Tarea 4}
    \author{Rubén Pérez Palacios Lic. Computación Matemática\\Profesor: Jesús Rodríguez Viorato}
    \date{\today}

% DOCUMENTO
\begin{document}
	\maketitle

	\begin{itemize}
		\item H
		\[\begin{tabular}{|c||c|c|c|c|c|}
			\hline
			pas,ejer & a) & b)  & c) & d) & e)\\
			\hline
			\hline
			1  & $q_{1}11$ & $q_{1}1\#1$ & $q_{1}1\#\#1$ & $q_{1}10\#11$ & $q_{1}10\#10$\\
			\hline
			2  & $xq_{3}1$ & $xq_{3}\#1$ & $xq_{3}\#\#1$ & $xq_{3}0\#11$ & $xq_{3}0\#10$\\
			\hline
			3  & $x1q_{3}$ & $x\#q_{5}1$ & $x\#q_{5}\#1$ & $x0q_{3}\#11$ & $x0q_{3}\#10$\\
			\hline
			4  & $x1q_{r}$ & $xq_{6}\#x$ & $x\#q_{r}\#1$ & $x0\#q_{5}11$ & $x0\#q_{5}10$\\
			\hline
			5  &    	   & $q_{7}x\#x$ & 			     & $x0q_{6}\#x1$ & $x0q_{6}\#x0$\\
			\hline
			6  &    	   & $xq_{1}\#x$ & 			     & $xq_{7}0\#x1$ & $xq_{7}0\#x0$\\
			\hline
			7  &    	   & $x\#q_{8}x$ & 			     & $q_{7}x0\#x1$ & $q_{7}x0\#x0$\\
			\hline
			8  &    	   & $x\#xq_{8}$ & 			     & $xq_{1}0\#x1$ & $xq_{1}0\#x0$\\
			\hline
			9  &    	   & $x\#xq_{a}$ & 			     & $xxq_{2}\#x1$ & $xxq_{2}\#x0$\\
			\hline
			10 &    	   &		     & 			     & $xx\#q_{4}x1$ & $xx\#q_{4}x0$\\
			\hline
			11 &    	   &		     & 			     & $xx\#xq_{4}1$ & $xx\#xq_{4}0$\\
			\hline
			12 &    	   &		     & 			     & $xx\#xq_{r}1$ & $xx\#q_{6}xx$\\
			\hline
			13 &		   &		     & 			     & 			     & $xxq_{6}\#xx$\\
			\hline
			14 &		   &		     & 			     & 			     & $xq_{7}x\#xx$\\
			\hline
			15 &		   &		     & 			     & 			     & $xxq_{8}\#xx$\\
			\hline
			16 &		   &		     & 			     & 			     & $xx\#q_{8}xx$\\
			\hline
			17 &		   &		     & 			     & 			     & $xx\#xq_{8}x$\\
			\hline
			18 &		   &		     & 			     & 			     & $xx\#xxq_{8}$\\
			\hline
			19 &		   &		     & 			     & 			     & $xx\#xxq_{a}$\\
			\hline
		\end{tabular}\]		   
	\end{itemize}
\end{document}

