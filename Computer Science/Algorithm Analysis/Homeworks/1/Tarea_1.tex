% Preámbulo
\documentclass[letterpaper]{article}
\usepackage[utf8]{inputenc}
\usepackage[spanish]{babel}

\usepackage{enumitem}
\usepackage{titling}

% Símbolos
	\usepackage{amsmath}
	\usepackage{amssymb}
	\usepackage{amsthm}
	\usepackage{amsfonts}
	\usepackage{mathtools}
	\usepackage{bbm}
	\usepackage[thinc]{esdiff}
	\allowdisplaybreaks

% Márgenes
	\usepackage
	[
		margin = 1.2in
	]
	{geometry}

% Imágenes
	\usepackage{float}
	\usepackage{graphicx}
	\graphicspath{{imagenes/}}
	\usepackage{subcaption}

% Ambientes
	\usepackage{amsthm}

	\theoremstyle{definition}
	\newtheorem{ejercicio}{Ejercicio}

	\newtheoremstyle{lemathm}{4pt}{0pt}{\itshape}{0pt}{\bfseries}{ --}{ }{\thmname{#1}\thmnumber{ #2}\thmnote{ (#3)}}
	\theoremstyle{lemathm}
	\newtheorem{lema}{Lema}

	\newtheoremstyle{lemathm}{4pt}{0pt}{\itshape}{0pt}{\bfseries}{ --}{ }{\thmname{#1}\thmnumber{ #2}\thmnote{ (#3)}}
	\theoremstyle{lemathm}
	\newtheorem{sol}{Solución}
	
	\newtheoremstyle{lemathm}{4pt}{0pt}{\itshape}{0pt}{\bfseries}{ --}{ }{\thmname{#1}\thmnumber{ #2}\thmnote{ (#3)}}
	\theoremstyle{lemathm}
	\newtheorem{theo}{Teorema}

	\newtheoremstyle{lemademthm}{0pt}{10pt}{\itshape}{ }{\mdseries}{ --}{ }{\thmname{#1}\thmnumber{ #2}\thmnote{ (#3)}}
	\theoremstyle{lemademthm}
	\newtheorem*{lemadem}{Demostración}

% Macros
	\newcommand{\sumi}[2]{\sum_{i=#1}^{#2}}
	\newcommand{\dint}[2]{\displaystyle\int_{#1}^{#2}}
	\newcommand{\inte}[2]{\int_{#1}^{#2}}
	\newcommand{\dlim}{\displaystyle\lim}
	\newcommand{\limxinf}{\lim_{x\to\infty}}
	\newcommand{\limninf}{\lim_{n\to\infty}}
	\newcommand{\dlimninf}{\displaystyle\lim_{n\to\infty}}
	\newcommand{\limh}{\lim_{h\to0}}
	\newcommand{\ddx}{\dfrac{d}{dx}}
	\newcommand{\txty}{\text{ y }}
	\newcommand{\txto}{\text{ o }}
	\newcommand{\Txty}{\quad\text{y}\quad}
	\newcommand{\Txto}{\quad\text{o}\quad}
	\newcommand{\si}{\text{si}\quad}

	\newcommand{\etiqueta}{\stepcounter{equation}\tag{\theequation}}
	\newcommand{\tq}{:}
	\renewcommand{\o}{\circ}
	\newcommand*{\QES}{\hfill\ensuremath{\blacksquare}}
	\newcommand*{\qes}{\hfill\ensuremath{\square}}
	\newcommand*{\QESHERE}{\tag*{$\blacksquare$}}
	\newcommand*{\qeshere}{\tag*{$\square$}}
	\newcommand*{\QED}{\hfill\ensuremath{\blacksquare}}
	\newcommand*{\QEDHERE}{\tag*{$\blacksquare$}}
	\newcommand*{\qel}{\hfill\ensuremath{\boxdot}}
	\newcommand*{\qelhere}{\tag*{$\boxdot$}}
	\renewcommand*{\qedhere}{\tag*{$\square$}}

	\newcommand{\suc}[1]{\left(#1_n\right)_{n\in\N}}
	\newcommand{\en}[2]{\binom{#1}{#2}}
	\newcommand{\upsum}[2]{U(#1,#2)}
	\newcommand{\lowsum}[2]{L(#1,#2)}
	\newcommand{\abs}[1]{\left| #1 \right| }
	\newcommand{\bars}[1]{\left \| #1 \right \| }
	\newcommand{\pars}[1]{\left( #1 \right) }
	\newcommand{\bracs}[1]{\left[ #1 \right] }
	\newcommand{\inprod}[1]{\left\langle #1 \right\rangle }
        \newcommand{\norm}[1]{\left\lVert#1\right\rVert}
        \newcommand{\floor}[1]{\left \lfloor #1 \right\rfloor }
	\newcommand{\ceil}[1]{\left \lceil #1 \right\rceil }
	\newcommand{\angles}[1]{\left \langle #1 \right\rangle }
	\newcommand{\set}[1]{\left \{ #1 \right\} }
	\newcommand{\norma}[2]{\left\| #1 \right\|_{#2} }


	\newcommand{\NN}{\mathbb{N}}
	\newcommand{\QQ}{\mathbb{Q}}
	\newcommand{\RR}{\mathbb{R}}
	\newcommand{\ZZ}{\mathbb{Z}}
	\newcommand{\PP}{\mathbb{P}}
        \newcommand{\EE}{\mathbb{E}}
	\newcommand{\1}{\mathbbm{1}}
	\newcommand{\eps}{\varepsilon}
	\newcommand{\ttF}{\mathtt{F}}
	\newcommand{\bfF}{\mathbf{F}}

	\newcommand{\To}{\longrightarrow}
	\newcommand{\mTo}{\longmapsto}
	\newcommand{\ssi}{\Longleftrightarrow}
	\newcommand{\sii}{\Leftrightarrow}
	\newcommand{\then}{\Rightarrow}

	\newcommand{\pTFC}{{\itshape 1er TFC\/}}
	\newcommand{\sTFC}{{\itshape 2do TFC\/}}


% Datos
    \title{Análisis de Algoritmos e introducción a Matemáticas Discretas \\ Tarea 1}
    \author{Rubén Pérez Palacios Lic. Computación Matemática\\Profesor: Dr. Carlos Segura González}
    \date{\today}

% DOCUMENTO
\begin{document}
	\maketitle

    \begin{enumerate}
        \item Se uso una cola de prioridad para resolver este problema, en la cual guardamos el tamaño de las flechas. Para que esta nos pueda devolver el tamaño más pequeño hicimos uso del comparador $greater$. Puesto que al insertar y eliminar tienen complejidad logaritmica, concluimos que la complejidad final de nuestro algoritmo es $O(NlogN)$.
        \item Para resolver el problema hicimos uso de un multiset para emular un maxheap de una pareja precios y nombre. Cómo el problema también nos pide eliminar y actualizar por nombre entonces necesitamos una estructura auxiliar para recordar el precio de estos alimentos y así poder eliminarlos del set, optamos por un map. Finalmente obtenemos que todas nuestras operaciones tienen complejidad logaritmica por lo tanto nuestra complejidad final es $O(NlogN)$.
        \item La solución de este problema nos conviene fijarnos en los zombies de izquierda a derecha iterando de uno en uno. Al procesar el primer zombie este deberá ser matado con algún conjuro que empiece en la posición $1$ de lo contrario seguira vivio; luego de todos los conjuros que cumplen esta condición nos convendra usar primero el conjuro que termine en la posición más a la derecha, ya que si una solución optima contiene a un conjuro que termine en una posición menos a la derecha al intercambiarlo por este en el peor de los casos nos deja la misma solución o la mejora. Por inducción veamos que si todos los zombies anteriores al actual procesado ya fueron destruidos, entonces usando los conjuros disponibles donde su rango la posición del actual zombie está incluido siempre nos convendra usarlos en orden del cuál termine más a la derecha. Si en algún punto la cantidad de conjuros es menor a la vida el zombie entonces no habrá solución ya que todos los conjuros que lo contienen en su rango fueron usados anteriormente o están disponibles no alcanzaron los suficientes puntos. Para la solución de este problema basta con ordenar los conjuros por el inicio de su rango (para ello se uso una cola de prioridad), tener una cola de prioridad para mantener los conjuros disponibles y que nos los ordene por el que termine más a la derecha (también se uso una cola de prioridad), depués iterar de izquierda a derecha y para cada uno encontrar cuales conjuros ahora pueden usarse, después usar los conjuros en el orden descrito y finalmente comprobar si fue posible matar al zombie. Debido a que las operaciones de una colda de prioridad son logaritmicas concluimos que la complejidad final del problema es $O(NlogM)$.
    \end{enumerate}
\end{document}