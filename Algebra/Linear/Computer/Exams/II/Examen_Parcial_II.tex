% Preámbulo
\documentclass[letterpaper]{article}
\usepackage[utf8]{inputenc}
\usepackage[spanish]{babel}

\usepackage{enumitem}
\usepackage{titling}

% Símbolos
	\usepackage{amsmath}
	\usepackage{amssymb}
	\usepackage{amsthm}
	\usepackage{amsfonts}
	\usepackage{mathtools}
	\usepackage{bbm}
	\usepackage[thinc]{esdiff}
	\allowdisplaybreaks

% Márgenes
	\usepackage
	[
		margin = 1.1in
	]
	{geometry}

% Imágenes
	\usepackage{float}
	\usepackage{graphicx}
	\graphicspath{{imagenes/}}
	\usepackage{subcaption}

% Ambientes
	\usepackage{amsthm}

	\theoremstyle{definition}
	\newtheorem{ejercicio}{Ejercicio}

	\newtheoremstyle{lemathm}{4pt}{0pt}{\itshape}{0pt}{\bfseries}{ --}{ }{\thmname{#1}\thmnumber{ #2}\thmnote{ (#3)}}
	\theoremstyle{lemathm}
	\newtheorem{lema}{Lema}

	\newtheoremstyle{lemademthm}{0pt}{10pt}{\itshape}{ }{\mdseries}{ --}{ }{\thmname{#1}\thmnumber{ #2}\thmnote{ (#3)}}
	\theoremstyle{lemademthm}
	\newtheorem*{lemadem}{Demostración}

% Macros
	\newcommand{\sumi}[2]{\sum_{i=#1}^{#2}}
	\newcommand{\dint}[2]{\displaystyle\int_{#1}^{#2}}
	\newcommand{\inte}[2]{\int_{#1}^{#2}}
	\newcommand{\dlim}{\displaystyle\lim}
	\newcommand{\limxinf}{\lim_{x\to\infty}}
	\newcommand{\limninf}{\lim_{n\to\infty}}
	\newcommand{\dlimninf}{\displaystyle\lim_{n\to\infty}}
	\newcommand{\limh}{\lim_{h\to0}}
	\newcommand{\ddx}{\dfrac{d}{dx}}
	\newcommand{\txty}{\text{ y }}
	\newcommand{\txto}{\text{ o }}
	\newcommand{\Txty}{\quad\text{y}\quad}
	\newcommand{\Txto}{\quad\text{o}\quad}
	\newcommand{\si}{\text{si}\quad}

	\newcommand{\etiqueta}{\stepcounter{equation}\tag{\theequation}}
	\newcommand{\tq}{:}
	\renewcommand{\o}{\circ}
	\newcommand*{\QES}{\hfill\ensuremath{\blacksquare}}
	\newcommand*{\qes}{\hfill\ensuremath{\square}}
	\newcommand*{\QESHERE}{\tag*{$\blacksquare$}}
	\newcommand*{\qeshere}{\tag*{$\square$}}
	\newcommand*{\QED}{\hfill\ensuremath{\blacksquare}}
	\newcommand*{\QEDHERE}{\tag*{$\blacksquare$}}
	\newcommand*{\qel}{\hfill\ensuremath{\boxdot}}
	\newcommand*{\qelhere}{\tag*{$\boxdot$}}
	\renewcommand*{\qedhere}{\tag*{$\square$}}

	\newcommand{\suc}[1]{\left(#1_n\right)_{n\in\N}}
	\newcommand{\en}[2]{\binom{#1}{#2}}
	\newcommand{\upsum}[2]{U(#1,#2)}
	\newcommand{\lowsum}[2]{L(#1,#2)}
	\newcommand{\abs}[1]{\left| #1 \right| }
	\newcommand{\bars}[1]{\left \| #1 \right \| }
	\newcommand{\pars}[1]{\left( #1 \right) }
	\newcommand{\bracs}[1]{\left[ #1 \right] }
	\newcommand{\floor}[1]{\left \lfloor #1 \right\rfloor }
	\newcommand{\ceil}[1]{\left \lceil #1 \right\rceil }
	\newcommand{\angles}[1]{\left \langle #1 \right\rangle }
	\newcommand{\set}[1]{\left \{ #1 \right\} }
	\newcommand{\norma}[2]{\left\| #1 \right\|_{#2} }

	\newcommand{\NN}{\mathbb{N}}
	\newcommand{\QQ}{\mathbb{Q}}
	\newcommand{\RR}{\mathbb{R}}
	\newcommand{\ZZ}{\mathbb{Z}}
	\newcommand{\PP}{\mathbb{P}}
	\newcommand{\EE}{\mathbb{E}}
	\newcommand{\1}{\mathbbm{1}}
	\newcommand{\eps}{\varepsilon}
	\newcommand{\ttF}{\mathtt{F}}
	\newcommand{\bfF}{\mathbf{F}}

	\newcommand{\To}{\longrightarrow}
	\newcommand{\mTo}{\longmapsto}
	\newcommand{\ssi}{\Longleftrightarrow}
	\newcommand{\sii}{\Leftrightarrow}
	\newcommand{\then}{\Rightarrow}

	\newcommand{\pTFC}{{\itshape 1er TFC\/}}
    \newcommand{\sTFC}{{\itshape 2do TFC\/}}
    
% Datos
    \title{Álgebra para Ciencias de la Computación\\Examen Parcial II}
    \author{Rubén Pérez Palacios\\Profesor: Dr. Rafael Herrera Guzmán}
    \date{\today}

% DOCUMENTO
\begin{document}
	\maketitle
    
    \section*{Problemas}

    \begin{enumerate}

		\item Calcula los determinantes, las adjuntas clásicas y, en caso de que exitan, la inversas de las siguientes matrices:
		
		\[A = \begin{pmatrix}
			-1 & 2 & 5\\[1ex]
			8 & 0 & -3\\[1ex]
			4 & 6 & 1
		\end{pmatrix}.\]

		Las matriz adjunta clásica es

		\[\text{adj}(A) = C^T = \begin{pmatrix}
			-18 & -20 & 48\\[1ex]
			28 & -21 & 14\\[1ex]
			-6 & 37 & -16
		\end{pmatrix}^T = \begin{pmatrix}
			-18 & 28 & -6\\[1ex]
			-20 & -21 & 37\\[1ex]
			48 & 14 & -16
		\end{pmatrix}.\]

		\item Resolver el sistema
		
		\begin{align*}
			x-y+4z &= -2,\\
			-8x+3y+z &= 0,\\
			2x-y+z &= 6,\\
		\end{align*}

		usando la regla de Cramer.

		La matriz extendida asociada al sistema de ecuaciones es

		\[E = \pars{\begin{array}{ccc|c}
			1 & -1 & 4 & 2\\[1ex]
			-8 & 3 & 1 & 0\\[1ex]
			2 & -1 & 1 & 6\\
		\end{array}},\]

		la matriz asociada al sitema de ecuaciones es

		\[A = \pars{\begin{array}{ccc}
			1 & -1 & 4\\[1ex]
			-8 & 3 & 1\\[1ex]
			2 & -1 & 1\\
		\end{array}},\]

		cuyo determinante es

		\[\abs{\begin{matrix}
			1 & -1 & 4 \\[1ex]
			-8 & 3 & 1 \\[1ex]
			2 & -1 & 1
		\end{matrix}} = 2.\]

		Las matrices $A_i$ que son las matrices formadas por remplazar la $i-esima$ columna de $A$ por la última columna de $E$, cuyos determinantes son

		\[det(A_1) = \abs{\begin{matrix}
			-2 & -1 & 4 \\[1ex]
			0 & 3 & 1 \\[1ex]
			6 & -1 & 1
		\end{matrix}} = -86,\]

		\[det(A_2) = \abs{\begin{matrix}
			1 & -2 & 4 \\[1ex]
			-8 & 0 & 1 \\[1ex]
			2 & 6 & 1
		\end{matrix}} = -218,\]

		\[det(A_3) = \abs{\begin{matrix}
			1 & -1 & -2 \\[1ex]
			-8 & 3 & 0 \\[1ex]
			2 & -1 & 6
		\end{matrix}} = -34.\]

		Por la regla de Cramer concluimos que las soluciones al sistema de ecuaciones son

		\[x = \frac{-86}{2} = -43, \quad y = \frac{-218}{2} = -109, \quad z = \frac{-34}{2} = -17.\]

		\item Encuentra la forma canónica de Jordan real de
		
		\[\pars{\begin{array}{cccccccc} \frac{3}{2} & 1 & -1 & \frac{3}{2} & -1 & 0 & \frac{1}{2} & -\frac{1}{2} \\[1ex] -\frac{1}{2} & 3 & -1 & \frac{3}{2} & -1 & 0 & \frac{1}{2} & -\frac{1}{2} \\[1ex] -\frac{1}{2} & 1 & 2 & -\frac{1}{2} & 0 & 0 & \frac{1}{2} & -\frac{1}{2} \\[1ex] -\frac{1}{2} & 0 & 1 & \frac{3}{2} & -1 & 1 & \frac{1}{2} & -\frac{1}{2} \\[1ex] -\frac{1}{2} & 0 & 1 & -\frac{1}{2} & 2 & 0 & \frac{1}{2} & -\frac{1}{2} \\[1ex] -\frac{1}{2} & 0 & 1 & -\frac{3}{2} & 1 & 2 & \frac{1}{2} & -\frac{1}{2} \\[1ex] -\frac{1}{2} & 0 & 1 & -\frac{3}{2} & 1 & 0 & \frac{5}{2} & -\frac{1}{2} \\[1ex] -\frac{1}{2} & 0 & 1 & -\frac{3}{2} & 1 & 0 & -\frac{1}{2} & \frac{5}{2} \end{array}}.\]

		El polinomio característico de la matriz anterior es

		\[(x-3)(x-2)^2\pars{x-\pars{2+i}}^2\pars{x-\pars{2-i}}^2,\]

		por lo tanto los valores propios de esta matriz son

		\[2,3,2+i,2-i.\]

		Ahora encontremoos los vectores propios de estos valores propios. Sea $v \in \RR^8$ de la forma

		\[v = \pars{\begin{array}{c}
			v_1\\[1ex]
			v_2\\[1ex]
			v_3\\[1ex]
			v_4\\[1ex]
			v_5\\[1ex]
			v_6\\[1ex]
			v_7\\[1ex]
			v_8\\
		\end{array}}.\]

		\begin{itemize}
			\item $\lambda = 2$
			
			Resolviendo el siguiente sistema de ecuaciones
			
			\[\pars{\begin{array}{cccccccc} \frac{3}{2} & 1 & -1 & \frac{3}{2} & -1 & 0 & \frac{1}{2} & -\frac{1}{2} \\[1ex] -\frac{1}{2} & 3 & -1 & \frac{3}{2} & -1 & 0 & \frac{1}{2} & -\frac{1}{2} \\[1ex] -\frac{1}{2} & 1 & 2 & -\frac{1}{2} & 0 & 0 & \frac{1}{2} & -\frac{1}{2} \\[1ex] -\frac{1}{2} & 0 & 1 & \frac{3}{2} & -1 & 1 & \frac{1}{2} & -\frac{1}{2} \\[1ex] -\frac{1}{2} & 0 & 1 & -\frac{1}{2} & 2 & 0 & \frac{1}{2} & -\frac{1}{2} \\[1ex] -\frac{1}{2} & 0 & 1 & -\frac{3}{2} & 1 & 2 & \frac{1}{2} & -\frac{1}{2} \\[1ex] -\frac{1}{2} & 0 & 1 & -\frac{3}{2} & 1 & 0 & \frac{5}{2} & -\frac{1}{2} \\[1ex] -\frac{1}{2} & 0 & 1 & -\frac{3}{2} & 1 & 0 & -\frac{1}{2} & \frac{5}{2} \end{array}} v = 2 v,\]

			obtenemos que $v$ es de la forma

			\[v = \pars{\begin{array}{c}
				a\\[1ex]
				a\\[1ex]
				a\\[1ex]
				a\\[1ex]
				a\\[1ex]
				a\\[1ex]
				b\\[1ex]
				b\\
			\end{array}}, \quad a,b\in\RR,\]

			por lo que obtenemos los siguientes 2 vectores propios

			\[p_1 = \pars{\begin{array}{c}
				1\\[1ex]
				1\\[1ex]
				1\\[1ex]
				1\\[1ex]
				1\\[1ex]
				1\\[1ex]
				0\\[1ex]
				0\\
			\end{array}}, \quad p_2 = \pars{\begin{array}{c}
				0\\[1ex]
				0\\[1ex]
				0\\[1ex]
				0\\[1ex]
				0\\[1ex]
				0\\[1ex]
				1\\[1ex]
				1\\
			\end{array}}.\]

			Nos falta un vector para llegar a la multiplicidad del valor propio $2$ por lo que calcularemos el generalizado de $p_1$ resolviendo el siguiente sistema de ecuaciones

			\[\pars{\begin{array}{cccccccc}
				\frac{-1}{2} & 1 & -1 & \frac{3}{2} & -1 & 0 & \frac{1}{2} & \frac{-1}{2} \\[1ex]
				\frac{-1}{2} & 1 & -1 & \frac{3}{2} & -1 & 0 & \frac{1}{2} & \frac{-1}{2} \\[1ex]
				\frac{-1}{2} & 1 & 0 & \frac{-1}{2} & 0 & 0 & \frac{1}{2} & \frac{-1}{2} \\[1ex]
				\frac{-1}{2} & 0 & 1 & \frac{-1}{2} & -1 & 1 & \frac{1}{2} & \frac{-1}{2} \\[1ex]
				\frac{-1}{2} & 0 & 1 & \frac{-1}{2} & 0 & 0 & \frac{1}{2} & \frac{-1}{2} \\[1ex]
				\frac{-1}{2} & 0 & 1 & \frac{-3}{2} & 1 & 0 & \frac{1}{2} & \frac{-1}{2} \\[1ex]
				\frac{-1}{2} & 0 & 1 & \frac{-3}{2} & 1 & 0 & \frac{1}{2} & \frac{-1}{2} \\
				\frac{-1}{2} & 0 & 1 & \frac{-3}{2} & 1 & 0 & \frac{-1}{2} & \frac{1}{2}
			 \end{array}} v = 2 p_1,\]

			obtenemos que $v$ es de la forma

			\[v = \pars{\begin{array}{c}
				-2+a\\[1ex]
				a\\[1ex]
				a\\[1ex]
				a\\[1ex]
				a\\[1ex]
				a\\[1ex]
				b\\[1ex]
				b\\
			\end{array}}, \quad a,b\in\RR,\]

			por lo que obtenemos elos siguiente vector generalizado

			\[g_1 = \pars{\begin{array}{c}
				-2\\[1ex]
				0\\[1ex]
				0\\[1ex]
				0\\[1ex]
				0\\[1ex]
				0\\[1ex]
				1\\[1ex]
				1\\
			\end{array}}.\]

			\item $\lambda = 3$
			
			Resolviendo el siguiente sistema de ecuaciones
			
			\[\pars{\begin{array}{cccccccc} \frac{3}{2} & 1 & -1 & \frac{3}{2} & -1 & 0 & \frac{1}{2} & -\frac{1}{2} \\[1ex] -\frac{1}{2} & 3 & -1 & \frac{3}{2} & -1 & 0 & \frac{1}{2} & -\frac{1}{2} \\[1ex] -\frac{1}{2} & 1 & 2 & -\frac{1}{2} & 0 & 0 & \frac{1}{2} & -\frac{1}{2} \\[1ex] -\frac{1}{2} & 0 & 1 & \frac{3}{2} & -1 & 1 & \frac{1}{2} & -\frac{1}{2} \\[1ex] -\frac{1}{2} & 0 & 1 & -\frac{1}{2} & 2 & 0 & \frac{1}{2} & -\frac{1}{2} \\[1ex] -\frac{1}{2} & 0 & 1 & -\frac{3}{2} & 1 & 2 & \frac{1}{2} & -\frac{1}{2} \\[1ex] -\frac{1}{2} & 0 & 1 & -\frac{3}{2} & 1 & 0 & \frac{5}{2} & -\frac{1}{2} \\[1ex] -\frac{1}{2} & 0 & 1 & -\frac{3}{2} & 1 & 0 & -\frac{1}{2} & \frac{5}{2} \end{array}} v = 3 v,\]

			obtenemos que $v$ es de la forma

			\[v = \pars{\begin{array}{c}
				a\\[1ex]
				a\\[1ex]
				a\\[1ex]
				a\\[1ex]
				a\\[1ex]
				a\\[1ex]
				a\\[1ex]
				-a\\
			\end{array}}, \quad a\in\RR,\]

			por lo que obtenemos el siguiente vector propio

			\[p_3 = \pars{\begin{array}{c}
				1\\[1ex]
				1\\[1ex]
				1\\[1ex]
				1\\[1ex]
				1\\[1ex]
				1\\[1ex]
				1\\[1ex]
				-1\\
			\end{array}}.\]

			\item $\lambda = 2+i$
			
			Resolviendo el siguiente sistema de ecuaciones
			
			\[\pars{\begin{array}{cccccccc} \frac{3}{2} & 1 & -1 & \frac{3}{2} & -1 & 0 & \frac{1}{2} & -\frac{1}{2} \\[1ex] -\frac{1}{2} & 3 & -1 & \frac{3}{2} & -1 & 0 & \frac{1}{2} & -\frac{1}{2} \\[1ex] -\frac{1}{2} & 1 & 2 & -\frac{1}{2} & 0 & 0 & \frac{1}{2} & -\frac{1}{2} \\[1ex] -\frac{1}{2} & 0 & 1 & \frac{3}{2} & -1 & 1 & \frac{1}{2} & -\frac{1}{2} \\[1ex] -\frac{1}{2} & 0 & 1 & -\frac{1}{2} & 2 & 0 & \frac{1}{2} & -\frac{1}{2} \\[1ex] -\frac{1}{2} & 0 & 1 & -\frac{3}{2} & 1 & 2 & \frac{1}{2} & -\frac{1}{2} \\[1ex] -\frac{1}{2} & 0 & 1 & -\frac{3}{2} & 1 & 0 & \frac{5}{2} & -\frac{1}{2} \\[1ex] -\frac{1}{2} & 0 & 1 & -\frac{3}{2} & 1 & 0 & -\frac{1}{2} & \frac{5}{2} \end{array}} v = (2+i) v,\]

			obtenemos que $v$ es de la forma

			\[v = \pars{\begin{array}{c}
				-a\\[1ex]
				-a\\[1ex]
				-ia\\[1ex]
				a\\[1ex]
				a\\[1ex]
				a\\[1ex]
				a\\[1ex]
				a\\
			\end{array}}, \quad a\in\RR,\]

			por lo que obtenemos el siguiente vector propio

			\[p_4 = \pars{\begin{array}{c}
				-1\\[1ex]
				-1\\[1ex]
				-i\\[1ex]
				1\\[1ex]
				1\\[1ex]
				1\\[1ex]
				1\\[1ex]
				1\\
			\end{array}} = \pars{\begin{array}{c}
				-1\\[1ex]
				-1\\[1ex]
				0\\[1ex]
				1\\[1ex]
				1\\[1ex]
				1\\[1ex]
				1\\[1ex]
				1\\
			\end{array}} + i\pars{\begin{array}{c}
				0\\[1ex]
				0\\[1ex]
				-1\\[1ex]
				0\\[1ex]
				0\\[1ex]
				0\\[1ex]
				0\\[1ex]
				0\\
			\end{array}}.\]

			\item $\lambda = 2-i$
			
			Resolviendo el siguiente sistema de ecuaciones
			
			\[\pars{\begin{array}{cccccccc} \frac{3}{2} & 1 & -1 & \frac{3}{2} & -1 & 0 & \frac{1}{2} & -\frac{1}{2} \\[1ex] -\frac{1}{2} & 3 & -1 & \frac{3}{2} & -1 & 0 & \frac{1}{2} & -\frac{1}{2} \\[1ex] -\frac{1}{2} & 1 & 2 & -\frac{1}{2} & 0 & 0 & \frac{1}{2} & -\frac{1}{2} \\[1ex] -\frac{1}{2} & 0 & 1 & \frac{3}{2} & -1 & 1 & \frac{1}{2} & -\frac{1}{2} \\[1ex] -\frac{1}{2} & 0 & 1 & -\frac{1}{2} & 2 & 0 & \frac{1}{2} & -\frac{1}{2} \\[1ex] -\frac{1}{2} & 0 & 1 & -\frac{3}{2} & 1 & 2 & \frac{1}{2} & -\frac{1}{2} \\[1ex] -\frac{1}{2} & 0 & 1 & -\frac{3}{2} & 1 & 0 & \frac{5}{2} & -\frac{1}{2} \\[1ex] -\frac{1}{2} & 0 & 1 & -\frac{3}{2} & 1 & 0 & -\frac{1}{2} & \frac{5}{2} \end{array}} v = (2+i) v,\]

			obtenemos que $v$ es de la forma

			\[v = \pars{\begin{array}{c}
				-a\\[1ex]
				-a\\[1ex]
				ia\\[1ex]
				a\\[1ex]
				a\\[1ex]
				a\\[1ex]
				a\\[1ex]
				a\\
			\end{array}}, \quad a\in\RR,\]

			por lo que obtenemos el siguiente vector propio

			\[p_5 = \pars{\begin{array}{c}
				-1\\[1ex]
				-1\\[1ex]
				i\\[1ex]
				1\\[1ex]
				1\\[1ex]
				1\\[1ex]
				1\\[1ex]
				1\\
			\end{array}} = \pars{\begin{array}{c}
				-1\\[1ex]
				-1\\[1ex]
				0\\[1ex]
				1\\[1ex]
				1\\[1ex]
				1\\[1ex]
				1\\[1ex]
				1\\
			\end{array}} + i\pars{\begin{array}{c}
				0\\[1ex]
				0\\[1ex]
				1\\[1ex]
				0\\[1ex]
				0\\[1ex]
				0\\[1ex]
				0\\[1ex]
				0\\
			\end{array}}.\]

		\end{itemize}

		Ahora de lo anterior podemos ver que la descomposición de los vectores complejos $p_4$ y $p_5$ nos dan 4 vectores de los cuales solo 2 son independientes por lo que calcularemos un vector generalizado para uno de estos vectores propios. Tomando $p_4$ y resolviendo el siguiente sistema de ecuaciones

		\[\pars{\begin{array}{cccccccc}
			\frac{-1-2*i}{2} & 1 & -1 & \frac{3}{2} & -1 & 0 & \frac{1}{2} & \frac{-1}{2} \\[1ex]
			\frac{-1}{2} & 1-i & -1 & \frac{3}{2} & -1 & 0 & \frac{1}{2} & \frac{-1}{2} \\[1ex]
			\frac{-1}{2} & 1 & -i & \frac{-1}{2} & 0 & 0 & \frac{1}{2} & \frac{-1}{2} \\[1ex]
			\frac{-1}{2} & 0 & 1 & \frac{-1-2*i}{2} & -1 & 1 & \frac{1}{2} & \frac{-1}{2} \\[1ex]
			\frac{-1}{2} & 0 & 1 & \frac{-1}{2} & -i & 0 & \frac{1}{2} & \frac{-1}{2} \\[1ex]
			\frac{-1}{2} & 0 & 1 & \frac{-3}{2} & 1 & -i & \frac{1}{2} & \frac{-1}{2} \\[1ex]
			\frac{-1}{2} & 0 & 1 & \frac{-3}{2} & 1 & 0 & \frac{1-2*i}{2} & \frac{-1}{2} \\[1ex]
			\frac{-1}{2} & 0 & 1 & \frac{-3}{2} & 1 & 0 & \frac{-1}{2} & \frac{1-2*i}{2}\\
		\end{array}}v = p_4,\]

		obtenemos que $v$ es de la forma

		\[v = \pars{\begin{array}{c}
			-a\\[1ex]
			-a\\[1ex]
			-1 + i(1-a)\\[1ex]
			-2 + a\\[1ex]
			-1 + a - i\\[1ex]
			a\\[1ex]
			a\\[1ex]
			a\\
		\end{array}}, \quad a\in\RR,\]

		por lo que obtenemos el siguiente vector generalizado

		\[g_2 = \pars{\begin{array}{c}
			-1\\[1ex]
			-1\\[1ex]
			-1\\[1ex]
			-1\\[1ex]
			-i\\[1ex]
			1\\[1ex]
			1\\[1ex]
			1\\
		\end{array}}.\]

		Ahora tomandonos como base $\beta = \set{p_1,p_2,g_1,p_3,p_4,p_{5_r},p_{5_c},g_2}$ obtenemos la siguiente matriz de cambio de coordenadas

		\[Q = \pars{\begin{array}{cccccccc}
				1 & -2 & 0 & -1 & -1 &  0 & -1 &  0\\[1ex]
				1 &  0 & 0 & -1 & -1 &  0 & -1 &  0\\[1ex]
				1 &  0 & 0 & -1 &  0 & -1 & -1 &  0\\[1ex]
				1 &  0 & 0 & -1 &  1 &  0 & -1 &  0\\[1ex]
				1 &  0 & 0 & -1 &  1 &  0 &  0 & -1\\[1ex]
				1 &  0 & 0 & -1 &  1 &  0 &  1 &  0\\[1ex]
				1 &  1 & 1 & -1 &  1 &  0 &  1 &  0\\[1ex]
				1 &  1 & 1 &  1 &  1 &  0 &  1 &  0\\
		\end{array}},\]

		cuya inversa es

		\[Q^{-1} = \pars{\begin{array}{cccccccc}
			0 & \frac{1}{2} & 0 & 0 & 0 & \frac{1}{2} & \frac{-1}{2} & \frac{1}{2} \\[1ex]
			\frac{-1}{2} & \frac{1}{2} & 0 & 0 & 0 & 0 & 0 & 0 \\[1ex]
			\frac{1}{2} & \frac{-1}{2} & 0 & 0 & 0 & -1 & 1 & 0 \\[1ex]
			0 & 0 & 0 & 0 & 0 & 0 & \frac{-1}{2} & \frac{1}{2} \\[1ex]
			0 & \frac{-1}{2} & 0 & \frac{1}{2} & 0 & 0 & 0 & 0 \\[1ex]
			0 & \frac{1}{2} & -1 & \frac{1}{2} & 0 & 0 & 0 & 0 \\[1ex]
			0 & 0 & 0 & \frac{-1}{2} & 0 & \frac{1}{2} & 0 & 0 \\[1ex]
			0 & 0 & 0 & \frac{1}{2} & -1 & \frac{1}{2} & 0 & 0
		\end{array}}.\]

		Por lo tanto concluimos que la forma canónica real de la matriz A es

		\[\pars{\begin{array}{cccccccc}
			2 & 1 & 0 & 0 & 0 & 0 & 0 & 0 \\[1ex]
			0 & 2 & 0 & 0 & 0 & 0 & 0 & 0 \\[1ex]
			0 & 0 & 2 & 0 & 0 & 0 & 0 & 0 \\[1ex]
			0 & 0 & 0 & 3 & 0 & 0 & 0 & 0 \\[1ex]
			0 & 0 & 0 & 0 & 2 & -1 & 1 & 0 \\[1ex]
			0 & 0 & 0 & 0 & 1 & 2 & 0 & 1 \\[1ex]
			0 & 0 & 0 & 0 & 0 & 0 & 2 & -1 \\[1ex]
			0 & 0 & 0 & 0 & 0 & 0 & 1 & 2
		\end{array}}.\]

	\end{enumerate}

	
\end{document}