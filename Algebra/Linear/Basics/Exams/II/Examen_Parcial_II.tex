% Preámbulo
\documentclass[letterpaper]{article}
\usepackage[utf8]{inputenc}
\usepackage[spanish]{babel}

\usepackage{enumitem}
\usepackage{titling}

% Símbolos
	\usepackage{amsmath}
	\usepackage{amssymb}

% Márgenes
	\usepackage
	[
		margin = 1.4in
	]
	{geometry}

% Imágenes
	\usepackage{float}
	\usepackage{graphicx}
	\graphicspath{{imagenes/}}
	\usepackage{subcaption}

% Ambientes
	\usepackage{amsthm}

	\renewcommand{\qedsymbol}{$\blacksquare$}

	\theoremstyle{definition}
	\newtheorem{ejercicio}{Ejercicio}

	\newtheoremstyle{lemathm}{4pt}{0pt}{\itshape}{0pt}{\bfseries}{ --}{ }{\thmname{#1}\thmnumber{ #2}\thmnote{ (#3)}}
	\theoremstyle{lemathm}
	\newtheorem{lema}{Lema}

	\newtheoremstyle{lemademthm}{0pt}{10pt}{\itshape}{ }{\mdseries}{ --}{ }{\thmname{#1}\thmnumber{ #2}\thmnote{ (#3)}}
	\theoremstyle{lemademthm}
	\newtheorem*{lemadem}{Demostración}

% Ajustes
	\allowdisplaybreaks	% Los align pueden cambiar de página

% Macros
	\newcommand{\sumi}[2]{\sum_{i=#1}^{#2}}
	\newcommand{\dint}[2]{\displaystyle\int_{#1}^{#2}}
	\newcommand{\inte}[2]{\int_{#1}^{#2}}
	\newcommand{\dlim}{\displaystyle\lim}
	\newcommand{\limxinf}{\lim_{x\to\infty}}
	\newcommand{\limninf}{\lim_{n\to\infty}}
	\newcommand{\dlimninf}{\displaystyle\lim_{n\to\infty}}
	\newcommand{\limh}{\lim_{h\to0}}
	\newcommand{\ddx}{\dfrac{d}{dx}}
	\newcommand{\txty}{\text{ y }}
	\newcommand{\txto}{\text{ o }}
	\newcommand{\Txty}{\quad\text{y}\quad}
	\newcommand{\Txto}{\quad\text{o}\quad}
	\newcommand{\si}{\text{si}\quad}

	\newcommand{\etiqueta}{\stepcounter{equation}\tag{\theequation}}
	\newcommand{\tq}{:}
	\renewcommand{\o}{\circ}
	% \newcommand*{\QES}{\hfill\ensuremath{\boxplus}}
	% \newcommand*{\qes}{\hfill\ensuremath{\boxminus}}
	% \newcommand*{\qeshere}{\tag*{$\boxminus$}}
	% \newcommand*{\QESHERE}{\tag*{$\boxplus$}}
	\newcommand*{\QES}{\hfill\ensuremath{\blacksquare}}
	\newcommand*{\qes}{\hfill\ensuremath{\square}}
	\newcommand*{\QESHERE}{\tag*{$\blacksquare$}}
	\newcommand*{\qeshere}{\tag*{$\square$}}
	\newcommand*{\QED}{\hfill\ensuremath{\blacksquare}}
	\newcommand*{\QEDHERE}{\tag*{$\blacksquare$}}
	\newcommand*{\qel}{\hfill\ensuremath{\boxdot}}
	\newcommand*{\qelhere}{\tag*{$\boxdot$}}
	\renewcommand*{\qedhere}{\tag*{$\square$}}

	\newcommand{\abs}[1]{\left\vert#1\right\vert}
	\newcommand{\suc}[1]{\left(#1_n\right)_{n\in\N}}
	\newcommand{\en}[2]{\binom{#1}{#2}}
	\newcommand{\upsum}[2]{U(#1,#2)}
	\newcommand{\lowsum}[2]{L(#1,#2)}

	\newcommand{\N}{\mathbb{N}}
	\newcommand{\Q}{\mathbb{Q}}
	\newcommand{\R}{\mathbb{R}}
	\newcommand{\Z}{\mathbb{Z}}
	\newcommand{\eps}{\varepsilon}
	\newcommand{\ttF}{\mathtt{F}}
	\newcommand{\bfF}{\mathbf{F}}

	\newcommand{\To}{\longrightarrow}
	\newcommand{\mTo}{\longmapsto}
	\newcommand{\ssi}{\Longleftrightarrow}
	\newcommand{\sii}{\Leftrightarrow}
	\newcommand{\then}{\Rightarrow}

	\newcommand{\pTFC}{{\itshape 1er TFC\/}}
	\newcommand{\sTFC}{{\itshape 2do TFC\/}}
    
% Datos
    \title{Álgebra Lineal I\\Examen Parcial II}
    \author{Rubén Pérez Palacios\\Profesor: Rafael Herrera Guzmán}
    \date{28 Abril 2020}

% DOCUMENTO
\begin{document}
	\maketitle
    
    \section*{Problemas}

    \begin{enumerate}
        
        \item Supóngase que $v_1, \cdots , v_n$ son vectores linealmente independientes en $V$ y $w \in V$. Demostrar que
		
		\[\dim(span(\{v_1+w,\cdots,v_n+w\})) \geq n-1.\]

		\begin{proof}
			Los vectores $v_2 - v_1, \cdots, v_n - v_1$ también son linealmente independientes, ya que de no ser así entonces hay una combinación lineal de los vectores $v_2 - v_1, \cdots, v_n - v_1$ tal que es igual a cero y no todos sus coeficientes son cero, pero esta combinación lineal también es de los vectores $v_1,\cdots,v_n$ por lo que serían dependientes lo cual es una contradicción; así que 
			
			\[\dim(span(\{v_2-v_1,\cdots,v_n-v_1\})) = n-1.\]
			
			También como
			
			\[v_i - v_1 = (v_i + w) - (v_1 + w) \text{ para } i = \{2,\cdots,n\},\]
			
			entonces
			
			\[v_i - v_1 \in span(\{v_1+w,\cdots,v_2+w\}), i = \{2,\cdots,n\},\]

			por lo tanto

			\[span(\{v_2-v_1,\cdots,v_n-v_1\}) \subset span(\{v_1+w,\cdots,v_2+w\})\]

			Por el teorema $1.11$ concluimos que

			\[\dim(span(\{v_1+w,\cdots,v_n+w\})) \geq \dim(span(\{v_2-v_1,\cdots,v_n-v_1\})) = n - 1.\]
			
		\end{proof}

		\newpage

		\item Sea
		
		\[V = \left\{\begin{pmatrix}
			0 & a\\
			b & c
		\end{pmatrix} \in M_{2\times2}(\R) | c = a + b \right\}.\]

		Definase $T: V \rightarrow V$ dada por

		\[T\left(\begin{pmatrix}
			0 & a\\
			b & c
		\end{pmatrix}\right) = \begin{pmatrix}
			0 & 3a + 3c\\
			-2a-b & a-b+3c
		\end{pmatrix}.\]

		De una base de $V$, la matriz de $T$ en dicha base y bases para $N(T)$,$R(T)$.

		Sea

		\[\beta = \left\{\begin{pmatrix}
			0 & 1\\
			0 & 1
		\end{pmatrix}, \begin{pmatrix}
			0 & 0\\
			1 & 1
		\end{pmatrix}\right\},\]

		en el cual sus vectores son independientes. Sea $\begin{pmatrix}
			0 & a\\
			b & c
		\end{pmatrix} \in V$ entonces

		\[a\begin{pmatrix}
			0 & 1\\
			0 & 1
		\end{pmatrix} + b\begin{pmatrix}
			0 & 0\\
			1 & 1
		\end{pmatrix} = \begin{pmatrix}
			0 & a\\
			b & c
		\end{pmatrix},\]

		por lo que estas vectores generan todo $V$, por lo tanto $\beta$ es una base de $V$.

		Ahora veamos que
		\begin{align*}
			T\left(\begin{pmatrix}
				0 & 1\\
				0 & 1
			\end{pmatrix}\right) &= \begin{pmatrix}
				0 & 6\\
				-2 & 4
			\end{pmatrix} = 6\begin{pmatrix}
				0 & 1\\
				0 & 1
			\end{pmatrix} - 2\begin{pmatrix}
				0 & 0\\
				1 & 1
			\end{pmatrix},\\
			T\left(\begin{pmatrix}
				0 & 0\\
				1 & 1
			\end{pmatrix}\right) &= \begin{pmatrix}
				0 & 3\\
				-1 & 2
			\end{pmatrix} = 3\begin{pmatrix}
				0 & 1\\
				0 & 1
			\end{pmatrix} - \begin{pmatrix}
				0 & 0\\
				1 & 1
			\end{pmatrix}.\\
		\end{align*}
		Por lo tanto la matriz $T$ en $\beta$ es

		\[[T]_\beta = \begin{pmatrix}
			6 & 3\\
			-2 & -1
		\end{pmatrix}.\]

		Luego veamos que

		\[T\left(\begin{pmatrix}
			0 & a\\
			b & c
		\end{pmatrix}\right) = 0,\]

		si y sólo si $3a + 3c = 0, -2a - b = 0$, es decir $a = -c, b = 2c$,	por lo tanto

		\[N(T) = \left\{\begin{pmatrix}
			0 & -c\\
			2c & c
		\end{pmatrix}, c \in \R \right\}.\]

		Una base de este es

		\[\left\{\begin{pmatrix}
			0 & -1\\
			2 & 1
		\end{pmatrix}\right\}.\]

		Por último veamos que

		\begin{align*}
			R(T) &= span(T(\beta))\\
			&= span\left(\left\{T\left(\begin{pmatrix}
				0 & 1\\
				0 & 1
			\end{pmatrix}\right),T\left(\begin{pmatrix}
				0 & 0\\
				1 & 1
			\end{pmatrix}\right)\right\}\right)\\
			&= span\left(\left\{\begin{pmatrix}
				0 & 6\\
				-2 & 4
			\end{pmatrix},\begin{pmatrix}
				0 & 3\\
				-1 & 2
			\end{pmatrix}\right\}\right).
		\end{align*}

		Ahora como

		\[2\begin{pmatrix}
			0 & 3\\
			-1 & 2
		\end{pmatrix} = \begin{pmatrix}
			0 & 6\\
			-2 & 4
		\end{pmatrix},\]

		entonces
		
		\[R(T) = span\left(\left\{\begin{pmatrix}
			0 & 3\\
			-1 & 2
		\end{pmatrix}\right\}\right),\]

		el cual es independiente y genera $R(T)$, por lo tanto concluimos que una base de $R(T)$ es

		\[\left\{\begin{pmatrix}
			0 & 3\\
			-1 & 2
		\end{pmatrix}\right\}\]
		
		\item Sea $T : \R^3 \rightarrow \R^3$ tal que $T(x,y,z) = (3x,x-y,2x+y+z)$. ¿Es $T$ invertible? Si sí, encuentra una regla para $T^{-1}$ como la dada para $T$.
		
		Sea $U : \R^3 \rightarrow \R^3$ tal que
		
		\[U(x,y,z) = \left(\frac{x}{3}, \frac{x}{3} - y, z + y - x\right).\]

		Veamos lo siguiente

		\begin{align*}
			UT &= U(T(x,y,z))\\
			&= U(3x, x-y, 2x+y+z)\\
			&= (x,y,z)
		\end{align*}

		\begin{align*}
			TU&=T(U(x,y,z))\\
			&= T\left(\frac{x}{3}, \frac{x}{3} - y, z + y - x\right)\\
			&=(x,y,z)
		\end{align*}

		Por definición $T$ es invertible y $T^{-1} = U$.

		\newpage

		\item Sea $V$ un espacio vectorial sobre $\R$ con $\dim V = 2$ y $\beta \subset V$ base ordenada. Si
		
		\[[T]_\beta = \begin{pmatrix}
			a & b\\
			c & d
		\end{pmatrix}.\]

		Probar que $T^2 - (a+d)T + (ad-bc)Id_V = T_0$.

		\begin{proof}

			Veamos como es la representación matricial de la anterior función

			\[[T^2]_\beta = [T]_\beta[T]_\beta = \begin{pmatrix}
				a^2 + bc & ab + bd\\
				ac + cd & bc + d^2
			\end{pmatrix}\]

			\[[(a+d)T]_\beta = (a+d)[T]_\beta = \begin{pmatrix}
				a^2 + ad & ab + bd\\
				ac + cd & ad + d^2
			\end{pmatrix}\]

			\[[(ad - bc)Id_V]_\beta = (ad-bc)[Id_V]_\beta = (ad-bc)\begin{pmatrix}
				1 & 0\\
				0 & 1
			\end{pmatrix} = \begin{pmatrix}
				ad - bc & 0\\
				0 & ad - bc
			\end{pmatrix}\]
			
			Entonces

			\begin{align*}
				&[T^2 - (a+d)T + (ad-bc)Id_V]_\beta\\
				&= [T^2]_\beta - [(a+d)T]_\beta + [(ad-bc)Id_V]_\beta\\
				&= \begin{pmatrix}
					a^2 + bc & ab + bd\\
					ac + cd & bc + d^2
				\end{pmatrix} - \begin{pmatrix}
					a^2 + ad & ab + bd\\
					ac + cd & ad + d^2
				\end{pmatrix} + \begin{pmatrix}
					ad - bc & 0\\
					0 & ad - bc
				\end{pmatrix}\\
				&= \begin{pmatrix}
					0 & 0\\
					0 & 0
				\end{pmatrix}.
			\end{align*}

			Como 

			\[[T_0]_\beta = \begin{pmatrix}
				0 & 0\\
				0 & 0
			\end{pmatrix},\]

			por el corolario del Teorema $2.6$ concluimos que $T^2 - (a+d)T + (ad-bc)Id_V = T_0$.
		
		\end{proof}

		\item Sea $T: \R^2 \rightarrow \R^2$ dada por $T(x,y) = (-y,x)$.
		
		\begin{enumerate}
			\item Calcula la matriz	de $T$ en la base estandar de $\R^2$.
			
			Veamos que

			\[T((1,0)) = (0,1) = 0e_1 + e_2,\]
			\[T((0,1)) = (-1,0) = -e_1 + 0e_2.\]

			Por lo tanto

			\[[T]_e = \begin{pmatrix}
				0 & -1\\
				1 & 0
			\end{pmatrix}.\]

			\item Calcula la matriz	de $T$ en la base $\gamma = \{(1,2),(1,-1)\}$.
			
			Veamos que

			\[T((1,2)) = (-2,1) = -\frac{1}{3}(1,2) - \frac{5}{3}(1,-1),\]
			\[T((1,-1)) = (1,1) = \frac{2}{3}(1,2) + \frac{1}{3}(1,-1).\]

			Por lo tanto

			\[[T]_\gamma = \begin{pmatrix}
				-\frac{1}{3} & \frac{2}{3}\\
				-\frac{5}{3} & \frac{1}{3}
			\end{pmatrix}.\]

			\item Demuestra que para todo número real $c$, $S = (T - cId_{\R^2})$ es invertible.
			
			\begin{proof}
			
				Sea $U_c: T^2 \rightarrow T^2$ dada por

				\[U_c(x,y) = \left(\frac{y - cx}{c^2 + 1}, -\frac{x + cy}{c^2 + 1}\right)\]

				Veamos lo siguiente

				\begin{align*}
					U_cS &= U_c(T - cId_{\R^2}) = U_c((-y,x) + c(x,y))\\
					&= U_c(-cx-y,x-cy)\\
					&= (x,y)
				\end{align*}

				\begin{align*}
					SU_c &= S\left(\left(\frac{y - cx}{c^2 + 1}, -\frac{x + cy}{c^2 + 1}\right)\right)\\
					&= \left(\frac{x + cy}{c^2 + 1}, \frac{y - cx}{c^2 + 1}\right) - c\left(\frac{y - cx}{c^2 + 1}, -\frac{x + cy}{c^2 + 1}\right)\\
					&= (x,y)
				\end{align*}

				Por definición $T$ es invertible y $T^{-1} = U$.

			\end{proof}

		\end{enumerate}

	\end{enumerate}

	
\end{document}
