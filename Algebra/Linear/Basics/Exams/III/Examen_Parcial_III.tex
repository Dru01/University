% Preámbulo
\documentclass[letterpaper]{article}
\usepackage[utf8]{inputenc}
\usepackage[spanish]{babel}

\usepackage{enumitem}
\usepackage{titling}

% Símbolos
	\usepackage{amsmath}
	\usepackage{amssymb}

% Márgenes
	\usepackage
	[
		margin = 1.4in
	]
	{geometry}

% Imágenes
	\usepackage{float}
	\usepackage{graphicx}
	\graphicspath{{imagenes/}}
	\usepackage{subcaption}

% Ambientes
	\usepackage{amsthm}

	\renewcommand{\qedsymbol}{$\blacksquare$}

	\theoremstyle{definition}
	\newtheorem{ejercicio}{Ejercicio}

	\newtheoremstyle{lemathm}{4pt}{0pt}{\itshape}{0pt}{\bfseries}{ --}{ }{\thmname{#1}\thmnumber{ #2}\thmnote{ (#3)}}
	\theoremstyle{lemathm}
	\newtheorem{lema}{Lema}

	\newtheoremstyle{lemademthm}{0pt}{10pt}{\itshape}{ }{\mdseries}{ --}{ }{\thmname{#1}\thmnumber{ #2}\thmnote{ (#3)}}
	\theoremstyle{lemademthm}
	\newtheorem*{lemadem}{Demostración}

% Ajustes
	\allowdisplaybreaks	% Los align pueden cambiar de página

% Macros
	\newcommand{\sumi}[2]{\sum_{i=#1}^{#2}}
	\newcommand{\dint}[2]{\displaystyle\int_{#1}^{#2}}
	\newcommand{\inte}[2]{\int_{#1}^{#2}}
	\newcommand{\dlim}{\displaystyle\lim}
	\newcommand{\limxinf}{\lim_{x\to\infty}}
	\newcommand{\limninf}{\lim_{n\to\infty}}
	\newcommand{\dlimninf}{\displaystyle\lim_{n\to\infty}}
	\newcommand{\limh}{\lim_{h\to0}}
	\newcommand{\ddx}{\dfrac{d}{dx}}
	\newcommand{\txty}{\text{ y }}
	\newcommand{\txto}{\text{ o }}
	\newcommand{\Txty}{\quad\text{y}\quad}
	\newcommand{\Txto}{\quad\text{o}\quad}
	\newcommand{\si}{\text{si}\quad}

	\newcommand{\etiqueta}{\stepcounter{equation}\tag{\theequation}}
	\newcommand{\tq}{:}
	\renewcommand{\o}{\circ}
	% \newcommand*{\QES}{\hfill\ensuremath{\boxplus}}
	% \newcommand*{\qes}{\hfill\ensuremath{\boxminus}}
	% \newcommand*{\qeshere}{\tag*{$\boxminus$}}
	% \newcommand*{\QESHERE}{\tag*{$\boxplus$}}
	\newcommand*{\QES}{\hfill\ensuremath{\blacksquare}}
	\newcommand*{\qes}{\hfill\ensuremath{\square}}
	\newcommand*{\QESHERE}{\tag*{$\blacksquare$}}
	\newcommand*{\qeshere}{\tag*{$\square$}}
	\newcommand*{\QED}{\hfill\ensuremath{\blacksquare}}
	\newcommand*{\QEDHERE}{\tag*{$\blacksquare$}}
	\newcommand*{\qel}{\hfill\ensuremath{\boxdot}}
	\newcommand*{\qelhere}{\tag*{$\boxdot$}}
	\renewcommand*{\qedhere}{\tag*{$\square$}}

	\newcommand{\abs}[1]{\left\vert#1\right\vert}
	\newcommand{\suc}[1]{\left(#1_n\right)_{n\in\N}}
	\newcommand{\en}[2]{\binom{#1}{#2}}
	\newcommand{\upsum}[2]{U(#1,#2)}
	\newcommand{\lowsum}[2]{L(#1,#2)}

	\newcommand{\N}{\mathbb{N}}
	\newcommand{\Q}{\mathbb{Q}}
	\newcommand{\R}{\mathbb{R}}
	\newcommand{\Z}{\mathbb{Z}}
	\newcommand{\eps}{\varepsilon}
	\newcommand{\ttF}{\mathtt{F}}
	\newcommand{\bfF}{\mathbf{F}}

	\newcommand{\To}{\longrightarrow}
	\newcommand{\mTo}{\longmapsto}
	\newcommand{\ssi}{\Longleftrightarrow}
	\newcommand{\sii}{\Leftrightarrow}
	\newcommand{\then}{\Rightarrow}

	\newcommand{\pTFC}{{\itshape 1er TFC\/}}
	\newcommand{\sTFC}{{\itshape 2do TFC\/}}
    
% Datos
    \title{Álgebra Lineal I\\Examen Parcial II}
    \author{Rubén Pérez Palacios\\Profesor: Rafael Herrera Guzmán}
    \date{28 Abril 2020}

% DOCUMENTO
\begin{document}
	\maketitle
    
    \section*{Problemas}

    \begin{enumerate}
        
        \item Utilizar operaciones elementales con renglones y columnas para transformar la siguiente matriz a una escalonada, y luego determinar el rango.
		
		\[A = \begin{pmatrix}
			2 & -1 & 3 & 2\\
			1 & 4 & 0 & -1\\
			2 & 6 & -1 & 5\\
		\end{pmatrix}\]

		Por el corolario del teorema 3.4 el rango de la matriz $A$ se preserva aunque se le hagan operaciones de renglon o columnas a esta, por lo que expresaremos esta matriz de una manera mas simple y poder determinar su rango.

		\begin{align*}
			\begin{pmatrix}
				2 & -1 & 3 & 2\\
				1 & 4 & 0 & -1\\
				2 & 6 & -1 & 5\\
			\end{pmatrix}
			&\rightarrow
			\begin{pmatrix}
				1 & 4 & 0 & -1\\
				2 & -1 & 3 & 2\\
				2 & 6 & -1 & 5\\
			\end{pmatrix}
			\rightarrow\\
			\begin{pmatrix}
				1 & 4 & 0 & -1\\
				0 & -9 & 3 & 4\\
				0 & -2 & -1 & 7\\
			\end{pmatrix}
			&\rightarrow
			\begin{pmatrix}
				1 & 4 & 0 & -1\\
				0 & -2 & -1 & 7\\
				0 & -9 & 3 & 4\\
			\end{pmatrix}
			\rightarrow\\
			\begin{pmatrix}
				1 & 4 & 0 & -1\\
				0 & 1 & \frac{1}{2} & -\frac{7}{2}\\
				0 & -9 & 3 & 4\\
			\end{pmatrix}
			&\rightarrow
			\begin{pmatrix}
				1 & 4 & 0 & -1\\
				0 & 1 & \frac{1}{2} & -\frac{7}{2}\\
				0 & 0 & \frac{15}{2} & -\frac{55}{2}\\
			\end{pmatrix}
			\rightarrow\\
			\begin{pmatrix}
				1 & 4 & 0 & -1\\
				0 & 1 & \frac{1}{2} & -\frac{7}{2}\\
				0 & 0 & 1 & -\frac{11}{3}\\
			\end{pmatrix}
			&\rightarrow
			\begin{pmatrix}
				1 & 0 & -2 & 13\\
				0 & 1 & \frac{1}{2} & -\frac{7}{2}\\
				0 & 0 & 1 & -\frac{11}{3}\\
			\end{pmatrix}
			\rightarrow\\
			\begin{pmatrix}
				1 & 0 & 0 & \frac{17}{3}\\
				0 & 1 & \frac{1}{2} & -\frac{7}{2}\\
				0 & 0 & 1 & -\frac{11}{3}\\
			\end{pmatrix}
			&\rightarrow
			\begin{pmatrix}
				1 & 0 & 0 & \frac{17}{3}\\
				0 & 1 & 0 & -\frac{5}{3}\\
				0 & 0 & 1 & -\frac{11}{3}\\
			\end{pmatrix}
		\end{align*}

		En las cuales podemos ver que las tres primeras columnas son linealmente independientes y la última columna puede ser expresado como combinación lineal de las otras, por lo tanto el rango es 3.

		\newpage

		\item Sí
		
		\[A = \begin{pmatrix}
			6 & -4 & 0\\
			4 & -2 & 0\\
			-1 & 0 & 3\\
		\end{pmatrix}\]		
		
		Encuentra todas las soluciones de $Ax = 2x$ y todas las soluciones de $Ax = 3x$.

		Empecemos por la soluciones de $AX = 3x$. Primero veamos que por distributividad si $x$ es solución de $Ax = 2x$ entonces también de 

		\[(A-2I)x=0,\]

		resolviendo esto por eliminación Gaussiana obtenemos

		\begin{align*}
			\begin{pmatrix}
				4 & -4 & 0\\
				4 & -4 & 0\\
				-1 & 0 & 1\\
			\end{pmatrix}
			& \rightarrow
			\begin{pmatrix}
				1 & 0 & -1\\
				0 & 1 & -1\\
				0 & 0 & 0\\
			\end{pmatrix}
		\end{align*}

		por lo tanto $x_1 = x_3$ y $x_2 = x_3$ es decir $x_1 = x_2 = x_3$, por lo tanto el conjunto de soluciones a esta ecuación es (todos los múltiplos de (1,1,1))

		\[\{x | x\in\R^3, x = (r, r, r), r \in \R\}\]

		Analogamente para las soluciones de $Ax = 3x$, vemos que esta ecuación es equivalente a

		\[(A-3I)x = 0,\]

		por eliminación Gaussiana obtenemos

		\begin{align*}
			\begin{pmatrix}
				3 & -4 & 0\\
				4 & -5 & 0\\
				-1 & 0 & 0\\
			\end{pmatrix}
			& \rightarrow
			\begin{pmatrix}
				1 & 0 & 0\\
				0 & 1 & 0\\
				0 & 0 & 0\\
			\end{pmatrix}
		\end{align*}

		por lo tanto $x_1 = x_2 = 0$ y $x_3 \in \R$, es decir nuestro conjunto solución es

		\[\{x \in \R^3| x = (0,0,r), r \in \R\}\]

		\newpage

		\item Demuestra que el sistema
		
		\begin{align*}
			x - 2y + z + 2w &= 1\\
			x + y - z + w &= 2\\
			x + 7y - 5z - w &= 3\\
		\end{align*}

		no tiene solución.

		\begin{proof}
			Veamos primero cual es el rango de la matriz de los coeficientes del sistema de ecuación anterior y de la matriz aumentada.

			\begin{align*}
				\begin{pmatrix}
					1 & -2 & 1 & 2\\
					1 & 1 & -1 & 1\\
					1 & 7 & -5 & -1\\
				\end{pmatrix}
				&& \rightarrow
				\begin{pmatrix}
					1 & -2 & 1 & 2\\
					0 & 3 & -2 & -1\\
					0 & 3 & -2 & -1\\
				\end{pmatrix}
			\end{align*}

			en esta última matriz podemos ver que el primero y el segundo renglón son independientes mientras que el último y el segundo no lo son, por lo tanto el rango de la matriz de coeficientes es 2.

			Analogamente para la matriz aumentada obtenemos

			\begin{align*}
				\begin{pmatrix}
					1 & -2 & 1 & 2 & 1\\
					1 & 1 & -1 & 1 & 2\\
					1 & 7 & -5 & -1 & 3\\
				\end{pmatrix}
				&& \rightarrow
				\begin{pmatrix}
					1 & -2 & 1 & 2 & 1\\
					0 & 3 & -2 & -1 & \frac{1}{3}\\
					0 & 3 & -2 & -1 & \frac{2}{9}\\
				\end{pmatrix}
			\end{align*}


			en esta última matriz todos los renglonnes son independientes por lo que el rango de la matriz aumentada es 3.

			Por el teorema 3. concluimos que el sistema de ecuaciones es inconsistente.

		\end{proof}

		\item Sea $V = P_2(\R)$ el espacio vectorial de los polinomios de grado a lo más 2. Sean $a_1, a_2, a_3 \in \R$ tres números reales distintos. Define los siguientes funcionales lineales $L_i : V \rightarrow \R$ dados por
		
		\[L_i(p) = p(a_i), \quad i = 1,2,3.\]

		Demuestra que $\beta^* = \{L_1,L_2,L_3\}$ es una base de $V^*$ y encuentra una base de $V$ para la cual $\beta^*$ es su base dual.

		\begin{proof}
			Sea $p(x) = a + bx + cx^2 \in \R^2$ entonces $p(a_i) = a + ba_i + ca_i^2$. Lo que deseamos es poder encontrar $c_1,c_2,c_3\in\R$ tal que

			\[p = c_1p(a_1) + c_2p(a_2) + c_3p(a_3),\]

			por lo que obtenemos el siguiente sistema de ecuaciones

			\begin{align*}
				c_1 + c_2 + c_3 &= 1\\
				c_1a_1 + c_2a_2 + c_3a_3 &= x\\
				c_1a_1^2 + c_2a_2^2 + c_3a_3^2 &= x^2 
			\end{align*}

			cuya solucion es

			\begin{align*}
				c_1 = \frac{(a_1-x)(a_3-x)}{(a_2-a_1)(a_2-a_3)} && c_2 = \frac{(a_2-x)(a_3-x)}{(a_1-a_2)(a_1-a_3)} && c_3 = \frac{(a_1-x)(a_2-x)}{(a_3-a_1)(a_3 - a_2)}
			\end{align*}

			Podemos verificar facilmente que estos tres vectores son una base para $V$ y por el teorema $2.24$ concluimos que $\beta^*$ es una base de $V^*$.

		\end{proof}

		\item Considera la transformación lineal
		
		\[T: \R^2 \rightarrow \R\]
		\[(x,y) \mapsto x - y\]

		Describe $N(T)$ y el espacio cociente $\R^2/N(T)$.

		Por definición tenemos que $N(T)$ son todos los vectores con entradas iguales es decir

		\[N(T) = \{v = (x,y) \in V| x = y\},\]

		y $\R^2/N(T)$ no es nada mas que la agrupación de las líneas paralelas a la identidad.

		Demuestra que la transformación lineal inducida por $T$:

		\begin{proof}
			\[\hat{T} : \frac{R^2}{N(T)} \rightarrow \R\]
			\[(x,y) + N(T) \mapsto T(x,y)\]

			está bien definida, es lineal y biyectiva.

			Sea $u + N(T), v + N(T) \in \frac{\R^2}{N(T)}$ tales que $u + N(T) = v + N(T)$ entonces

			\[u - v \in N(T) \Rightarrow T(u - v) = 0 \Rightarrow T(u) = T(v).\]

			Por lo tanto

			\[\hat{T}(u + N(T)) = T(u),\]

			concluimos

			\[\hat{T}(u + N(T)) = \hat{T}(v + N(T)),\]

			por lo que esta bien definida.

			Sea $\alpha \in F$, $u + N(T), v + N(T) \in \frac{\R^2}{N(T)}$, entonces

			\begin{align*}
				\hat{T}(\alpha(u + N(T)) + (v + N(T))) &= \hat{T}((\alpha u + v) + N(T))\\
				&= T(\alpha u + v)\\
				&= \alpha T(u) + T(v)\\
				&= \alpha\hat{T}(u+N(T)) + \hat{T}(v + N(T))
			\end{align*}

			Por lo que concluimos que es lineal.

			Sea $u + N(T), v + N(T) \in \frac{\R^2}{N(T)}$ tales que
			
			\[\hat{T}(u + N(T)) = \hat{T}(v + N(T)),\]
			
			entonces

			\[T(u) = T(v)\]

			y

			\[T(u -v) = 0,\]

			por lo que

			\[u \sim v,\]

			por lo tanto

			\[u + N(T) = v + N(T).\]

			Por lo que concluimos que es inyectiva.

			Sea $\alpha \in F$, $u + N(T), v + N(T) \in \frac{\R^2}{N(T)}$, entonces

			\begin{align*}
				\hat{T}(\alpha(u + N(T)) + (v + N(T))) &= \hat{T}((\alpha u + v) + N(T))\\
				&= T(\alpha u + v)\\
				&= \alpha T(u) + T(v)\\
				&= \alpha\hat{T}(u+N(T)) + \hat{T}(v + N(T))
			\end{align*}

			Por lo que concluimos que es lineal.

			Sea $x \in R$ entonces podemos ver que $(x,0) \in \R^2, T((x,0)) = x$, por lo que $\hat{T}((x,0) + N(T)) = x$ por lo tanto es suprayectiva.

			Finalmente concluimos que es biyectiva.
		
		\end{proof}
		
		\item Calcula en terminos de la base estándar
		
		\begin{enumerate}
			\item $u \wedge v$, donde $u = (-2,3)$ y $v = (-1,1)$.
			
			Por lo visto en clase sabemos que

			\[u \wedge v = (ad-bc)e_1\wedge e_2 = (-2 + 3)e_1\wedge e_2 = e_1\wedge e_2\]
		
			\item $u \wedge v \wedge w$, donde $u = (6, -4, 0)$, $v = (4, -2, 0)$ y $w = (-1,0,3)$.
			
			Por lo visto en clase sabemos que

			\[u \wedge v \wedge w = (aei - bdi - afh + cdh + bfg - ceg)e_1\wedge e_2\wedge e_3 = (-36 + 48)e_1\wedge e_2\wedge e_3\]\[= 12e_1\wedge e_2\wedge e_3\]
		\end{enumerate}

	\end{enumerate}

	
\end{document}
