% Preámbulo
\documentclass[letterpaper]{article}
\usepackage[utf8]{inputenc}
\usepackage[spanish]{babel}

\usepackage{enumitem}
\usepackage{titling}

% Símbolos
	\usepackage{amsmath}
	\usepackage{amssymb}

% Márgenes
	\usepackage
	[
		margin = 1.4in
	]
	{geometry}

% Imágenes
	\usepackage{float}
	\usepackage{graphicx}
	\graphicspath{{imagenes/}}
	\usepackage{subcaption}

% Ambientes
	\usepackage{amsthm}

	\theoremstyle{definition}
	\newtheorem{ejercicio}{Ejercicio}

	\newtheoremstyle{lemathm}{4pt}{0pt}{\itshape}{0pt}{\bfseries}{ --}{ }{\thmname{#1}\thmnumber{ #2}\thmnote{ (#3)}}
	\theoremstyle{lemathm}
	\newtheorem{lema}{Lema}

	\newtheoremstyle{lemademthm}{0pt}{10pt}{\itshape}{ }{\mdseries}{ --}{ }{\thmname{#1}\thmnumber{ #2}\thmnote{ (#3)}}
	\theoremstyle{lemademthm}
	\newtheorem*{lemadem}{Demostración}

% Ajustes
	\allowdisplaybreaks	% Los align pueden cambiar de página

% Macros
	\newcommand{\sumi}[2]{\sum_{i=#1}^{#2}}
	\newcommand{\dint}[2]{\displaystyle\int_{#1}^{#2}}
	\newcommand{\inte}[2]{\int_{#1}^{#2}}
	\newcommand{\dlim}{\displaystyle\lim}
	\newcommand{\limxinf}{\lim_{x\to\infty}}
	\newcommand{\limninf}{\lim_{n\to\infty}}
	\newcommand{\dlimninf}{\displaystyle\lim_{n\to\infty}}
	\newcommand{\limh}{\lim_{h\to0}}
	\newcommand{\ddx}{\dfrac{d}{dx}}
	\newcommand{\txty}{\text{ y }}
	\newcommand{\txto}{\text{ o }}
	\newcommand{\Txty}{\quad\text{y}\quad}
	\newcommand{\Txto}{\quad\text{o}\quad}
	\newcommand{\si}{\text{si}\quad}

	\newcommand{\etiqueta}{\stepcounter{equation}\tag{\theequation}}
	\newcommand{\tq}{:}
	\renewcommand{\o}{\circ}
	% \newcommand*{\QES}{\hfill\ensuremath{\boxplus}}
	% \newcommand*{\qes}{\hfill\ensuremath{\boxminus}}
	% \newcommand*{\qeshere}{\tag*{$\boxminus$}}
	% \newcommand*{\QESHERE}{\tag*{$\boxplus$}}
	\newcommand*{\QES}{\hfill\ensuremath{\blacksquare}}
	\newcommand*{\qes}{\hfill\ensuremath{\square}}
	\newcommand*{\QESHERE}{\tag*{$\blacksquare$}}
	\newcommand*{\qeshere}{\tag*{$\square$}}
	\newcommand*{\QED}{\hfill\ensuremath{\blacksquare}}
	\newcommand*{\QEDHERE}{\tag*{$\blacksquare$}}
	\newcommand*{\qel}{\hfill\ensuremath{\boxdot}}
	\newcommand*{\qelhere}{\tag*{$\boxdot$}}
	\renewcommand*{\qedhere}{\tag*{$\square$}}

	\newcommand{\abs}[1]{\left\vert#1\right\vert}
	\newcommand{\suc}[1]{\left(#1_n\right)_{n\in\N}}
	\newcommand{\en}[2]{\binom{#1}{#2}}
	\newcommand{\upsum}[2]{U(#1,#2)}
	\newcommand{\lowsum}[2]{L(#1,#2)}

	\newcommand{\N}{\mathbb{N}}
	\newcommand{\Q}{\mathbb{Q}}
	\newcommand{\R}{\mathbb{R}}
	\newcommand{\Z}{\mathbb{Z}}
	\newcommand{\eps}{\varepsilon}
	\newcommand{\ttF}{\mathtt{F}}
	\newcommand{\bfF}{\mathbf{F}}

	\newcommand{\To}{\longrightarrow}
	\newcommand{\mTo}{\longmapsto}
	\newcommand{\ssi}{\Longleftrightarrow}
	\newcommand{\sii}{\Leftrightarrow}
	\newcommand{\then}{\Rightarrow}

	\newcommand{\pTFC}{{\itshape 1er TFC\/}}
	\newcommand{\sTFC}{{\itshape 2do TFC\/}}
    
% Datos
    \title{Álgebra Lineal I\\Tarea 08}
    \author{Rubén Pérez Palacios\\Profesor: Rafael Herrera Guzmán}
    \date{02 Marzo 2020}

% DOCUMENTO
\begin{document}
	\maketitle
    
    \section*{Problemas}

    \begin{enumerate}
        
        \item Funciones Lineales
		
		\begin{enumerate}
			\item $V = \R^2; f((x,y)) = (2x+4y).$
			
			Sea $u,v \in V$ y $a \in F$ entonces

			\begin{align*}
				f(au + v) &= f((ax_u + x_v), (ay_u + y_v))\\
				&= (2(ax_u + x_v), 4(ay_u + y_v))\\
				&= a(2ax_u, 4y_u) + (2x_v, 4y_v))\\
				&= af(u) + f(v).
			\end{align*}

			Por lo tanto $f$ es lineal.

			\item $V = M_{2\times2}(\R); f(A) = tr(A).$
			
			Sea $u,v \in V$ y $a \in F$ entonces

			\begin{align*}
				f(au + v) &= f((ax_u + x_v), (ay_u + y_v))\\
				&= (2(ax_u + x_v), 4(ay_u + y_v))\\
				&= a(2ax_u, 4y_u) + (2x_v, 4y_v))\\
				&= af(u) + f(v).	
			\end{align*}

			Por lo tanto $f$ es lineal.

			\item $V = \R^3; f((x,y,z)) = x^2+y^2+z^2.$
			
			Sea $u,v \in V$ y $a \in F$ entonces

			\begin{align*}
				f(au + v) &= f((ax_u + x_v), (ay_u + y_v), (az_u + z_v))\\
				&= (ax_u + x_v)^2 + (ay_u + y_v)^2 + (az_u + z_v)^2\\
				&= a^2(x_u^2 + y_u^2 + z_u^2) + (x_v^2 + y_v^2 + z_v^2) + 2(x_ux_v + y_uy_v + z_uz_v)\\
				&= a^2T(u) + T(v) + 2(x_ux_v + y_uy_v + z_uz_v)\\
				&\neq af(u) + f(v).
			\end{align*}

			Por lo tanto $f$ no es lineal.

		\end{enumerate}

		\item Para los espacios vectoriales $V$ y bases $\beta$ que aparecen a continuación, encontrar la base dual $β^∗$ para $V^*$.
		
		\begin{enumerate}
			\item $V=\R^3; \beta = \{(1,0,1),(1,2,1),(0,0,1)\}.$
			
			Sea $\beta^* = \{f_1,f_2,f_3\}$ una base de dual de $V$, entonces tenemos lo siguiente

			\begin{align*}
				1 &= f_1((1,0,1)) &= f_1((e_1 + 0e_2 + e_3)) &= f_1(e_1) + f_1(e_3),\\
				0 &= f_1((1,2,1)) &= f_1((e_1 + 2e_2 + e_3)) &= f_1(e_1) + 2f_1(e_2) + f_1(e_3),\\
				0 &= f_1((0,0,1)) &= f_1((0e_1 + 0e_2 + e_3)) &= f_1(e_3),\\
			\end{align*}

			resolviendo el sistema de ecuaciones obtenemos

			\begin{align*}
				f_1(e_1) = 1 && f_1(e_2) = -\frac{1}{2} && f_1(e_3) = 0.
			\end{align*}

			Por lo tanto

			\[f_1((x,y,z)) = x - \frac{y}{2}.\]

			Ahora veamos que

			\begin{align*}
				0 &= f_2((1,0,1)) &= f_2((e_1 + 0e_2 + e_3)) &= f_2(e_1) + f_2(e_3),\\
				1 &= f_2((1,2,1)) &= f_2((e_1 + 2e_2 + e_3)) &= f_2(e_1) + 2f_2(e_2) + f_3(e_3),\\
				0 &= f_2((0,0,1)) &= f_2((0e_1 + 0e_2 + e_3)) &= f_2(e_3),\\
			\end{align*}

			resolviendo el sistema de ecuaciones obtenemos

			\begin{align*}
				f_2(e_1) = 0 && f_2(e_2) = \frac{1}{2} && f_2(e_3) = 0.
			\end{align*}

			Por lo tanto

			\[f_2((x,y,z)) = \frac{y}{2}.\]

			Por último veamos que

			\begin{align*}
				0 &= f_3((1,0,1)) &= f_3((e_1 + 0e_2 + e_3)) &= f_3(e_1) + f_3(e_3),\\
				0 &= f_3((1,2,1)) &= f_3((e_1 + 2e_2 + e_3)) &= f_3(e_1) + 2f_2(e_2) + f_3(e_3),\\
				1 &= f_3((0,0,1)) &= f_3((0e_1 + 0e_2 + e_3)) &= f_3(e_3),\\
			\end{align*}

			resolviendo el sistema de ecuaciones obtenemos

			\begin{align*}
				f_3(e_1) = -1 && f_3(e_2) = 0 && f_3(e_3) = 1.
			\end{align*}

			Por lo tanto

			\[f_3((x,y,z)) = - x + z.\]

			\item $V=P_2(\R); \beta = \{1,x,x^2\}.$
			
			Sea $\beta^* = \{f_1,f_2,f_3\}$ una base de dual de $V$, entonces tenemos lo siguiente

			\begin{align*}
				1 &= f_1(1) &= f_1(1 + 0x + 0x^2) &= f_1(e_1) + f_1(e_3),\\
				0 &= f_1(x) &= f_1(0 + x + 0x^2) &= f_1(e_1) + 2f_1(e_2) + f_1(e_3),\\
				0 &= f_1(x^2) &= f_1(0 + 0x + x^2)) &= f_1(e_3),\\
			\end{align*}

			resolviendo el sistema de ecuaciones obtenemos

			\begin{align*}
				f_1(e_1) = 1 && f_1(e_2) = -\frac{1}{2} && f_1(e_3) = 0.
			\end{align*}

			Por lo tanto

			\[f_1((x,y,z)) = x - \frac{y}{2}.\]
			
		\end{enumerate}

		\item Sea $f \in (\R^2)^*$ mediante $f((x,y)) = 2x+y$ y $T : \R^2 \rightarrow \R^2$ mediante $T(x,y) = (3x+2y,x)$.
		
		\begin{enumerate}
			\item Calcular $T^t(f)$
		\end{enumerate}

	\end{enumerate}
	
\end{document}
