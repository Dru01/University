\documentclass[10pt,a4paper]{article}
\usepackage[utf8]{inputenc}
\usepackage{amsmath}
\usepackage{amsfonts}
\usepackage{amssymb}
\usepackage{enumitem}
\usepackage{graphicx}
\usepackage{geometry}[margin = 2.4 inch]

\usepackage{amsthm}
\theoremstyle{definition}
\newtheorem*{definicion}{Definición}
\usepackage{amssymb}

\usepackage{amsthm}
\renewcommand\qedsymbol{$\blacksquare$}

\usepackage{mathtools}
\DeclarePairedDelimiter\ceil{\lceil}{\rceil}
\DeclarePairedDelimiter\floor{\lfloor}{\rfloor}

\title{Álgebra Lineal I\\Tarea 05}
\author{Rubén Pérez Palacios\\Profesor: Rafael Herrera Guzmán}
\date{02 Marzo 2020}

\newcommand{\R}{\mathbb{R}}

\begin{document}

\maketitle

\section*{Problemas}

\begin{enumerate}
    \item Sean $\beta$ y $\gamma$ las bases ordenadas estándar para $\R^n$ y $\R^m$, respectivamente. Para las siguientes transformaciones $T : \R^n \rightarrow \R^m$, calcular $[T]^\gamma_\beta$.
    
    \begin{enumerate}
        \item $T: \R^2 \rightarrow \R^3$ definida mediante $T((x,y)) = (2x-y,3x+4y,x)$.
        
        Calculemos las transformaciones lineales de $\beta$.

        \begin{align*}
            T((1,0)) &= (2,3,1)\\
            T((0,1)) &= (-1,4,0).
        \end{align*}
        
        Por lo que

        \begin{align*}
            [T]^\gamma_\beta = \begin{pmatrix}
                2 & -1\\
                3 & 4\\
                1 & 0
            \end{pmatrix}
        \end{align*}

        \item $T: \R^3 \rightarrow \R^2$ definida mediante $T((x,y,z)) = (2x+3y-z,x+z)$.
        
        Calculemos las transformaciones lineales de $\beta$.

        \begin{align*}
            T((1,0,0)) &= (2,1)\\
            T((0,1,0)) &= (3,0)\\
            T((0,0,1)) &= (-1,1).
        \end{align*}
        
        Por lo que

        \begin{align*}
            [T]^\gamma_\beta = \begin{pmatrix}
                2 & 3 & -1\\
                1 & 0 & 1
            \end{pmatrix}
        \end{align*}
 
    \end{enumerate}

    \item Sea $T : \R^2 \rightarrow \R^3$ definida como $T(x, y) = (x - y, x, 2x + y)$. Sea $\beta$ la base ordenada estándar para $\R^2$ y $\gamma = \{(1, 1, 0),(0, 1, 1),(2, 2, 3)\}$. Calcular $[T]^\gamma_\beta$. Si $\alpha = \{(1, 2),(2, 3)\}$, calcular $[T]^\gamma_\alpha$.
    
    Calculemos las transformaciones lineales de $\beta$.

    \begin{align*}
        T((1,0)) &= (1,1,2)\\
        T((0,1)) &= (-1,0,1).
    \end{align*}
    
    Luego veamos cuales son los coeficientes de la combinaciones lineales de $\gamma$ que nos dan estas transformaciones linelaes.

    \begin{align*}
        (1,1,2) &= a(1,1,0) + b(0,1,1) + c(2,2,3)\\
        (-1,0,1) &= r(1,1,0) + s(0,1,1) + t(2,2,3)
    \end{align*}

    De los cuales obtenemos el siguiente sistema de ecuaciones

    \begin{align*}
        a + 2c &= 1\\
        a + b + 2c &= 1\\
        b + 3c &= 2\\
        r + 2t &= -1\\
        r + s + 2t &= 0\\
        s + 3t &= 1\\
    \end{align*}

    cuyas soluciones son

    \begin{align*}
        a &= -\frac{1}{3} & b &= 0 & c &= \frac{2}{3}\\
        r &= -1 & s &= 1 & t &= 0
    \end{align*}

    \begin{align*}
        [T]^\gamma_\beta = \begin{pmatrix}
            -\frac{1}{3} & -1\\
            0 & 1\\
            \frac{2}{3} & 0
        \end{pmatrix}
    \end{align*}

    Calculemos las transformaciones lineales de $\alpha$.

    \begin{align*}
        T((1,2)) &= (-1,1,4)\\
        T((2,3)) &= (-1,2,7).
    \end{align*}
    
    Luego veamos cuales son los coeficientes de la combinaciones lineales de $\gamma$ que nos dan estas transformaciones linelaes.

    \begin{align*}
        (-1,1,4) &= a(1,1,0) + b(0,1,1) + c(2,2,3)\\
        (-1,2,7) &= r(1,1,0) + s(0,1,1) + t(2,2,3)
    \end{align*}

    De los cuales obtenemos el siguiente sistema de ecuaciones

    \begin{align*}
        a + 2c &= -1\\
        a + b + 2c &= 1\\
        b + 3c &= 4\\
        r + 2t &= -1\\
        r + s + 2t &= 2\\
        s + 3t &= 7\\
    \end{align*}

    cuyas soluciones son

    \begin{align*}
        a &= -\frac{7}{3} & b &= 2 & c &= \frac{2}{3}\\
        r &= -\frac{11}{3} & s &= 3 & t &= \frac{4}{3}
    \end{align*}

    \begin{align*}
        [T]^\gamma_\beta = \begin{pmatrix}
            -\frac{7}{3} & -\frac{11}{3}\\
            2 & 3\\
            \frac{2}{3} & \frac{4}{3}
        \end{pmatrix}
    \end{align*}

    \item Para los siguientes incisos, sean
    
    \[\alpha = \left\{\alpha_1 = \begin{pmatrix}
        1 & 0\\
        0 & 0
    \end{pmatrix}, \alpha_2 = \begin{pmatrix}
        0 & 1\\
        0 & 0
    \end{pmatrix}, \alpha_3 = \begin{pmatrix}
        0 & 0\\
        1 & 0
    \end{pmatrix}, \alpha_4 = \begin{pmatrix}
        0 & 0\\
        0 & 1
    \end{pmatrix}\right\}\]

    \[\beta = \{1,x,x^2\}\]

    y

    \[\gamma = \{1\}.\]

    \begin{enumerate}
        \item Defina $T: M_{2\times 2}(\R) \rightarrow M_{2\times 2}(\R)$ mediante $T(A) = A^t$, calcular $[T]_\alpha$.
        
        Calculemos las transformaciones lineales de $\beta$.

        \begin{align*}
            T(\alpha_1) &= \alpha_1\\
            T(\alpha_2) &= \alpha_3\\
            T(\alpha_3) &= \alpha_2\\
            T(\alpha_4) &= \alpha_4
        \end{align*}
        
        Por lo que

        \begin{align*}
            [T]^\gamma_\beta = \begin{pmatrix}
                1 & 0 & 0 & 0\\
                0 & 1 & 0 & 0\\
                0 & 0 & 1 & 0\\
                0 & 0 & 0 & 1
            \end{pmatrix}
        \end{align*}

        \item Defina $T: M_{2\times 2}(\R) \rightarrow \R$ mediante $T(A) = tr(A)$, calcular $[T]^\gamma_\alpha$.
        
        Calculemos las transformaciones lineales de $\beta$.

        \begin{align*}
            T(\alpha_1) &= 1\\
            T(\alpha_2) &= 0\\
            T(\alpha_3) &= 0\\
            T(\alpha_4) &= 1
        \end{align*}
        
        Por lo que

        \begin{align*}
            [T]^\gamma_\beta = \begin{pmatrix}
                1 & 0 & 0 & 1
            \end{pmatrix}
        \end{align*}

        \item Si $p(x) = 3 - 6x + x^2$, calcular $[p]_\beta$.
        
        Por lo que

        \begin{align*}
            [T]^\gamma_\beta = \begin{pmatrix}
                3 \\ -6 \\ 1
            \end{pmatrix}
        \end{align*}

    \end{enumerate}

    \item Sea $V$ un espacio vectorial de dimensión $n$ con base ordenada $\beta$. Definiendo a $T : V \rightarrow \R^n$ mediante %% $T(x) = [x]_\beta$, demostrar que $T$ es lineal.
    
    Para demostrar que $T$ es lineal entonces falta con demostrar que $cT(x) + T(y) = T(ax + y)$, veamos lo siguiente

    \[cx = c\sum_{i=1}^{n}a_i\beta_i, \quad y = \sum_{i=0}^{n}b_i\beta_i,\]

    por lo que

    \[cx + y = \sum_{i=1}^{n}(ca_i + b_i)\beta_i,\]

    por lo tanto

    %%\[c[x]_{\beta} + [y]_{\beta} = [cx+y]_{\beta}.\]

    Concluimos que

    \[T(cx + y) = cT(x) + T(y),\]

    por lo tanto $T$ es lineal.
    
\end{enumerate}

\end{document}