% Preámbulo
\documentclass[letterpaper]{article}
\usepackage[utf8]{inputenc}
\usepackage[spanish]{babel}

\usepackage{enumitem}
\usepackage{titling}

% Símbolos
	\usepackage{amsmath}
	\usepackage{amssymb}
	\usepackage{amsthm}
	\usepackage{amsfonts}
	\usepackage{mathtools}
	\usepackage{bbm}
	\usepackage[thinc]{esdiff}
	\allowdisplaybreaks

% Márgenes
	\usepackage
	[
		margin = 1.2in
	]
	{geometry}

% Imágenes
	\usepackage{float}
	\usepackage{graphicx}
	\graphicspath{{imagenes/}}
	\usepackage{subcaption}

% Ambientes
	\usepackage{amsthm}

	\theoremstyle{definition}
	\newtheorem{ejercicio}{Ejercicio}

	\newtheoremstyle{lemathm}{4pt}{0pt}{\itshape}{0pt}{\bfseries}{ --}{ }{\thmname{#1}\thmnumber{ #2}\thmnote{ (#3)}}
	\theoremstyle{lemathm}
	\newtheorem{lema}{Lema}

	\newtheoremstyle{lemathm}{4pt}{0pt}{\itshape}{0pt}{\bfseries}{ --}{ }{\thmname{#1}\thmnumber{ #2}\thmnote{ (#3)}}
	\theoremstyle{lemathm}
	\newtheorem{sol}{Solución}
	
	\newtheoremstyle{lemathm}{4pt}{0pt}{\itshape}{0pt}{\bfseries}{ --}{ }{\thmname{#1}\thmnumber{ #2}\thmnote{ (#3)}}
	\theoremstyle{lemathm}
	\newtheorem{theo}{Teorema}

	\newtheoremstyle{lemademthm}{0pt}{10pt}{\itshape}{ }{\mdseries}{ --}{ }{\thmname{#1}\thmnumber{ #2}\thmnote{ (#3)}}
	\theoremstyle{lemademthm}
	\newtheorem*{lemadem}{Demostración}

% Macros
	\newcommand{\sumi}[2]{\sum_{i=#1}^{#2}}
	\newcommand{\dint}[2]{\displaystyle\int_{#1}^{#2}}
	\newcommand{\inte}[2]{\int_{#1}^{#2}}
	\newcommand{\dlim}{\displaystyle\lim}
	\newcommand{\limtoinf}[1]{\lim_{#1\to\infty}}
	\newcommand{\dlimtoinf}[1]{\displaystyle\lim_{#1\to\infty}}
	\newcommand{\limtozero}[1]{\lim_{#1\to0}}
	\newcommand{\limh}{\lim_{h\to0}}
	\newcommand{\ddx}{\dfrac{d}{dx}}
	\newcommand{\txty}{\text{ y }}
	\newcommand{\txto}{\text{ o }}
	\newcommand{\Txty}{\quad\text{y}\quad}
	\newcommand{\Txto}{\quad\text{o}\quad}
	\newcommand{\si}{\text{si}\quad}

	\newcommand{\etiqueta}{\stepcounter{equation}\tag{\theequation}}
	\newcommand{\tq}{:}
	\renewcommand{\o}{\circ}
	\newcommand*{\QES}{\hfill\ensuremath{\blacksquare}}
	\newcommand*{\qes}{\hfill\ensuremath{\square}}
	\newcommand*{\QESHERE}{\tag*{$\blacksquare$}}
	\newcommand*{\qeshere}{\tag*{$\square$}}
	\newcommand*{\QED}{\hfill\ensuremath{\blacksquare}}
	\newcommand*{\QEDHERE}{\tag*{$\blacksquare$}}
	\newcommand*{\qel}{\hfill\ensuremath{\boxdot}}
	\newcommand*{\qelhere}{\tag*{$\boxdot$}}
	\renewcommand*{\qedhere}{\tag*{$\square$}}

	\newcommand{\suc}[1]{\left(#1_n\right)_{n\in\N}}
	\newcommand{\en}[2]{\binom{#1}{#2}}
	\newcommand{\upsum}[2]{U(#1,#2)}
	\newcommand{\lowsum}[2]{L(#1,#2)}
	\newcommand{\abs}[1]{\left| #1 \right| }
	\newcommand{\bars}[1]{\left \| #1 \right \| }
	\newcommand{\pars}[1]{\left( #1 \right) }
	\newcommand{\bracs}[1]{\left[ #1 \right] }
	\newcommand{\inprod}[1]{\left\langle #1 \right\rangle }
    \newcommand{\norm}[1]{\left\lVert#1\right\rVert}
	\newcommand{\floor}[1]{\left \lfloor #1 \right\rfloor }
	\newcommand{\ceil}[1]{\left \lceil #1 \right\rceil }
	\newcommand{\angles}[1]{\left \langle #1 \right\rangle }
	\newcommand{\set}[1]{\left \{ #1 \right\} }
	\newcommand{\norma}[2]{\left\| #1 \right\|_{#2} }

	\newcommand{\NN}{\mathbb{N}}
	\newcommand{\QQ}{\mathbb{Q}}
	\newcommand{\RR}{\mathbb{R}}
	\newcommand{\ZZ}{\mathbb{Z}}
	\newcommand{\PP}{\mathbb{P}}
    \newcommand{\EE}{\mathbb{E}}
	\newcommand{\1}{\mathbbm{1}}
	\newcommand{\eps}{\varepsilon}
	\newcommand{\ttF}{\mathtt{F}}
	\newcommand{\bfF}{\mathbf{F}}

	\newcommand{\To}{\longrightarrow}
	\newcommand{\mTo}{\longmapsto}
	\newcommand{\ssi}{\Longleftrightarrow}
	\newcommand{\sii}{\Leftrightarrow}
	\newcommand{\then}{\Rightarrow}

	\newcommand{\pTFC}{{\itshape 1er TFC\/}}
	\newcommand{\sTFC}{{\itshape 2do TFC\/}}


% Datos
    \title{Probabilidad Aplicada y Optimización \\ Tarea 1}
    \author{Rubén Pérez Palacios Lic. Computación Matemática\\Profesor: Dr. Daniel Hernández Hernández}
    \date{\today}

% DOCUMENTO
\begin{document}
	\maketitle

	\begin{enumerate}
		\item La $\sigma$-álgebra de Borel de los reales $\pars{\mathcal{B}}$ contiene a las uniones numerables de intervalos (de cualquier tipo).
		
		\begin{proof}
			Veamos que si $a,c,b \in R$ cumplen que $a\leq c \leq b$ entonces
			
			\begin{enumerate}
				\item Por definición todo intervalo abierto esta contenido en $\pars{\mathcal{B}}$.
				\item Puesto que $(-\infty,a),(b,\infty) \in \pars{\mathcal{B}}$ por definición y es cerrado bajo unión finita y complemento entonces $[a,b] \in \pars{\mathcal{B}}$.
				\item Por último puesto que $(a,c),[c,b] \in \pars{\mathcal{B}}$ y es cerrado bajo unión entonces $(a,b] \in \pars{\mathcal{B}}$. Analogamente podemos ver que $[a,b) \in \pars{\mathcal{B}}$.
			\end{enumerate}

			Debido a que $\pars{\mathcal{B}}$ es cerrado bajo uniones numerables y demostramos que todo tipo de intervalo esta contenido en $\pars{\mathcal{B}}$ concluimos que contiene a las uniones numerables de intervalos (de cualquier tipo).
		\end{proof}

		\item Prueba que la función
		
		\[
			f(x) = \begin{cases}
				\lambda e^{-\lambda x}, & x\geq 0,\\
				\hfil 0, & x < 0.
			\end{cases}
		\]

		es, en efecto una función de densidas. Esto es, que $f$ no sea negativa y que $\int_{-\infty}{\infty} f(x)dx = 1$.

		\begin{proof}
			Veamos que

			\[
				\int_{-\infty}^{\infty} f(x) = \int_{0}^{\infty} f(x) = \int_{0}^{\infty} \lambda e^{-\lambda x} = -e^{-\lambda x} \Big|_0^\infty = -(0-1) = 1.
			\]

			Puesto que $e > 0$ entonces $e^{-\lambda x} > 0$. Por lo tanto concluimos que $f$ es una función de densidad.
		\end{proof}
		\item Si $\PP\bracs{A\cap B} = \PP\bracs{A}\PP\bracs{B}$, entonces $\PP\bracs{A^c \cap B} = \PP\bracs{A^c}\PP\bracs{B}$.
		
		\begin{proof}
			\begin{align*}
				\PP\bracs{A^c \cap B} &= 1 - \PP\bracs{\pars{A^c \cap B}^c} &\text{Debido a que } \PP\bracs{X} = 1 - \PP\bracs{X^c}\\
				&= 1 - \pars{\PP\bracs{\pars{B^c \cup A}}} &\text{Ya que } \PP\bracs{\pars{A\cap B}^c} = \PP\bracs{A^c \cup B^c}\\
				&= 1 - \pars{\PP\bracs{\pars{B^c \cup A} \cap \pars{B^c \cup B}}} &\text{Puesto que } \PP\bracs{A\cup A^c} = \PP\bracs{\Omega}\\
				&= 1 - \pars{\PP\bracs{B^c \cup \pars{A\cap B}}} &\text{Por distribución}.\\
				&= 1 - \pars{\PP\bracs{B^c} + \PP\bracs{A\cap B}} &\text{Por sigmatividad de } \PP\\
				&= \PP\bracs{B} - \PP\bracs{A\cap B} &\text{Debido a que } \PP\bracs{X} = 1 - \PP\bracs{X^c}\\
				&= \PP\bracs{B} - \PP\bracs{A}\PP\bracs{B} &\text{Por hipotesis}\\
				&= \pars{1 - \PP\bracs{A}}\PP\bracs{B}\\
				&= \PP\bracs{A^c}\PP\bracs{B} &\text{Debido a que } \PP\bracs{X} = 1 - \PP\bracs{X^c}.
			\end{align*}
		\end{proof}

		\item Dada una función de distribución $F$, simular una variable aleatoria con función de distribución $F$ a partir de un generador de observaciones de una variable aleatoria uniforme en $[0,1]$.
	\end{enumerate}
\end{document}

