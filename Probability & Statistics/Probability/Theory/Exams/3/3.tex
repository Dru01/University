% Preámbulo
\documentclass[letterpaper]{article}
\usepackage[utf8]{inputenc}
\usepackage[spanish]{babel}

\usepackage{enumitem}
\usepackage{titling}

% Símbolos
	\usepackage{amsmath}
	\usepackage{amssymb}
	\usepackage{amsthm}
	\usepackage{amsfonts}
	\usepackage{mathtools}
	\usepackage{bbm}
	\usepackage[thinc]{esdiff}
	\allowdisplaybreaks

% Márgenes
	\usepackage
	[
		margin = 0.8in
	]
	{geometry}

% Imágenes
	\usepackage{float}
	\usepackage{graphicx}
	\graphicspath{{imagenes/}}
	\usepackage{subcaption}

% Ambientes
	\usepackage{amsthm}

	\theoremstyle{definition}
	\newtheorem{ejercicio}{Ejercicio}

	\newtheoremstyle{lemathm}{4pt}{0pt}{\itshape}{0pt}{\bfseries}{ --}{ }{\thmname{#1}\thmnumber{ #2}\thmnote{ (#3)}}
	\theoremstyle{lemathm}
	\newtheorem{lema}{Lema}

	\newtheoremstyle{lemathm}{4pt}{0pt}{\itshape}{0pt}{\bfseries}{ --}{ }{\thmname{#1}\thmnumber{ #2}\thmnote{ (#3)}}
	\theoremstyle{lemathm}
	\newtheorem{sol}{Solución}
	
	\newtheoremstyle{lemathm}{4pt}{0pt}{\itshape}{0pt}{\bfseries}{ --}{ }{\thmname{#1}\thmnumber{ #2}\thmnote{ (#3)}}
	\theoremstyle{lemathm}
	\newtheorem{theo}{Teorema}

	\newtheoremstyle{lemademthm}{0pt}{10pt}{\itshape}{ }{\mdseries}{ --}{ }{\thmname{#1}\thmnumber{ #2}\thmnote{ (#3)}}
	\theoremstyle{lemademthm}
	\newtheorem*{lemadem}{Demostración}

% Macros
	\newcommand{\sumi}[2]{\sum_{i=#1}^{#2}}
	\newcommand{\dint}[2]{\displaystyle\int_{#1}^{#2}}
	\newcommand{\inte}[2]{\int_{#1}^{#2}}
	\newcommand{\dlim}{\displaystyle\lim}
	\newcommand{\limtoinf}[1]{\lim_{#1\to\infty}}
	\newcommand{\dlimtoinf}[1]{\displaystyle\lim_{#1\to\infty}}
	\newcommand{\limtozero}[1]{\lim_{#1\to0}}
	\newcommand{\limh}{\lim_{h\to0}}
	\newcommand{\ddx}{\dfrac{d}{dx}}
	\newcommand{\txty}{\text{ y }}
	\newcommand{\txto}{\text{ o }}
	\newcommand{\Txty}{\quad\text{y}\quad}
	\newcommand{\Txto}{\quad\text{o}\quad}
	\newcommand{\si}{\text{si}\quad}

	\newcommand{\etiqueta}{\stepcounter{equation}\tag{\theequation}}
	\newcommand{\tq}{:}
	\renewcommand{\o}{\circ}
	\newcommand*{\QES}{\hfill\ensuremath{\blacksquare}}
	\newcommand*{\qes}{\hfill\ensuremath{\square}}
	\newcommand*{\QESHERE}{\tag*{$\blacksquare$}}
	\newcommand*{\qeshere}{\tag*{$\square$}}
	\newcommand*{\QED}{\hfill\ensuremath{\blacksquare}}
	\newcommand*{\QEDHERE}{\tag*{$\blacksquare$}}
	\newcommand*{\qel}{\hfill\ensuremath{\boxdot}}
	\newcommand*{\qelhere}{\tag*{$\boxdot$}}
	\renewcommand*{\qedhere}{\tag*{$\square$}}

	\newcommand{\suc}[1]{\left(#1_n\right)_{n\in\N}}
	\newcommand{\en}[2]{\binom{#1}{#2}}
	\newcommand{\upsum}[2]{U(#1,#2)}
	\newcommand{\lowsum}[2]{L(#1,#2)}
	\newcommand{\abs}[1]{\left| #1 \right| }
	\newcommand{\bars}[1]{\left \| #1 \right \| }
	\newcommand{\pars}[1]{\left( #1 \right) }
	\newcommand{\bracs}[1]{\left[ #1 \right] }
	\newcommand{\inprod}[1]{\left\langle #1 \right\rangle }
    \newcommand{\norm}[1]{\left\lVert#1\right\rVert}
	\newcommand{\floor}[1]{\left \lfloor #1 \right\rfloor }
	\newcommand{\ceil}[1]{\left \lceil #1 \right\rceil }
	\newcommand{\angles}[1]{\left \langle #1 \right\rangle }
	\newcommand{\set}[1]{\left \{ #1 \right\} }
	\newcommand{\norma}[2]{\left\| #1 \right\|_{#2} }

	\newcommand{\NN}{\mathbb{N}}
	\newcommand{\QQ}{\mathbb{Q}}
	\newcommand{\RR}{\mathbb{R}}
	\newcommand{\ZZ}{\mathbb{Z}}
	\newcommand{\PP}{\mathbb{P}}
    \newcommand{\EE}{\mathbb{E}}
	\newcommand{\1}{\mathbbm{1}}
	\newcommand{\eps}{Varepsilon}
	\newcommand{\ttF}{\mathtt{F}}
	\newcommand{\bfF}{\mathbf{F}}

	\newcommand{\To}{\longrightarrow}
	\newcommand{\mTo}{\longmapsto}
	\newcommand{\ssi}{\Longleftrightarrow}
	\newcommand{\sii}{\Leftrightarrow}
	\newcommand{\then}{\Rightarrow}

	\newcommand{\pTFC}{{\itshape 1er TFC\/}}
	\newcommand{\sTFC}{{\itshape 2do TFC\/}}


% Datos
    \title{Probabilidad \\ Examen Tarea 2}
    \author{Rubén Pérez Palacios Lic. Computación Matemática\\Profesor: Dr. Juan Carlos Pardo Millán}
    \date{\today}

% DOCUMENTO
\begin{document}
	\maketitle

	\begin{enumerate}
		\item Sean $X_1, X_2, X_3$ variables aleatorias independientes con valores en los enteros positivos y con función de masas de probabilidad
		
		\[\PP\bracs{X_i = k} = \pars{1-p_i}p_i^{k-1}, \quad k = 1,2,\cdots\]
		
		e $i=1,2,3$. Probar que
		
		\[\PP\bracs{X_1 < X_2 < X_3} = \frac{\pars{1-p_1}\pars{1-p_2}p_2p_3^2}{\pars{1-p_2p_3}\pars{1-p_1p_2p_3}}.\]

		Además calcule $\PP\bracs{X_1 \leq X_2 \leq X_3}$.
		
		\begin{sol}
			Notemos que

			\[\sum_{k=x+1}^{\infty} \pars{1-p}p^{k-1} = p^x(1)\]

			al ser una suma telescópica.

			Ahora por probabilidad total obtenemos que

			\[\PP\bracs{X_1 < X_2 < X_3} = \sum_{x1,x_2,x_3\in\NN} \PP\bracs{X_1 < X_2 < X_3|X_1 = x_1, X_2 = x_2, X_3 = x_3}\PP\bracs{X_1 = x_1, X_2 = x_2, X_3 = x_3},\]

			esto es lo mismo que

			\[\PP\bracs{X_1 < X_2 < X_3} = \sum_{x1,x_2,x_3\in\NN} \PP\bracs{x_1 < x_2 < x_3}\PP\bracs{X_1 = x_1, X_2 = x_2, X_3 = x_3}.\]

			Si no se cumple $x_1<x_2<x_3$ entonces $\PP\bracs{x_1 < x_2 < x_3} = 0$ y de lo contratio $\PP\bracs{x_1 < x_2 < x_3}=1$ por lo tanto

			\[\PP\bracs{X_1 < X_2 < X_3} = \sum_{1\leq x1 < x_2 < x_3} \PP\bracs{X_1 = x_1, X_2 = x_2, X_3 = x_3}.\]

			Al ser $X_1,X_2,X_3$ independientes entonces

			\[\PP\bracs{X_1 < X_2 < X_3} = \sum_{1\leq x1 < x_2 < x_3} \PP\bracs{X_1 = x_1}\PP\bracs{X_2 = x_2}\PP\bracs{X_3 = x_3},\]

			por distribución de $X_3$ tenemos que

			\[\PP\bracs{X_1 < X_2 < X_3} = \sum_{1 \leq x1 < x_2 < x_3} \PP\bracs{X_1 = x_1}\PP\bracs{X_2 = x_2}\pars{1-p_3}p_3^{x_3-1}.\]

			Por (1) obtenemos

			\[\PP\bracs{X_1 < X_2 < X_3} = \sum_{1 \leq x1 < x_2} \PP\bracs{X_1 = x_1}\PP\bracs{X_2 = x_2}p_3^{x_2},\]

			nuevamente por distribución de $X_3$ tenemos que

			\[\PP\bracs{X_1 < X_2 < X_3} = \sum_{1 \leq x1 < x_2} \PP\bracs{X_1 = x_1}\pars{1-p_i}p_i^{x_2-1}p_3^{x_2},\]

			lo cuál es lo mismo que

			\[\PP\bracs{X_1 < X_2 < X_3} = \frac{\pars{1-p_2}p_3}{\pars{1-p_2p_3}}\sum_{1 \leq x1 < x_2} \PP\bracs{X_1 = x_1}\pars{1-p_2p_3}\pars{p_2p_3}^{x_2-1}.\]

			Aplicando de nuevo (1) obtenemos

			\[\PP\bracs{X_1 < X_2 < X_3} = \frac{\pars{1-p_2}p_3}{\pars{1-p_2p_3}}\sum_{1 \leq x1} \PP\bracs{X_1 = x_1}\pars{p_2p_3}^{x_1},\]

			sustiyendo la función de masas de $X_1$ obtenemos

			\[\PP\bracs{X_1 < X_2 < X_3} = \frac{\pars{1-p_2}p_3}{\pars{1-p_2p_3}}\sum_{1 \leq x1} \pars{1-p_1}p_1^{x_1-1}\pars{p_2p_3}^{x_1},\]

			reorganizando tenemos que

			\[\PP\bracs{X_1 < X_2 < X_3} = \frac{\pars{1-p_1}\pars{1-p_2}p_2p_3^2}{\pars{1-p_2p_3}\pars{1-p_1p_2p_3}}\sum_{1 \leq x1} \pars{1-p_1p_2p_3}\pars{p_1p_2p_3}^{x_1-1},\]

			por último aplicando (1) concluimos que

			\[\PP\bracs{X_1 < X_2 < X_3} = \frac{\pars{1-p_1}\pars{1-p_2}p_2p_3^2}{\pars{1-p_2p_3}\pars{1-p_1p_2p_3}}.\]

			Analogamente veamos que

			\begin{align*}
				\PP\bracs{X_1 < X_2 < X_3} &= \sum_{1\leq x1 < x_2 < x_3} \PP\bracs{X_1 = x_1}\PP\bracs{X_2 = x_2}\PP\bracs{X_3 = x_3} & \text{Por prob. total e indep.}\\
				&= \sum_{1\leq x1 < x_2 < x_3} \pars{1-p_1}\pars{1-p_2}\pars{1-p_3}p_1^{x_1-1}p_2^{x_2-1}p_3^{x_3-1} & \text{Por distribución de las $X_i$'s}\\
				&= \sum_{1\leq x1 < x_2} \pars{1-p_1}\pars{1-p_2}p_1^{x_1-1}p_2^{x_2-1}p_3^{x_2-1} & \text{Por (1)}\\
				&= \frac{\pars{1-p_1}\pars{1-p_2}}{\pars{1-p_2p_3}}\sum_{1\leq x1} p_1^{x_1-1}p_2^{x_1-1}p_3^{x_1-1} & \text{Por (1)}\\
				&= \frac{\pars{1-p_1}\pars{1-p_2}}{\pars{1-p_2p_3}\pars{1-p_1p_2p_3}} & \text{Por (1)}
			\end{align*}
		\end{sol}

		\newpage
		\item Definamos la varianza condicional de $\xi$ dado $\eta$ por
		\[Var\pars{\xi|\eta} = \EE\bracs{\xi^2|\eta}-\pars{\EE\bracs{\xi|\eta}}^2.\]
		Sean $\xi$ y $\eta$ dos v.a. con segundo momento finito.
		\begin{enumerate}
			\item Muestre que
			\[Var\pars{\xi} = \EE\bracs{Var\pars{\xi|\eta}} + Var\pars{\EE\bracs{\xi|\eta}}.\]
			\begin{sol}
				Empezaremos por desarrollar el primer sumando del termino derecho de nuestra expresión. Por la ley de esperanza total y linealidad de la esperanza obtenemos

				\[\EE\bracs{Var\pars{\xi|\eta}} = \EE\bracs{\EE\bracs{\xi^2|\eta}-\pars{\EE\bracs{\xi|\eta}}^2} = \EE\bracs{\EE\bracs{\xi^2|\eta}}-\EE\bracs{\pars{\EE\bracs{\xi|\eta}}^2} = \EE\bracs{\xi^2}-\EE\bracs{\pars{\EE\bracs{\xi|\eta}}^2},\]

				ahora desarrollaremos el segundo sumando del termino derecho de nuestra expresión. Por definición de varianza y la ley de esperanza total obtenemos

				\[Var\pars{\EE\bracs{\xi|\eta}} = \EE\bracs{\pars{\EE\bracs{\xi|\eta}}^2} - \EE\bracs{\EE\bracs{\xi|\eta}}^2 = \EE\bracs{\pars{\EE\bracs{\xi|\eta}}^2} - \EE\bracs{\xi}^2.\]

				Haciendo uso de los resultados anteriores y por definición de varianza concluimos que

				\[\EE\bracs{Var\pars{\xi|\eta}} + Var\pars{\EE\bracs{\xi|\eta}} = \EE\bracs{\xi^2}-\EE\bracs{\pars{\EE\bracs{\xi|\eta}}^2} + \EE\bracs{\pars{\EE\bracs{\xi|\eta}}^2} - \EE\bracs{\xi}^2 = \EE\bracs{\xi^2} - \EE\bracs{\xi}^2 = Var\pars{\xi}.\]
			\end{sol}
			\item Sea $\Delta$ otra v.a. Muestre la siguiente formula
			\[Cov\pars{\xi,\eta} = \EE\bracs{Cov\pars{\pars{\xi,\eta}|\Delta}} + Cov\pars{\EE\bracs{\xi|\Delta},\EE\bracs{\eta|\Delta}},\]
			donde
			\[Cov\pars{\pars{\xi,\eta}|\Delta} = \EE\bracs{\xi\eta|\Delta}-\EE\bracs{\xi|\Delta}\EE\bracs{\eta|\Delta}.\]
			\begin{sol}
				Empezaremos por desarrollar el primer sumando del termino derecho de nuestra expresión. Por definición de covarianza condicional, la ley de esperanza total y linealidad de la esperanza obtenemos

				\[\EE\bracs{Cov\pars{\pars{\xi,\eta}|\Delta}} = \EE\bracs{\EE\bracs{\xi\eta|\Delta}-\EE\bracs{\xi|\Delta}\EE\bracs{\eta|\Delta}} = \EE\bracs{\xi\eta}-\EE\bracs{\EE\bracs{\xi|\Delta}\EE\bracs{\eta|\Delta}}\]

				ahora desarrollaremos el segundo sumando del termino derecho de nuestra expresión. Por definición de covarianza y la ley de esperanza total obtenemos

				\[Cov\pars{\EE\bracs{\xi|\Delta},\EE\bracs{\eta|\Delta}} = \EE\bracs{\EE\bracs{\xi|\Delta}\EE\bracs{\eta|\Delta}} - \EE\bracs{\EE\bracs{\xi|\Delta}}\EE\bracs{\EE\bracs{\eta|\Delta}} = \EE\bracs{\EE\bracs{\xi|\Delta}\EE\bracs{\eta|\Delta}} - \EE\bracs{\xi|\Delta}\EE\bracs{\eta|\Delta}.\]

				Haciendo uso de los resultados anteriores y por definición de covarianza concluimos que

				\[\EE\bracs{Cov\pars{\pars{\xi,\eta}|\Delta}} + Cov\pars{\EE\bracs{\xi|\Delta},\EE\bracs{\eta|\Delta}} = \EE\bracs{\xi\eta} - \EE\bracs{\xi|\Delta}\EE\bracs{\eta|\Delta} = Cov\pars{\xi,\eta}.\]
			\end{sol}
		\end{enumerate}
	\end{enumerate}
\end{document}

