% Preámbulo
\documentclass[letterpaper]{article}
\usepackage[utf8]{inputenc}
\usepackage[spanish]{babel}

\usepackage{enumitem}
\usepackage{titling}

% Símbolos
	\usepackage{amsmath}
	\usepackage{amssymb}
	\usepackage{amsthm}
	\usepackage{amsfonts}
	\usepackage{mathtools}
	\usepackage{bbm}
	\usepackage[thinc]{esdiff}
	\allowdisplaybreaks

% Márgenes
	\usepackage
	[
		margin = 1.2in
	]
	{geometry}

% Imágenes
	\usepackage{float}
	\usepackage{graphicx}
	\graphicspath{{imagenes/}}
	\usepackage{subcaption}

% Ambientes
	\usepackage{amsthm}

	\theoremstyle{definition}
	\newtheorem{ejercicio}{Ejercicio}

	\newtheoremstyle{lemathm}{4pt}{0pt}{\itshape}{0pt}{\bfseries}{ --}{ }{\thmname{#1}\thmnumber{ #2}\thmnote{ (#3)}}
	\theoremstyle{lemathm}
	\newtheorem{lema}{Lema}

	\newtheoremstyle{lemathm}{4pt}{0pt}{\itshape}{0pt}{\bfseries}{ --}{ }{\thmname{#1}\thmnumber{ #2}\thmnote{ (#3)}}
	\theoremstyle{lemathm}
	\newtheorem{sol}{Solución}
	
	\newtheoremstyle{lemathm}{4pt}{0pt}{\itshape}{0pt}{\bfseries}{ --}{ }{\thmname{#1}\thmnumber{ #2}\thmnote{ (#3)}}
	\theoremstyle{lemathm}
	\newtheorem{theo}{Teorema}

	\newtheoremstyle{lemademthm}{0pt}{10pt}{\itshape}{ }{\mdseries}{ --}{ }{\thmname{#1}\thmnumber{ #2}\thmnote{ (#3)}}
	\theoremstyle{lemademthm}
	\newtheorem*{lemadem}{Demostración}

% Macros
	\newcommand{\sumi}[2]{\sum_{i=#1}^{#2}}
	\newcommand{\dint}[2]{\displaystyle\int_{#1}^{#2}}
	\newcommand{\inte}[2]{\int_{#1}^{#2}}
	\newcommand{\dlim}{\displaystyle\lim}
	\newcommand{\limxinf}{\lim_{x\to\infty}}
	\newcommand{\limninf}{\lim_{n\to\infty}}
	\newcommand{\dlimninf}{\displaystyle\lim_{n\to\infty}}
	\newcommand{\limh}{\lim_{h\to0}}
	\newcommand{\ddx}{\dfrac{d}{dx}}
	\newcommand{\txty}{\text{ y }}
	\newcommand{\txto}{\text{ o }}
	\newcommand{\Txty}{\quad\text{y}\quad}
	\newcommand{\Txto}{\quad\text{o}\quad}
	\newcommand{\si}{\text{si}\quad}

	\newcommand{\etiqueta}{\stepcounter{equation}\tag{\theequation}}
	\newcommand{\tq}{:}
	\renewcommand{\o}{\circ}
	\newcommand*{\QES}{\hfill\ensuremath{\blacksquare}}
	\newcommand*{\qes}{\hfill\ensuremath{\square}}
	\newcommand*{\QESHERE}{\tag*{$\blacksquare$}}
	\newcommand*{\qeshere}{\tag*{$\square$}}
	\newcommand*{\QED}{\hfill\ensuremath{\blacksquare}}
	\newcommand*{\QEDHERE}{\tag*{$\blacksquare$}}
	\newcommand*{\qel}{\hfill\ensuremath{\boxdot}}
	\newcommand*{\qelhere}{\tag*{$\boxdot$}}
	\renewcommand*{\qedhere}{\tag*{$\square$}}

	\newcommand{\suc}[1]{\left(#1_n\right)_{n\in\N}}
	\newcommand{\en}[2]{\binom{#1}{#2}}
	\newcommand{\upsum}[2]{U(#1,#2)}
	\newcommand{\lowsum}[2]{L(#1,#2)}
	\newcommand{\abs}[1]{\left| #1 \right| }
	\newcommand{\bars}[1]{\left \| #1 \right \| }
	\newcommand{\pars}[1]{\left( #1 \right) }
	\newcommand{\bracs}[1]{\left[ #1 \right] }
	\newcommand{\inprod}[1]{\left\langle #1 \right\rangle }
    \newcommand{\norm}[1]{\left\lVert#1\right\rVert}
    \newcommand{\floor}[1]{\left \lfloor #1 \right\rfloor }
	\newcommand{\ceil}[1]{\left \lceil #1 \right\rceil }
	\newcommand{\angles}[1]{\left \langle #1 \right\rangle }
	\newcommand{\set}[1]{\left \{ #1 \right\} }
	\newcommand{\norma}[2]{\left\| #1 \right\|_{#2} }


	\newcommand{\NN}{\mathbb{N}}
	\newcommand{\QQ}{\mathbb{Q}}
	\newcommand{\RR}{\mathbb{R}}
	\newcommand{\ZZ}{\mathbb{Z}}
	\newcommand{\PP}{\mathbb{P}}
    \newcommand{\EE}{\mathbb{E}}
	\newcommand{\1}{\mathbbm{1}}
	\newcommand{\eps}{\varepsilon}
	\newcommand{\ttF}{\mathtt{F}}
	\newcommand{\bfF}{\mathbf{F}}

	\newcommand{\To}{\longrightarrow}
	\newcommand{\mTo}{\longmapsto}
	\newcommand{\ssi}{\Longleftrightarrow}
	\newcommand{\sii}{\Leftrightarrow}
	\newcommand{\then}{\Rightarrow}

	\newcommand{\pTFC}{{\itshape 1er TFC\/}}
	\newcommand{\sTFC}{{\itshape 2do TFC\/}}


% Datos
    \title{Probabilidad \\ Tarea 1}
    \author{Rubén Pérez Palacios Lic. Computación Matemática\\Profesor: Dr. Antonio Murillo Salas}
    \date{\today}

% DOCUMENTO
\begin{document}
	\maketitle

	\textbf{Nota:} Justifique con claridad cada una de las respuestas.

	\begin{enumerate}
		\item Sea $\pars{X,Y}$ un vector aleatorio con función de probabilidades conjunta
		
		\[f_{X,Y}(x,y) = \begin{cases}
			cx^2, &x,y\in\set{1,\cdots,N},\\
			0, &\text{en otro caso},
		\end{cases}\]

		donde $N$ es un número natural y $c$ una constante positiva.

		\begin{enumerate}
			\item Determine el valor de $c$.
			
			\begin{sol}
				Por definición de función de probabilidades tenemos que

				\[\sum_{x=1}^N\sum_{y=1}^N f_{X,Y}(x,y) = 1,\]

				entonces

				\[1 = \sum_{x=1}^N\sum_{y=1}^N f_{X,Y}(x,y) = \sum_{x=1}^N\sum_{y=1}^N cx^2 = Nc \frac{N(N+1)(2N+1)}{6},\]

				por lo tanto

				\[c = \frac{6}{N^2(N+1)(2N+1)}.\]
			\end{sol}

			\item Encuentre las funciones de probabilidades marginal de $X$ y $Y$.
			
			\begin{sol}
				Por definición de función de probabilidades tenemos que

				\[f_X(x) = \sum_{y=1}^N f_{X,Y}(x,y),\]

				entonces

				\[f_X(x) = \sum_{y=1}^N f_{X,Y}(x,y) = \sum_{y=1}^N \frac{6}{N^2(N+1)(2N+1)}x^2 = \frac{6x^2}{N(N+1)(2N+1)},\]

				claro si $x \in \set{1,\cdots,N}$ en otro caso $0$, y análogamente

				\[f_Y(y) = \sum_{x=1}^N f_{X,Y}(x,y) = \sum_{x=1}^N \frac{6}{N^2(N+1)(2N+1)}x^2 = \frac{1}{N}.\]

				si $y \in \set{1,\cdots,N}$ en otro caso $0$,
			\end{sol}

			\item ¿Son $X$ y $Y$ variables aleatorias independientes? 
			
			\begin{proof}
				Puesto que

				\[f_{X,Y}(x,y) = \frac{6}{N^2(N+1)(2N+1)}x^2 = \frac{6x^2}{N(N+1)(2N+1)} \frac{1}{N} = f_X(x) f_Y(y),\]

				concluimos que $X$ y $Y$ son independientes.
			\end{proof}
		\end{enumerate}

		\item Suponga que $\pars{X,Y}$ un vector aleatorio con función de probabilidades conjunta dada por
		
		\[f_{X,Y}(x,y) = \begin{cases}
			cx, & x,y\in\set{1,\cdots,N}, x\leq y^2,\\
			0, &\text{en otro caso},
		\end{cases}\]

		donde $N$ es un número natural y $c$ una constante positiva. Encuentre la función de probabilidades marginal de $X$ y $Y$.

		\begin{sol}

			Comenzaremos por encontrar el valor de $c$. Por definición de función de probabilidades tenemos que

			\[\sum_{y=1}^{N^2}\sum_{x=1}^{y^2} f_{X,Y}(x,y) = 1,\]

			entonces

			\begin{align*}
				1 &= \sum_{y=1}^{N^2}\sum_{x=1}^{y^2} f_{X,Y}(x,y)\\
				&= \sum_{y=1}^{N^2}\sum_{x=1}^{y^2} cx\\
				&= c\sum_{y=1}^{N^2} \frac{y^2(y^2+1)}{2}\\
				&= \frac{c}{2} \sum_{y=1}^{N^2} y^4 + y^2\\
				&= \frac{c}{2} \bracs{\frac{N^2(N^2+1)(2N^2+1)(3N^4+3N^2-1)}{30} + \frac{N^2(N^2+1)(2N^2+1)}{6}}\\
				&= \frac{cN^2(N^2+1)(2N^2+1)(3N^4+3N^2+4)}{60}.
			\end{align*}

			por lo tanto

			\[c = \frac{60}{N^2(N^2+1)(2N^2+1)(3N^4+3N^2+4)}.\]

			Ahora veamos que por definición de función de probabilidades tenemos que

			\[f_X(x) = \sum_{y=\ceil{\sqrt{x}}}^{N^2} f_{X,Y}(x,y),\]

			entonces

			\begin{align*}
				f_X(x) &= \sum_{y=\ceil{\sqrt{x}}}^{N^2} f_{X,Y}(x,y)\\
				&= \sum_{y=\ceil{\sqrt{x}}}^{N^2} \frac{60}{N^2(N^2+1)(2N^2+1)(3N^4+3N^2+4)}x\\
				&= \frac{60x}{N^2(N^2+1)(2N^2+1)(3N^4+3N^2+4)} \pars{N-\ceil{\sqrt{x}} + 1},
			\end{align*}

			claro si $x \in \set{1,\cdots,N^2}$ en otro caso $0$, y análogamente

			\begin{align*}
				f_Y(y) &= \sum_{x=1}^{\min\pars{y^2,N^2}} f_{X,Y}(x,y)\\
				&= \sum_{x=1}^{\min\pars{y^2,N^2}} \frac{60}{N^2(N^2+1)(2N^2+1)(3N^4+3N^2+4)}x\\
				&= \frac{60}{N^2(N^2+1)(2N^2+1)(3N^4+3N^2+4)} \sum_{x=1}^{\min\pars{y^2,N^2}} x\\
				&= \frac{30\pars{\min\pars{y^2,N^2}(\min\pars{y^2,N^2}+1)}}{N^2(N^2+1)(2N^2+1)(3N^4+3N^2+4)}.
			\end{align*}

			si $y \in \set{1,\cdots,N^2}$ en otro caso $0$.

			Notemos que en este caso $X$ y $Y$ no son independientes.
			
		\end{sol}

		\item Suponga que $2N$ bolas se colocan al azar en cualquiera de $N$ cajas disponibles. Sea $X_i$ el número de bolas que quedan en la caja $i$. Encuentre la función de probabilidades conjunta de $X_1,\cdots,X_N$.
		
		\begin{sol}
			Primero veamos que el total de formas de acomodar las $2N$ bolas en las $N$ cajas por seperadores es

			\[A = \binom{3N-1}{N-1}.\]

			% Ahora contaremos las formas de acomodar las $2N$ bolas en las $N$ cajas tal que si la caja $i$ tiene $a_i$ pelotas entonces $a_i\leq x_i$ con $x_i \in \NN_{0}$ tales que $x_1+\cdots+x_N \geq 2N$, $i = 1,\cdots, N$. Definamos $b_i = x_i - a_i \geq 0$ luego entonces $b_1+\cdots+b_N = \pars{x_1+\cdots+x_N} - a_1+\cdots+a_N = \pars{x_1+\cdots+x_N} - 2N \geq 0$, por lo que para contar lo anterior es lo análogo a contar de cuantas formas podemos acomodar $\pars{x_1+\cdots+x_N} - 2N$ objetos en $N$ urnas, por separadores tenemos que son

			% \[B = \binom{\pars{x_1+\cdots+x_N} - 2N + N - 1}{N-1} = \binom{\pars{x_1+\cdots+x_N} - N - 1}{N - 1}.\]

			Por lo tanto concluimos que la función de distirbución conjunta de $X_1,\cdots,X_N$ es

			\[f_{x_1,\cdots,X_n}(x_1,\cdots,x_N) = \begin{cases}
				\frac{1}{\binom{3N-1}{N-1}} & x_1+\cdots+x_N = 2N\\
				0 & \text{en otro caso}.
			\end{cases}\]
		\end{sol}
		
		\item Considere un experimento aleatorio que tiene tres posibles resultados con probabilidades $p_1$,$p_2$ y $p_3$, respectivamente. Suponga que se realizan $N$ ensayos independientes y denote por $X_i$ el número de veces que ocurre el resultado $i$.
		
		\begin{enumerate}
			\item Determine la distribución de $X_1+X_2$.
			
			\begin{sol}

				Primero veamos que este experimento lo podemos ver para cada resultado $i$ con $i = 1,2,3$ como un experimento bernoulli (donde el experimento se considera exitoso si el resutlado es $i$ y fallido si no) con probabilidad $p = p_i$ y $q = 1 - p_i$, luego $X_i$ es la suma de $N$ ensayos independientes de estos experimentos bernoullis, por definición tenemos que $X_i\sim Binomial(N,p_i)$, entonces la suma de binomiales es $X_1+X_2\sim Binomial(N,p_1+p_2)$.

			\end{sol}

			\item Encuentre $\PP\bracs{X_2 = y | X_1+X_2=z}, y = 0,1,\cdots,z$.
			
			\begin{sol}
				
				Si $X_1+X_2=z$ entonces sabemos que de $z$ ensayos de un experimento tuvieron resultados $1,2$, ahora cada uno de esos $z$ ensayos son de un experimento bernoulli con dos posibles resultados $1,2$. Ahora calcularemos la probabilidad de que en un experimento $E$ el resultado sea $2$, por probabilidad condicional

				\[\PP\bracs{E=2 | E\in\set{1,2}} = \frac{\PP[E=2,E\in\set{1,2}]}{\PP\bracs{E\in\set{1,2}}} = \frac{\PP\bracs{E=2}}{\PP\bracs{E\in\set{1,2}}} = \frac{p_2}{p_1+p_2}.\]

				Entonces la variable aleatoria $X_2 = y | X_1+X_2=z$ tiene una distribución Binomial con parametros $z, \frac{p_2}{p_1+p_2}$. Por lo tanto concluimos que

				\[\PP\bracs{X_2 = y | X_1+X_2=z} = \binom{z}{y} \pars{\frac{p_2}{p_1+p_2}}^{y}\pars{\frac{p_1}{p_1+p_2}}^{z-y}.\]

			\end{sol}

		\end{enumerate}
	\end{enumerate}
\end{document}

