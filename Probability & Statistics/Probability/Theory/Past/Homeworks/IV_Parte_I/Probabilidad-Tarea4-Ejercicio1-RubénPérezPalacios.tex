\documentclass[letterpaper, 12pt]{article}
\usepackage[utf8]{inputenc}
\usepackage{amsmath,amssymb,latexsym}
\usepackage{enumerate}
\textheight=25cm \textwidth=18cm \topmargin=-2cm \oddsidemargin=-1cm
\usepackage[inner=2.0cm,outer=2.0cm,top=2.5cm,bottom=2.5cm]{geometry}
\usepackage{amsmath,amssymb,latexsym}
\usepackage{graphicx}
\usepackage{enumerate}
\usepackage[spanish,mexico]{babel}
\usepackage{setspace}
\usepackage{verbatim}
\usepackage{setspace}
\usepackage{amsthm}
\usepackage{multicol}
\usepackage{mathtools}

\DeclarePairedDelimiter{\ceil}{\lceil}{\rceil}
\newcommand{\Ls}{\lim_{n\to \infty}}
\newcommand{\A}{\mathcal{A}}
\newcommand{\one}{\mathbf{1}}
\newcommand{\B}{\mathcal{B}}
\renewcommand{\L}{\mathcal{L}}
\newcommand{\Lp}{\mathcal{L}_p}
\newcommand{\E}{\mathbb{E}}
\newcommand{\R}{\mathbb{R}}
\newcommand{\N}{\mathbb{N}}
\renewcommand{\P}{\mathbb{P}}
\newcommand{\F}{\mathcal{F}}
\newcommand{\abs}[1]{\left\vert#1\right\vert}
\newcommand{\pac}[1]{\left[#1\right]}
\newcommand{\norm}[1]{\left\Vert#1\right\Vert}
\newcommand{\pint}[1]{\langle #1 \rangle}
\newcommand{\eps}{\varepsilon}
\newcommand{\To}{\longrightarrow}
\newcommand{\iden}{\setlength{\parindent}{12pt}}
\newcommand{\mat}{M_{n\times n}(\F)}
\newcommand{\pai}{\left(}
\newcommand{\pad}{\right)}

\newcommand{\solfin}{\hfill\diamondsuit}
\newcommand{\solin}{\iden\textit{Solución: }}
\newtheorem{lemma}{Lema}
\newtheorem{notac}{Notación}
\newtheorem{theorem}{Teorema}
\newtheorem{obs}{Observación}
\newtheorem{prop}{Proposición}
\newtheorem{enunciado}{Enunciado}


\title{Tarea 4\\Probabilidad\\Ejercicio 1}
\author{
    Ricardo Alberto Gloria Picazzo\\
    Rubén Pérez Palacios\\
    Mercé Nachón Moreno
}
\date{Octubre 2020}
\begin{document}
\maketitle

(Ricardo Alberto Gloria Picazzo)
\begin{lemma}
    Si $\{X_n\}_{n\in \N}$ es una sucesión de variables aleatorias iid con distribución $Poisson(\lambda)$, entonces:
    \[
        \frac{\overline{X_n}-\lambda}{\sqrt{\overline{X_n}}/\sqrt{n}} \overset{d}{\to} Z
        \]
    dónde $Z\sim N(0, 1)$.
\end{lemma}
\begin{proof}
    Como $X_n \sim  Poisson(\lambda)$ para $n\in \N$, entonces 
    \[
        \E[X_n] = Var(X_n) = \lambda
        \] 
    para toda $n\in \N$. Luego, por el Teorema del Límite central se sigue que 
    \begin{align}
        \frac{\overline{X_n}-\lambda}{\sqrt{\lambda}/\sqrt{n}} \overset{d}{\to} Z
    \end{align}
    dónde $Z\sim N(0, 1)$.

    Por otra parte, por la ley debil de los grandes números se sigue que $\overline{X_n}  \overset{P}{\to} \lambda$. Aplicando el Teorema del mapeo continuo son la función $g_1(x) = 1/x$ se obtiene que $1/(\overline{X_n})  \overset{P}{\to} 1/\lambda$. Luego, aplicando nuevamente el teorema del mapeo continuo con la función $g_2(x)=\sqrt{x}$ se sigue que  $1/\sqrt{\overline{X_n}}  \overset{P}{\to} 1/\sqrt{\lambda}$. Finalemnte aplicando el teorema del mapeo continuo con la función continua $g_3(x)=\sqrt{\lambda}\cdot x$ se obtiene que 
    \begin{align}
        \frac{\sqrt{\lambda}}{\sqrt{\overline{X_n}}} \overset{P}{\to} 1.
    \end{align}

    Así aplicando el tercer literal del Toerema de Slutsky con (1) y (2) se obtiene que 
    \begin{align*}
        \frac{\overline{X_n}-\lambda}{\sqrt{\overline{X_n}}/\sqrt{n}} \overset{d}{\to} Z
    \end{align*}
    dónde $Z\sim N(0, 1)$.
\end{proof}



\begin{enunciado}
    El medidor de temperatura de cierta maquinaria presenta fallas a lo largo del día. El archivo ``tarea4e3.txt'' contiene una muestra de fallas registradas durante 718 días distintos. Suponiendo que las fallas en cada día son independientes ¿se puede decir que el medidor presenta en promedio más de 5 fallas y menos de 10 fallas?
\end{enunciado}



\solin Dado el contexto del problema, modelemos la situación con variables aleatorias con distribución Poisson. Especificamente, modelemos el número de fallas en el $n$-ésimo día con una variable $X_n\sim Poisson(\lambda)$. Observamos así que $\{X_n\}_{n\in \N}$ es una sucesión de variables aleatorias identicamente distribuidas e independientes con espereanza y varianza $\lambda$. Se busca el promedio de fallas del medidor en un día. Hallemos, entonces un intervalo de $95\%$ de confianza para el parametro $\lambda$, el cual, como ya se mencionó, es el valor esperado de $X_n$ para cada $n\in \N$.

Veamos que si $(a, b)\subset \R$ es un intervalo, entonces 
\begin{align}
    \P\left[a < \frac{\overline{X_n}-\lambda}{\sqrt{\overline{X_n}}/\sqrt{n}} < b \right] = 
    \P\pac{\frac{\overline{X_n}-\lambda}{\sqrt{\overline{X_n}}/\sqrt{n}} < b } - \P\pac{\frac{\overline{X_n}-\lambda}{\sqrt{\overline{X_n}}/\sqrt{n}} \leq a}.
\end{align}
Del Lema 1 recordamos:
\begin{align*}
    \frac{\overline{X_n}-\lambda}{\sqrt{\overline{X_n}}/\sqrt{n}} \overset{d}{\to} Z
\end{align*}
dónde $Z\sim N(0, 1)$.  Nos intersa el caso $n=718$, por lo que considerando $n=718$ como ``suficientemente grande'', obtenemos de (3) que
\begin{align}
    \P\left[a < \frac{\overline{X_{718}}-\lambda}{\sqrt{\overline{X_{718}}}/\sqrt{718}} < b \right]  
    &\approx \P[Z < b] - \P[Z \leq a] = F_Z(b) - F_Z(a).
\end{align}
Luego, se desea un intervalo del $95\%$ de confianza. Es decir, $a$ y $b$ son tales que 
\begin{align}
    \P\left[a < \frac{\overline{X_{718}}-\lambda}{\sqrt{\overline{X_{718}}}/\sqrt{718}} < b \right] \approx 1-\alpha = 0.95
\end{align}
con $\alpha = 0.05$. Elijamos así $a$ y $b$ de modo que $F_Z(b) = 1-0.025 = 0.975$ y $F_Z(a)=0.025$, pues de esta manera se cumple (5) debido a la relación (4). Entonces:
\begin{align*}
    a &:= F_Z^{-1}(0.025) = -1.9599639845400545,\\
    b &:= F_Z^{-1}(0.975) = 1.9599639845400545.
\end{align*}

Observemos así que 
\begin{align*}
    0.95 &\approx \P\left[a < \frac{\overline{X_m}-\lambda}{\sqrt{\overline{X_m}}/\sqrt{m}} < b \right].\\
    &= \P\left[  a\cdot \frac{\sqrt{\overline{X_m}}}{\sqrt{m}} < \overline{X_m} - \lambda < b\cdot \frac{\sqrt{\overline{X_m}}}{\sqrt{m}}  \right].\\
    &= \P\left[  a\cdot \frac{\sqrt{\overline{X_m}}}{\sqrt{m}} - \overline{X_m}<  - \lambda < b\cdot \frac{\sqrt{\overline{X_m}}}{\sqrt{m}} - \overline{X_m} \right].\\
    &= \P\left[ \overline{X_m} -b\cdot \frac{\sqrt{\overline{X_m}}}{\sqrt{m}}  <  \lambda < \overline{X_m}-a\cdot \frac{\sqrt{\overline{X_m}}}{\sqrt{m}} \right].
\end{align*} 
Dónde $m=718$. Se calcula $\overline{X_m}$ con la muestra dada. Así se obtiene que:
\begin{align*}
    t_0 := \overline{X_m} -b\cdot \frac{\sqrt{\overline{X_m}}}{\sqrt{m}} &=  8.984193859877442.\\
    t_1 := \overline{X_m}-a\cdot \frac{\sqrt{\overline{X_m}}}{\sqrt{m}} &=  9.42806240753203.
\end{align*}
Entonces,
\[
    \P[t_0 < \lambda < t_1] \approx 0.95.
\]

Por lo tanto, se concluye con un $95\%$ de confinaza que el promedio de fallas del medidor en un día es mayor a $t_0$ y menor a $t_1$. Entonces se puede decir con $95\%$ de confinaza que el promedio de fallas del medidor en un día es mayor a $5$ y menor a $10$.

\textbf{Nota:} Los cálculos para obtener el valor de $a, b, t_0$ y $t_0$ fueron hechos en python. Se incluye el código. Para que el código funcioné se debe de borrar la primera línea del archivo ``tarea4e3.txt'' y se debe ejecutar en la misma carpeta dónde se encuntra el programa.

$\hfill\diamondsuit$
\end{document}