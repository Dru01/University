% Preámbulo
\documentclass[letterpaper]{article}
\usepackage[utf8]{inputenc}
\usepackage[spanish]{babel}

\usepackage{enumitem}
\usepackage{titling}

% Símbolos
	\usepackage{amsmath}
	\usepackage{amssymb}
	\usepackage{amsthm}
	\usepackage{amsfonts}
	\usepackage{mathtools}
	\usepackage{bbm}
	\usepackage[thinc]{esdiff}
	\allowdisplaybreaks

% Márgenes
	\usepackage
	[
		margin = 1.1in
	]
	{geometry}

% Imágenes
	\usepackage{float}
	\usepackage{graphicx}
	\graphicspath{{imagenes/}}
	\usepackage{subcaption}

% Ambientes
	\usepackage{amsthm}

	\theoremstyle{definition}
	\newtheorem{ejercicio}{Ejercicio}

	\newtheoremstyle{lemathm}{4pt}{0pt}{\itshape}{0pt}{\bfseries}{ --}{ }{\thmname{#1}\thmnumber{ #2}\thmnote{ (#3)}}
	\theoremstyle{lemathm}
	\newtheorem{lema}{Lema}

	\newtheoremstyle{lemademthm}{0pt}{10pt}{\itshape}{ }{\mdseries}{ --}{ }{\thmname{#1}\thmnumber{ #2}\thmnote{ (#3)}}
	\theoremstyle{lemademthm}
	\newtheorem*{lemadem}{Demostración}

% Macros
	\newcommand{\sumi}[2]{\sum_{i=#1}^{#2}}
	\newcommand{\dint}[2]{\displaystyle\int_{#1}^{#2}}
	\newcommand{\inte}[2]{\int_{#1}^{#2}}
	\newcommand{\dlim}{\displaystyle\lim}
	\newcommand{\limxinf}{\lim_{x\to\infty}}
	\newcommand{\limninf}{\lim_{n\to\infty}}
	\newcommand{\dlimninf}{\displaystyle\lim_{n\to\infty}}
	\newcommand{\limh}{\lim_{h\to0}}
	\newcommand{\ddx}{\dfrac{d}{dx}}
	\newcommand{\txty}{\text{ y }}
	\newcommand{\txto}{\text{ o }}
	\newcommand{\Txty}{\quad\text{y}\quad}
	\newcommand{\Txto}{\quad\text{o}\quad}
	\newcommand{\si}{\text{si}\quad}

	\newcommand{\etiqueta}{\stepcounter{equation}\tag{\theequation}}
	\newcommand{\tq}{:}
	\renewcommand{\o}{\circ}
	\newcommand*{\QES}{\hfill\ensuremath{\blacksquare}}
	\newcommand*{\qes}{\hfill\ensuremath{\square}}
	\newcommand*{\QESHERE}{\tag*{$\blacksquare$}}
	\newcommand*{\qeshere}{\tag*{$\square$}}
	\newcommand*{\QED}{\hfill\ensuremath{\blacksquare}}
	\newcommand*{\QEDHERE}{\tag*{$\blacksquare$}}
	\newcommand*{\qel}{\hfill\ensuremath{\boxdot}}
	\newcommand*{\qelhere}{\tag*{$\boxdot$}}
	\renewcommand*{\qedhere}{\tag*{$\square$}}

	\newcommand{\suc}[1]{\left(#1_n\right)_{n\in\N}}
	\newcommand{\en}[2]{\binom{#1}{#2}}
	\newcommand{\upsum}[2]{U(#1,#2)}
	\newcommand{\lowsum}[2]{L(#1,#2)}
	\newcommand{\abs}[1]{\left| #1 \right| }
	\newcommand{\bars}[1]{\left \| #1 \right \| }
	\newcommand{\pars}[1]{\left( #1 \right) }
	\newcommand{\bracs}[1]{\left[ #1 \right] }
	\newcommand{\floor}[1]{\left \lfloor #1 \right\rfloor }
	\newcommand{\ceil}[1]{\left \lceil #1 \right\rceil }
	\newcommand{\angles}[1]{\left \langle #1 \right\rangle }
	\newcommand{\set}[1]{\left \{ #1 \right\} }
	\newcommand{\norma}[2]{\left\| #1 \right\|_{#2} }

	\newcommand{\NN}{\mathbb{N}}
	\newcommand{\QQ}{\mathbb{Q}}
	\newcommand{\RR}{\mathbb{R}}
	\newcommand{\ZZ}{\mathbb{Z}}
	\newcommand{\PP}{\mathbb{P}}
	\newcommand{\EE}{\mathbb{E}}
	\newcommand{\1}{\mathbbm{1}}
	\newcommand{\eps}{\varepsilon}
	\newcommand{\ttF}{\mathtt{F}}
	\newcommand{\bfF}{\mathbf{F}}

	\newcommand{\To}{\longrightarrow}
	\newcommand{\mTo}{\longmapsto}
	\newcommand{\ssi}{\Longleftrightarrow}
	\newcommand{\sii}{\Leftrightarrow}
	\newcommand{\then}{\Rightarrow}

	\newcommand{\pTFC}{{\itshape 1er TFC\/}}
    \newcommand{\sTFC}{{\itshape 2do TFC\/}}
    
% Datos
	\title{Probabilidad \\Tarea III - Ejercicio 3}
    \author{Rubén Pérez Palacios\\Ricardo Alberto Gloria Picazzo\\Mercé Nachón Moreno\\ Josue Emmanuel Ornelas Hernández\\Profesor: Dr. Ehyter Matías Martín González}
    \date{\today}

% DOCUMENTO
\begin{document}
	\maketitle
    
    \section*{Problemas}

    \begin{enumerate}
			
		\item (Rubén Pérez Palacios) Sea $F$ una distribución y sea $c \in (0,1)$. Demuestre que
		
		\[G(x) = (1-c)\sum_{n=0}^{\infty} c^nF^{*n}(x), x \in \RR,\]

		es una función de distribución y calcule su función característica.

		Sean $X_1,\cdots,X_n$ variables aleatorias con distirbución $F$ independientes luego entonces

		\[F_{X_1+\cdots+X_N}(x) = F^{*n}(x).\]

		Puesto que $F^{*n}$ es una función de distribución entonces

		\[F^{*n}(x) \leq 1, \forall x \in \RR,\]

		por lo tanto

		\[G(x) = (1-c)\sum_{n=0}^{\infty} c^nF^{*n}(x) \leq (1-c)\sum_{n=0}^{\infty} c^n, x \in \RR,\]

		y al ser $0 < c < 1$ tenemos
		
		\[\sum_{n=0}^{\infty} c^n = \frac{1}{1-c},\]
		
		por lo tanto
		
		\[\sum_{n=0}^{\infty} c^nF^{*n}(x)\]
		
		converge.

		Ahora veremos que se cumplen las tres propiedades de una distribución para $G$.

		\begin{itemize}
			\item \textbf{Monótona no decreciente}
			
			Sea $x < y$ tales que son $F$ medibles. Por ser $F^{*n}$ por ser una distirbución entonces se cumple

			\[F^{*n}(x) \leq F^{*n}(y),\]

			y al ser $0 < c < 1$ tenemos

			\[c^n F^{*n}(x) \leq c^n F^{*n}(y),\]

			al converger $\frac{G(x)}{(1-c)}$ concluimos
			
			\[G(x) = (1-c)\sum_{n=0}^{\infty} c^nF^{*n}(x) \leq (1-c)\sum_{n=0}^{\infty} c^nF^{*n}(y) = G(y).\]

			\item \textbf{Continua por la derecha}
			
			Al ser $F^{*n}$ una distribucion entonces es continua por la derecha es decir para todo $a \in \RR$ existe el

			\[\lim_{x\to a^+} F^{*n}(x),\]

			por linealidad del operador límite, existe el

			\[\lim_{x\to a^+} c^nF^{*n}(x),\]

			como $G(x)/(1-c)$ converge y como la sucesión $c^nF^{*n}$ es dominada entonces existe el

			\[\lim_{x\to a^+} \sum_{n=0}^{\infty} c^nF^{*n}(x),\]

			por linealidad del operador límite, concluimos que existe el

			\[\lim_{x\to a^+} (1-c)\sum_{n=0}^{\infty} c^nF^{*n}(x) = \lim_{x\to a^+} G(x).\]

			\item \textbf{límites infinitos}
			
			Como habiamos visto nuestra sucesión $c^nF^{*n}(x)$ esta dominada por $c^n$ cuya serie convergía, luego por convergencia dominada tenemos que

			\[\limxinf \sum_{n=0}^{\infty} c^nF^{*n}(x) = \sum_{n=0}^{\infty} \limxinf c^nF^{*n}(x),\]

			por linealidad del operador límite y por continuidad de $P$ tenemos que

			\[\limxinf \sum_{n=0}^{\infty} c^nF^{*n}(x) = \sum_{n=0}^{\infty}  c^nF^{*n}(\limxinf x),\]

			analogamente obtenemos

			\[\lim_{x\to-\infty} \sum_{n=0}^{\infty} c^nF^{*n}(x) = \sum_{n=0}^{\infty}  c^nF^{*n}(\lim_{x\to-\infty} x),\]

			Luego por ser $F^{*n}(x)$ una función de distribución concluimos

			\[\limxinf (1-c)\sum_{n=0}^{\infty} c^nF^{*n}(x) = (1-c)\sum_{n=0}^{\infty}  c^n = 1,\]

			y

			\[\lim_{x\to-\infty} (1-c)\sum_{n=0}^{\infty} c^nF^{*n}(x) = (1-c)\sum_{n=0}^{\infty}  c^n\cdot0 = 0.\]
			
		\end{itemize}

		\newpage

		\item (Rubén Pérez Palacios) El siguiente ejercicio demuestra que la convergencia en distribucion \textbf{no} implica la convergencia de las
		funciones de densidad, cuando estas existen. Sean $\set{X_n}$ v.a. con funciones de distribución

		\[F_n(x) = \pars{x - \frac{\sin\pars{2\pi nx}}{2\pi n}} \1{(0,1)}(x) + \1_{[1,\infty)]}(x).\]

		Por conveniencia expresaremos a $F_n(x)$ como 

		\[F_n(x) = \begin{cases}
			0 & \text{si } x \leq 0\\
			\pars{x - \frac{\sin\pars{2\pi nx}}{2\pi n}} & \text{si } 0 < x < 1\\
			1 & \text{si } 1 \leq x\\
		\end{cases}.\]

		Luego veamos que pasa cuando $0 < x < 1$. Por linealidad de la derivada tenemos que

		\[\diff{F_n(x)}{x} = \diff{x}{x} - \diff{\sin\pars{2\pi nx}}{2\pi n} = 1 - \cos\pars{2\pi nx} = f_n(x).\]

		Ahora veamos que $F_n(x)$ converge a una distribución uniforme

		\[\limninf F_n(x) = \limninf \pars{x - \frac{\sin\pars{2\pi nx}}{2\pi n}},\]

		por linealidad del operador límite tenemos

		\[\limninf F_n(x) = x - \limninf \frac{\sin\pars{2\pi nx}}{2\pi n},\]

		puesto que $|sin(2\pi nx)| \leq 1$ entonces tenemos que

		\[\limninf F_n(x) = x - 0 = x.\]

		Ahora fijemonos en $cos(2\pi n x)$, pero esta no converge a la función de densidad de una distribución $\mathcal{U}\bracs{0,1}$, ya que cuando $x = \frac{p}{q}$ es un racional no existe $\limninf cos(2\pi nx)$,  esto puesto que si nos agarramos la subseción $n_k = k*q$ entonces esta subseción converge a $cos(2\pi n_kx) = cos(2\pi kp) = cos(2\pi) = 0$, pero si nos agarramos la subseción $n_l = ql + 1$ entonces esta converge a $cos(2\pi n_lx) = cos(2\pi (ql + 1)p/q) = cos(2\pi lp + 2\pi p/q) = cos(2\pi p/q) \neq 0$ por lo tanto $\limninf cos(2\pi nx)$ no existe. Es decir

		\[\limninf f_n(x), \forall x \in \bracs{0,1}\cap\QQ,\]

		por lo tanto

		\[\limninf f_n(x) \neq 1, x\in[0,1].\]


    \end{enumerate}

	\end{document}
			