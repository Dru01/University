\documentclass[10pt]{extarticle}
\usepackage{enumerate}
\usepackage{amsmath}
\usepackage{amsfonts}
\usepackage{amssymb,times}
\usepackage{graphicx}
\usepackage{latexsym}
\usepackage{color}
\usepackage{pifont}
\usepackage{multicol}
\usepackage{appendix}
%\usepackage{dsfont}
\usepackage{mathrsfs}
%\spanishdecimal{.}
\usepackage[sort&compress]{natbib}
\bibliographystyle{plainnat}
%\usepackage{suthesis-2e}
%\usepackage{verbatim}
%\spanishdecimal{.}
\def\theequation{\thesection.\arabic{equation}}

\textwidth=6.5in
%\lineskip .25cm
%\lineskiplimit .25cm
\textheight=9.5in
\topmargin=-.75in
\topskip=-4pt
\evensidemargin=-2pt
\oddsidemargin=-1pt
%%%%%%%%%%%%%%%%%%%%
%\oddsidemargin 0in \textwidth 6.75in \topmargin 0in \textheight
%8.5in
\parindent 0em
\parskip 1ex
\newtheorem{lemma}{Lemma}
\newtheorem{remark}{Remark}
\newtheorem{propo}{Proposition}
\newtheorem{teo}{Theorem}
\newtheorem{coro}{Corollary}
\newtheorem{ejer}{Ejercicio}
\newtheorem{defi}{Definition}
\newtheorem{notac}{Notation}
\newtheorem{hypo}{Hypothesis}
%\newenvironment{proof}[1][Proof]{\textbf{#1.} }{\ \rule{0.5em}{0.5em}}

\def\CC{\mathbb{C}}
\newcommand{\rr}{\mathbb{R}}
\newcommand{\pai}{\left(}
\newcommand{\pad}{\right)}
\newcommand{\ci}{\left[}
\newcommand{\cd}{\right]}
\newcommand{\nn}{\mathbb{N}}
\newcommand{\B}{\mathcal{B}}
\newcommand{\p}{\mathbb{P}}
\newcommand{\E}{\mathbb{E}}
\newcommand{\lphi}{\widehat{\phi}}
\newcommand{\D}{\mathfrak{D}}
%\renewcommand{\qedsymbol}{\rule{1ex}{1ex}}

\begin{document}
 \begin{flushright}
  \begin{tabular}{|c|c|c|c|c|c|c|c|c|c|c|c|c|c|c|c|}
  \hline
  \ \ $E_1$\ \ &\ \ $E_2$\ \ &\ \ $E_3$\ \ &\ \ $E_4$\ \ &\ \ $E_5$\ \ &Calif.\\
  \hline
  & &  & & & \\
    & &  & & &\\
   \hline
 \end{tabular}\\
 \end{flushright}
\begin{center}
\large
Probabilidad\\
Rubén Pérez Palacios Lic. Computación Matemática NUA 424730
\normalsize

Agosto - Diciembre de 2020\\

Parcial 2 (13 de noviembre)
\end{center}

Resuelva \textbf{formal y detalladamente} los siguientes ejercicios, de manera totalmente individual.

Está estrictamente prohibido consultar cualquier otro tipo de bibliografía distinto a las notas del curso. Ante cualquier sospecha de esto, el examen se anulará (se calificará automáticamente con cero) sin derecho a réplica.

Coloque una y solamente una respuesta a cada ejercicio. En caso de colocar más de una, solamente se revisará la primera.

La nota de este examen será el promedio de las puntuaciones obtenidas en cada ejercicio.

\begin{enumerate}
\item (100 pts.) Sea $Z\sim N(0,1)$ y $X\sim\Gamma( \nu/2,1/2)$ con $\nu>0$ no precisamente entero. Utilice el Teorema de Cambio de Variable Multivariado para demostrar que $T=\frac{Z}{\sqrt{X/\nu}}$ tiene distribución $t_\nu$, bajo la hipótesis $Z\perp X$.

\item (100 pts.) Sean $X\sim\Gamma( \nu/2,1/2)$ y $Y\sim\Gamma( \mu/2,1/2)$ con $\nu,\mu>0$ no precisamente enteros. Utilizando Probabilidad Total halle la distribución de $F=\frac{\mu X}{\nu Y}$, bajo la hipótesis $X\perp Y$.

\item Una sucesión de vectores aleatorios $\{\vec{X}_n\}$ converge en distribución a otro vector aleatorio $\vec{X}$ ssi $F_{\vec{X}_n}(\vec{x})$ converge a $F_{\vec{X}}(\vec{x})$ para todo $\vec{x}$ en el que $F_{\vec{X}}$ es continua (según la distancia euclidiana).
\begin{enumerate}
\item (70 pts.) Demuestre que si $\{\vec{X}_n\}$ es una sucesión de vectores aleatorios tales que convergen en distribución a $\vec{X}$, entonces cada entrada de $\{\vec{X}_n\}$ converge a la correspondiente entrada de $\vec{X}$. ¿Se cumple el recíproco?

\item (30 pts.) Sea $\vec{X}\sim N_d(\mu \vec{1},\sigma^2 I_d)$. Halle la distribución de $\overline{X}_d$ condicionada a $\max\{X_1,\dots,X_d\}-\min\{X_1,\dots,X_d\}$.
\end{enumerate}

\item (100 pts.) Sean $\{X_n\}$ variables aleatorias iid con media $\mu$ y varianza finita $\sigma^2$. Demuestre que 
$$\sqrt{n}\pai e^{\overline{X}_n}-e^\mu\pad\overset{d}{\to} \sigma e^\mu Z,\quad Z\sim N(0,1).$$
\item (100 pts.) Sea $\{X_n\}$ una sucesión de variables aleatorias iid cuya función de distribución tiene extremo derecho infinito. Para $x>0$ fijo, sea

$$T(x):=\inf\{n\in\nn:X_n>x\}.$$

$T(x)$ es el índice de la primera variable de la sucesión que toma un valor mayor a $x$. Sea $X$ otra variable aleatoria con la misma distribución que las $X_n$ e independiente de todas las $X_n$ y sea $\{Y_n\}$ v.a. iid con distribución $Bernoulli(\p\ci X>x\cd)$. Sea $m\in\nn$ arbitrario, demuestre que 
$$ \frac{1}{mx}T(x)\sum\limits_{j=1}^{\lceil mx\rceil}Y_j\overset{d}{\to} E,x\to\infty,\quad E\sim exp(1).$$

\item (10 pts. extra en la nota final del examen). Sea $A\in\mathcal{B}(\rr)$ y definamos
$$\partial A:=\{x:\exists \{y_n\}\subseteq A, y_n\to x\wedge \exists \{z_n\}\subseteq A^c, z_n\to x\}.$$

$\partial A$ se conoce como la \textbf{frontera de $A$}. Sea $\{X_n\}$ una sucesión de v.a. con funciones de distribución $\{F_n\}$ y sea $X$ otra v.a. con función de distribución $F$. Demuestre que $X_n\overset{d}{\to}X$ssi 
$$\int_A F_n(dx)\to \int_A F(dx), n\to\infty,$$
para todo $A$ tal que $F(\partial A)=0$. ($F(A):=\p\ci X\in A\cd$).
\end{enumerate}

\end{document}