% Preámbulo
\documentclass[letterpaper]{article}
\usepackage[utf8]{inputenc}
\usepackage[spanish]{babel}

\usepackage{enumitem}
\usepackage{titling}

% Símbolos
	\usepackage{amsmath}
	\usepackage{amssymb}
	\usepackage{amsthm}
	\usepackage{amsfonts}
	\usepackage{mathtools}
	\usepackage{bbm}
	\usepackage[thinc]{esdiff}
	\allowdisplaybreaks

% Márgenes
	\usepackage
	[
		margin = 1.2in
	]
	{geometry}

% Imágenes
	\usepackage{float}
	\usepackage{graphicx}
	\graphicspath{{imagenes/}}
	\usepackage{subcaption}

% Ambientes
	\usepackage{amsthm}

	\theoremstyle{definition}
	\newtheorem{ejercicio}{Ejercicio}

	\newtheoremstyle{lemathm}{4pt}{0pt}{\itshape}{0pt}{\bfseries}{ --}{ }{\thmname{#1}\thmnumber{ #2}\thmnote{ (#3)}}
	\theoremstyle{lemathm}
	\newtheorem{lema}{Lema}
	
	\newtheoremstyle{lemathm}{4pt}{0pt}{\itshape}{0pt}{\bfseries}{ --}{ }{\thmname{#1}\thmnumber{ #2}\thmnote{ (#3)}}
	\theoremstyle{lemathm}
	\newtheorem{theo}{Teorema}

	\newtheoremstyle{lemademthm}{0pt}{10pt}{\itshape}{ }{\mdseries}{ --}{ }{\thmname{#1}\thmnumber{ #2}\thmnote{ (#3)}}
	\theoremstyle{lemademthm}
	\newtheorem*{lemadem}{Demostración}

% Macros
	\newcommand{\sumi}[2]{\sum_{i=#1}^{#2}}
	\newcommand{\dint}[2]{\displaystyle\int_{#1}^{#2}}
	\newcommand{\inte}[2]{\int_{#1}^{#2}}
	\newcommand{\dlim}{\displaystyle\lim}
	\newcommand{\limxinf}{\lim_{x\to\infty}}
	\newcommand{\limninf}{\lim_{n\to\infty}}
	\newcommand{\dlimninf}{\displaystyle\lim_{n\to\infty}}
	\newcommand{\limh}{\lim_{h\to0}}
	\newcommand{\ddx}{\dfrac{d}{dx}}
	\newcommand{\txty}{\text{ y }}
	\newcommand{\txto}{\text{ o }}
	\newcommand{\Txty}{\quad\text{y}\quad}
	\newcommand{\Txto}{\quad\text{o}\quad}
	\newcommand{\si}{\text{si}\quad}

	\newcommand{\etiqueta}{\stepcounter{equation}\tag{\theequation}}
	\newcommand{\tq}{:}
	\renewcommand{\o}{\circ}
	\newcommand*{\QES}{\hfill\ensuremath{\blacksquare}}
	\newcommand*{\qes}{\hfill\ensuremath{\square}}
	\newcommand*{\QESHERE}{\tag*{$\blacksquare$}}
	\newcommand*{\qeshere}{\tag*{$\square$}}
	\newcommand*{\QED}{\hfill\ensuremath{\blacksquare}}
	\newcommand*{\QEDHERE}{\tag*{$\blacksquare$}}
	\newcommand*{\qel}{\hfill\ensuremath{\boxdot}}
	\newcommand*{\qelhere}{\tag*{$\boxdot$}}
	\renewcommand*{\qedhere}{\tag*{$\square$}}

	\newcommand{\suc}[1]{\left(#1_n\right)_{n\in\N}}
	\newcommand{\en}[2]{\binom{#1}{#2}}
	\newcommand{\upsum}[2]{U(#1,#2)}
	\newcommand{\lowsum}[2]{L(#1,#2)}
	\newcommand{\abs}[1]{\left| #1 \right| }
	\newcommand{\bars}[1]{\left \| #1 \right \| }
	\newcommand{\pars}[1]{\left( #1 \right) }
	\newcommand{\bracs}[1]{\left[ #1 \right] }
	\newcommand{\floor}[1]{\left \lfloor #1 \right\rfloor }
	\newcommand{\ceil}[1]{\left \lceil #1 \right\rceil }
	\newcommand{\angles}[1]{\left \langle #1 \right\rangle }
	\newcommand{\set}[1]{\left \{ #1 \right\} }
	\newcommand{\norma}[2]{\left\| #1 \right\|_{#2} }


	\newcommand{\N}{\mathbb{N}}
	\newcommand{\Q}{\mathbb{Q}}
	\newcommand{\R}{\mathbb{R}}
	\newcommand{\Z}{\mathbb{Z}}
	\newcommand{\PP}{\mathbb{P}}
	\newcommand{\1}{\mathbbm{1}}
	\newcommand{\eps}{\varepsilon}
	\newcommand{\ttF}{\mathtt{F}}
	\newcommand{\bfF}{\mathbf{F}}

	\newcommand{\To}{\longrightarrow}
	\newcommand{\mTo}{\longmapsto}
	\newcommand{\ssi}{\Longleftrightarrow}
	\newcommand{\sii}{\Leftrightarrow}
	\newcommand{\then}{\Rightarrow}

	\newcommand{\pTFC}{{\itshape 1er TFC\/}}
    \newcommand{\sTFC}{{\itshape 2do TFC\/}}
    
% Datos
    \title{Probabilidad \\Parcial II - Ejercicio 4}
    \author{Rubén Pérez Palacios Lic. Computación Matemática\\Profesor: Dr. Ehyter Matías Martín González}
    \date{\today}

% DOCUMENTO
\begin{document}
	\maketitle
    
    \section*{Problemas}

    \begin{enumerate}
        
		\item (100 pts.) Sean $\{X_n\}$ variables aleatorias iid con media $\mu$ y varianza finita $\sigma^2$. Demuestre que 
		
		\[\sqrt{n} e^{\overline{X}_n}-e^\mu \overset{d}{\to} \sigma e^\mu Z,\quad Z\sim N(0,1).\]

		\begin{theo}
			
			(Metodo Delta) Sean $\{X_n\}$ variables aleatorias iid con media $\mu$ y varianza finita $\sigma^2$, y $g$ una función derivable cuya derivado no se anula, entonces
			
			\[\sqrt{n} \pars{\frac{g\pars{\overline{X}_n} - g\pars{\mu}}{\sigma g'(\mu)}} \xrightarrow{d} Z, Z \sim N(0,1).\]
		
		\end{theo}
		
		\begin{proof}
			Por el teorema de limite central tenemos que

			\[\sqrt{n} \pars{\frac{\overline{X}_n - \mu}{\sigma}} \xrightarrow{d} Z, Z \sim N(0,1).\]

			Por el teorema del Bebe de Skorohod tenemos que existen $Y_n'$ y $Z'$ tales que

			\[Y_n' \sim \sqrt{n} \pars{\frac{\overline{X}_n - \mu}{\sigma}}, \quad Z' \sim Z,\]

			y que

			\[\limninf Y_n' = Z'.\]
			
			Entonces

			\[\overline{X}_n \sim \mu + \frac{\sigma Y_n'}{\sqrt{n}},\]

			por lo que

			\[g\pars{\overline{X}_n} \sim g\pars{\mu + \frac{\sigma Y_n'}{\sqrt{n}}}\]

			\[\sqrt{n} \pars{\frac{g\pars{\overline{X}_n} - g\pars{\mu}}{\sigma g'(\mu)}} \sim \sqrt{n} \pars{\frac{g\pars{\mu + \frac{\sigma Y_n'}{\sqrt{n}}} - g\pars{\mu}}{\sigma g'(\mu)}},\]

			multiplicando por $1 = \frac{Y_n'}{Y_n'}$ obtenemos

			\[\sqrt{n} \pars{\frac{g\pars{\overline{X}_n} - g\pars{\mu}}{\sigma g'(\mu)}} \sim \pars{\frac{g\pars{\mu + \frac{\sigma Y_n'}{\sqrt{n}}} - g\pars{\mu}}{\frac{\sigma Y_n'}{\sqrt{n}}}} \pars{\frac{Y_n'}{g'(\mu)}}.\]

			Por definición de $g'(\mu)$ y que $\frac{\sigma Y_n'}{\sqrt{n}} \xrightarrow{c.s.} 0$ tenemos que

			\[g'(\mu) = \limninf \pars{\frac{g\pars{\mu + \frac{\sigma Y_n'}{\sqrt{n}}} - g\pars{\mu}}{\frac{\sigma Y_n'}{\sqrt{n}}}},\]

			además como

			\[\limninf Y_n' = Z',\]

			concluimos que

			\[\sqrt{n} \pars{\frac{g\pars{\overline{X}_n} - g\pars{\mu}}{\sigma g'(\mu)}} \xrightarrow{d} g'(\mu)\frac{Z'}{g'(\mu)} = Z' \sim Z \sim N(0,1).\]

		\end{proof}

		Ahora seguiremos por demostrar el ejercicio

		\begin{proof}
			Un corolario del teorema anterior es que

			\[\sqrt{n} \pars{g\pars{\overline{X}_n} - g\pars{\mu}} \xrightarrow{d} \sigma g'(\mu) Z, Z \sim N(0,1),\]

			esto por Slutsky.

			Tomando $g(x) = e^x$ obtenemos

			\[\sqrt{n} \pars{e^{\overline{X}_n} - e^{\mu}} \xrightarrow{d} \sigma e^\mu Z, Z \sim N(0,1)\]
		\end{proof}

    \end{enumerate}

	\end{document}
			