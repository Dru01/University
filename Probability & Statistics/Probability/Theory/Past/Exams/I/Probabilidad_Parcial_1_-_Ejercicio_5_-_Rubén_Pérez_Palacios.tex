% Preámbulo
\documentclass[letterpaper]{article}
\usepackage[utf8]{inputenc}
\usepackage[spanish]{babel}

\usepackage{enumitem}
\usepackage{titling}

% Símbolos
	\usepackage{amsmath}
	\usepackage{amssymb}
	\usepackage{amsthm}
	\usepackage{amsfonts}
	\usepackage{mathtools}
	\usepackage{bbm}
	\usepackage[thinc]{esdiff}
	\allowdisplaybreaks

% Márgenes
	\usepackage
	[
		margin = 1.2in
	]
	{geometry}

% Imágenes
	\usepackage{float}
	\usepackage{graphicx}
	\graphicspath{{imagenes/}}
	\usepackage{subcaption}

% Macros
	\newcommand{\sumi}[2]{\sum_{i=#1}^{#2}}
	\newcommand{\dint}[2]{\displaystyle\int_{#1}^{#2}}
	\newcommand{\inte}[2]{\int_{#1}^{#2}}
	\newcommand{\dlim}{\displaystyle\lim}
	\newcommand{\limxinf}{\lim_{x\to\infty}}
	\newcommand{\limninf}{\lim_{n\to\infty}}
	\newcommand{\dlimninf}{\displaystyle\lim_{n\to\infty}}
	\newcommand{\limh}{\lim_{h\to0}}
	\newcommand{\ddx}{\dfrac{d}{dx}}
	\newcommand{\txty}{\text{ y }}
	\newcommand{\txto}{\text{ o }}
	\newcommand{\Txty}{\quad\text{y}\quad}
	\newcommand{\Txto}{\quad\text{o}\quad}
	\newcommand{\si}{\text{si}\quad}

	\newcommand{\etiqueta}{\stepcounter{equation}\tag{\theequation}}
	\newcommand{\tq}{:}
	\renewcommand{\o}{\circ}
	% \newcommand*{\QES}{\hfill\ensuremath{\boxplus}}
	% \newcommand*{\qes}{\hfill\ensuremath{\boxminus}}
	% \newcommand*{\qeshere}{\tag*{$\boxminus$}}
	% \newcommand*{\QESHERE}{\tag*{$\boxplus$}}
	\newcommand*{\QES}{\hfill\ensuremath{\blacksquare}}
	\newcommand*{\qes}{\hfill\ensuremath{\square}}
	\newcommand*{\QESHERE}{\tag*{$\blacksquare$}}
	\newcommand*{\qeshere}{\tag*{$\square$}}
	\newcommand*{\QED}{\hfill\ensuremath{\blacksquare}}
	\newcommand*{\QEDHERE}{\tag*{$\blacksquare$}}
	\newcommand*{\qel}{\hfill\ensuremath{\boxdot}}
	\newcommand*{\qelhere}{\tag*{$\boxdot$}}
	\renewcommand*{\qedhere}{\tag*{$\square$}}

	\newcommand{\suc}[1]{\left(#1_n\right)_{n\in\N}}
	\newcommand{\en}[2]{\binom{#1}{#2}}
	\newcommand{\upsum}[2]{U(#1,#2)}
	\newcommand{\lowsum}[2]{L(#1,#2)}
	\newcommand{\abs}[1]{\left| #1 \right| }
	\newcommand{\bars}[1]{\left \| #1 \right \| }
	\newcommand{\pars}[1]{\left( #1 \right) }
	\newcommand{\bracs}[1]{\left[ #1 \right] }
	\newcommand{\floor}[1]{\left \lfloor #1 \right\rfloor }
	\newcommand{\ceil}[1]{\left \lceil #1 \right\rceil }
	\newcommand{\angles}[1]{\left \langle #1 \right\rangle }
	\newcommand{\set}[1]{\left \{ #1 \right\} }
	\newcommand{\norma}[2]{\left\| #1 \right\|_{#2} }


	\newcommand{\N}{\mathbb{N}}
	\newcommand{\Q}{\mathbb{Q}}
	\newcommand{\R}{\mathbb{R}}
	\newcommand{\Z}{\mathbb{Z}}
	\newcommand{\PP}{\mathbb{P}}
	\newcommand{\1}{\mathbbm{1}}
	\newcommand{\eps}{\varepsilon}
	\newcommand{\ttF}{\mathtt{F}}
	\newcommand{\bfF}{\mathbf{F}}

	\newcommand{\To}{\longrightarrow}
	\newcommand{\mTo}{\longmapsto}
	\newcommand{\ssi}{\Longleftrightarrow}
	\newcommand{\sii}{\Leftrightarrow}
	\newcommand{\then}{\Rightarrow}

	\newcommand{\pTFC}{{\itshape 1er TFC\/}}
    \newcommand{\sTFC}{{\itshape 2do TFC\/}}
    
% Datos
    \title{Probabilidad \\Parcial I - Ejercicio 5}
    \author{Rubén Pérez Palacios Lic. Computación Matemática\\Profesor: Dr. Ehyter Matías Martín González}
    \date{26 de Septiembre 2020}

% DOCUMENTO
\begin{document}
	\maketitle
    
    \section*{Problemas}

    \begin{enumerate}

		\item Sea $m_n$ el mínimo de $n$ variables aleatorias iid con distribución común $exp(\theta)$, todas sobre el mismo espacio de probabilidad. Demuestre que $m_n\overset{L_p}{\to}0$ para todo $p>0$.
		
		\begin{proof}
			Empecemos por ver cual es la función de distirbución de $m_n$, para ello digamos que nuestras $n$ variables aleatorias son $X_1,\cdots,X_n$, ahora

			\begin{align*}
				F_{m_n}(x) &= \mathbb{P}\bracs{m_n \leq x}\\
				&= 1 - \mathbb{P}\bracs{m_n > x}\\
				&= 1 - \mathbb{P}\bracs{X_1 > x, \cdots, X_n > x}\\
				&= 1 - \prod_{i=1}^n \mathbb{P}\bracs{X_i > x}\\
				&= 1 - \prod_{i=1}^n \mathbb{P}\bracs{X_1 > x}\\
				&= 1 - \mathbb{P}\bracs{X_1 > x}^n\\
				&= 1 - \pars{e^{-\theta x}}^n\\
				&= 1 - e^{-\theta n x}\\
			\end{align*}

			por lo que

			\[f_{m_n}(x) = \theta n e^{-\theta n x},\]

			Ahora veamos que

			\begin{align*}
				\mathbb{E}\bracs{|m_n|^p} = \mathbb{E}\bracs{m_n^p}
				&= \int_{0}^{\infty} x^p f_{m_n}(x) dx\\
				&= \int_{0}^{\infty} x^p \theta n e^{-\theta n x} dx\\
				&= \theta n \int_{0}^{\infty} x^p e^{-\theta n x} dx\\
			\end{align*}

			En clase se vio como esto se completaba a una gama no me salieron los calculo pero después de ver eso evaluas los límites y de ser cierto deberían converger a 0

		\end{proof}

    \end{enumerate}

	\end{document}
			