\documentclass[10pt]{extarticle}
\usepackage{enumerate}
\usepackage{amsmath}
\usepackage{amsfonts}
\usepackage{amssymb,times}
\usepackage{graphicx}
\usepackage{latexsym}
\usepackage{color}
\usepackage{pifont}
\usepackage{multicol}
\usepackage{appendix}
%\usepackage{dsfont}
\usepackage{mathrsfs}
%\spanishdecimal{.}
\usepackage[sort&compress]{natbib}
\bibliographystyle{plainnat}
%\usepackage{suthesis-2e}
%\usepackage{verbatim}
%\spanishdecimal{.}
\def\theequation{\thesection.\arabic{equation}}

\textwidth=6.5in
%\lineskip .25cm
%\lineskiplimit .25cm
\textheight=9.5in
\topmargin=-.75in
\topskip=-4pt
\evensidemargin=-2pt
\oddsidemargin=-1pt
%%%%%%%%%%%%%%%%%%%%
%\oddsidemargin 0in \textwidth 6.75in \topmargin 0in \textheight
%8.5in
\parindent 0em
\parskip 1ex
\newtheorem{lemma}{Lemma}
\newtheorem{remark}{Remark}
\newtheorem{propo}{Proposition}
\newtheorem{teo}{Theorem}
\newtheorem{coro}{Corollary}
\newtheorem{ejer}{Ejercicio}
\newtheorem{defi}{Definition}
\newtheorem{notac}{Notation}
\newtheorem{hypo}{Hypothesis}
%\newenvironment{proof}[1][Proof]{\textbf{#1.} }{\ \rule{0.5em}{0.5em}}

\def\CC{\mathbb{C}}
\newcommand{\rr}{\mathbb{R}}
\newcommand{\pai}{\left(}
\newcommand{\pad}{\right)}
\newcommand{\ci}{\left[}
\newcommand{\cd}{\right]}
\newcommand{\nn}{\mathbb{N}}
\newcommand{\B}{\mathcal{B}}
\newcommand{\p}{\mathbb{P}}
\newcommand{\E}{\mathbb{E}}
\newcommand{\lphi}{\widehat{\phi}}
\newcommand{\D}{\mathfrak{D}}
%\renewcommand{\qedsymbol}{\rule{1ex}{1ex}}

\begin{document}
 \begin{flushright}
  \begin{tabular}{|c|c|c|c|c|c|c|c|c|c|c|c|c|c|c|c|}
  \hline
  \ \ $E_1$\ \ &\ \ $E_2$\ \ &\ \ $E_3$\ \ &\ \ $E_4$\ \ &\ \ $E_5$\ \ &Calif.\\
  \hline
  & &  & & & \\
    & &  & & &\\
   \hline
 \end{tabular}\\
 \end{flushright}
\begin{center}
\large
Probabilidad
\normalsize

Rubén Pérez Palacios
Lic. Computación Matemática
NUA 424730

Agosto - Diciembre de 2020\\

Parcial 1 (25 de septiembre)
\end{center}

Resuelva \textbf{formal y detalladamente} los siguientes ejercicios, de manera totalmente individual.

Está estrictamente prohibido consultar las notas o cualquier otro tipo de bibliografía. Ante cualquier sospecha de esto, el examen se anulará (se calificará automáticamente con cero) sin derecho a réplica.

Puede utilizar únicamente la teoría vista en clase o en cursos de los dos semestres previos. Cualquier otro tipo de resultado, no podrá utilizarse (ni siquiera incluyendo su demostración).

Todos los ejercicios valen 100 puntos y cada inciso en cada ejercicio tiene el mismo valor.

La nota final de este examen será el promedio de las puntuaciones obtenidas en cada ejercicio.

\begin{enumerate}
\item \begin{enumerate}
\item Explique clara y concisamente los conceptos de convergencia casi segura, en probabilidad y en $L_p$, resaltando las relaciones entre ellos y qué tan restrictivo es cada tipo de convergencia.

\item Demuestre que existe una sucesión $\{X_n\}$ tal que ella converge casi seguramente, en probabilidad y en $L_p$ al mismo límite, donde $\{X_n\}$ es tal que para toda $n$, $X_n$ \textbf{no} es degenerada.

\item Escriba la definición de $X_n\to\infty$ en probabilidad.
\end{enumerate}

\item Sea $\{X_n\}$ una sucesión de variables aleatorias tales que $X_n\overset{L_p}{\to} X$ para algún $p\geq 1$ y sea $g:A\to B$ medible y \textbf{no constante}, donde $A,B\subseteq \rr$. Mencione al menos dos casos en los cuales la convergencia en $L_p$ de la sucesión $\{X_n\}$, implique $g(X_n)\overset{L_p}{\to} g(X)$.

\item Sean $\{X_n\},\{Y_n\}$ sucesiones de variables aleatorias tales que $X_n\overset{P}{\to} X$, $Y_n\overset{L_q}{\to} Y$ y $|X_n|\leq |Y_n|$ casi seguramente. 

\begin{enumerate}
\item Utilice el Teorema 7.3 de las notas para probar que $\big\{\frac{|X_n|}{1+|Y_n|}\big\}$ converge en probabilidad y determine la variable límite. \textbf{Nota}: cualquier solución que no use dicho teorema, será calificada automáticamente con cero puntos, independientemente de si es correcta o no.

\item Determine si $\big\{\frac{|X_n|}{1+|Y_n|}\big\}$ converge en $L_p$ para algún $p\geq 1$. Justifique formalmente su respuesta.
\end{enumerate}

\item Sea $\{\vec{X}_n\}$ una sucesión de vectores aleatorios $d$-dimensionales, tales que $\vec{X}_n=(X_{n,1},\dots,X_{n,d})$ y $X_{n,j}\overset{L_p}{\to}X_j$ para algún $p\geq 1$ ($p$ \textbf{no necesariamente es el mismo} para todos los vectores y tampoco para todas las entradas de cada vector). Demuestre que $\vec{X}_n\overset{P}{\to} \vec{X}$, donde $\vec{X}=(X_1,\dots,X_d)$.

\item Sea $m_n$ el mínimo de $n$ variables aleatorias iid con distribución común $exp(\theta)$, todas sobre el mismo espacio de probabilidad. Demuestre que $m_n\overset{L_p}{\to}0$ para todo $p>0$.
\end{enumerate}

\end{document}
