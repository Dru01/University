% Preámbulo
\documentclass[letterpaper]{article}
\usepackage[utf8]{inputenc}
\usepackage[spanish]{babel}

\usepackage{enumitem}
\usepackage{titling}

% Símbolos
	\usepackage{amsmath}
    \usepackage{amssymb}
	\usepackage{mathtools}
	\usepackage[thinc]{esdiff}

% Márgenes
	\usepackage
	[
		margin = 1.4in
	]
	{geometry}

% Imágenes
	\usepackage{float}
	\usepackage{graphicx}
	\graphicspath{{imagenes/}}
	\usepackage{subcaption}

% Ambientes
	\usepackage{amsthm}

	\theoremstyle{definition}
	\newtheorem{ejercicio}{Ejercicio}

	\newtheoremstyle{lemathm}{4pt}{0pt}{\itshape}{0pt}{\bfseries}{ --}{ }{\thmname{#1}\thmnumber{ #2}\thmnote{ (#3)}}
	\theoremstyle{lemathm}
	\newtheorem{lema}{Lema}

	\newtheoremstyle{lemademthm}{0pt}{10pt}{\itshape}{ }{\mdseries}{ --}{ }{\thmname{#1}\thmnumber{ #2}\thmnote{ (#3)}}
	\theoremstyle{lemademthm}
	\newtheorem*{lemadem}{Demostración}

% Ajustes
	\allowdisplaybreaks	% Los align pueden cambiar de página

% Macros
	\newcommand{\sumi}[2]{\sum_{i=#1}^{#2}}
	\newcommand{\dint}[2]{\displaystyle\int_{#1}^{#2}}
	\newcommand{\inte}[2]{\int_{#1}^{#2}}
	\newcommand{\dlim}{\displaystyle\lim}
	\newcommand{\limxinf}{\lim_{x\to\infty}}
	\newcommand{\limninf}{\lim_{n\to\infty}}
	\newcommand{\dlimninf}{\displaystyle\lim_{n\to\infty}}
	\newcommand{\limh}{\lim_{h\to0}}
	\newcommand{\ddx}{\dfrac{d}{dx}}
	\newcommand{\txty}{\text{ y }}
	\newcommand{\txto}{\text{ o }}
	\newcommand{\Txty}{\quad\text{y}\quad}
	\newcommand{\Txto}{\quad\text{o}\quad}
	\newcommand{\si}{\text{si}\quad}

	\newcommand{\etiqueta}{\stepcounter{equation}\tag{\theequation}}
	\newcommand{\tq}{:}
	\renewcommand{\o}{\circ}
	% \newcommand*{\QES}{\hfill\ensuremath{\boxplus}}
	% \newcommand*{\qes}{\hfill\ensuremath{\boxminus}}
	% \newcommand*{\qeshere}{\tag*{$\boxminus$}}
	% \newcommand*{\QESHERE}{\tag*{$\boxplus$}}
	\newcommand*{\QES}{\hfill\ensuremath{\blacksquare}}
	\newcommand*{\qes}{\hfill\ensuremath{\square}}
	\newcommand*{\QESHERE}{\tag*{$\blacksquare$}}
	\newcommand*{\qeshere}{\tag*{$\square$}}
	\newcommand*{\QED}{\hfill\ensuremath{\blacksquare}}
	\newcommand*{\QEDHERE}{\tag*{$\blacksquare$}}
	\newcommand*{\qel}{\hfill\ensuremath{\boxdot}}
	\newcommand*{\qelhere}{\tag*{$\boxdot$}}
	\renewcommand*{\qedhere}{\tag*{$\square$}}

	\newcommand{\abs}[1]{\left\vert#1\right\vert}
	\newcommand{\suc}[1]{\left(#1_n\right)_{n\in\N}}
	\newcommand{\en}[2]{\binom{#1}{#2}}
	\newcommand{\upsum}[2]{U(#1,#2)}
	\newcommand{\lowsum}[2]{L(#1,#2)}

	\newcommand{\N}{\mathbb{N}}
	\newcommand{\Q}{\mathbb{Q}}
	\newcommand{\R}{\mathbb{R}}
	\newcommand{\Z}{\mathbb{Z}}
	\newcommand{\eps}{\varepsilon}
	\newcommand{\ttF}{\mathtt{F}}
	\newcommand{\bfF}{\mathbf{F}}

	\newcommand{\To}{\longrightarrow}
	\newcommand{\mTo}{\longmapsto}
	\newcommand{\ssi}{\Longleftrightarrow}
	\newcommand{\sii}{\Leftrightarrow}
	\newcommand{\then}{\Rightarrow}

	\newcommand{\pTFC}{{\itshape 1er TFC\/}}
    \newcommand{\sTFC}{{\itshape 2do TFC\/}}
    \DeclarePairedDelimiter\ceil{\lceil}{\rceil}
    \DeclarePairedDelimiter\floor{\lfloor}{\rfloor}
    
% Datos
    \title{Probabilidad y Estadística\\Tarea 06}
    \author{Rubén Pérez Palacios\\Profesor: Dr. Octavio Arizmendi Echegaray}
    \date{24 Marzo 2020}

% DOCUMENTO
\begin{document}
	\maketitle
    
	\section*{Problemas}

    \begin{enumerate}
        
        \item Sean $X_1, \cdots, X_n$ variables aleatorias exponenciales independientes con parametros \linebreak $\lambda_1, \cdots, \lambda_n$, respectivamente.
		
		\begin{enumerate}
			\item Sea $Y_n = \min(X_1,\cdots,X_n)$, luego
			
			\begin{align*}
				F_Y(y) &= P(Y \leq y)\\
				&= 1 - P(Y > y)\\
				&= 1 - P(X_1 > y, \cdots, X_n > y) & \text{por definición de mínimo}\\
				&= 1 - \prod_{i=1}^n P(X_i > y) & \text{al ser independientes}\\
				&= 1 - \prod_{i=1}^n e^{-\lambda_i y} & \text{por ser exponenciales}\\
				&= 1 - e^{-\left(\sum_{i=1}^n \lambda_i\right) y}
			\end{align*}

			La cual es una distirbución exponencial con parámtero $\sum_{i=1}^n \lambda_i$.

			\item Veamos como es $P(X_1 < X_2)$
			
			\begin{align*}
				P(X_1 < X_2) &= \int_0^\infty P(X < Y | Y = y) f_Y(y) dy\\
				&= \int_0^\infty F_X(y) f_Y(y) dy\\
				&= \int_0^\infty (1 - e^{-\lambda_1 y}) \lambda_2 e^{-\lambda_2 y} dy\\
				&= \int_0^\infty \lambda_2 e^{-\lambda_2 y} dy - \int_0^\infty \lambda_2 e^{-(\lambda_1 + \lambda_2) y} dy\\
				&= 1 - \frac{\lambda_2}{\lambda_1 + \lambda_2}\\
				&= \frac{\lambda_1}{\lambda_1 + \lambda_2}
			\end{align*}

			\item Veamos $X_1 = Y$, quiere decir que $X_1$ es el minimo de $X_1,\cdots,X_n$, lo cual es cierto si y sólo si $X_1 \leq X_2,\cdots, X_1 \leq X_n$, por lo que
			
			\[P(X_1 = Y) = P(X_1 \leq X_2,\cdots, X)\]
			
			Al ser independientes entonces
			
			\[P(X_1 = Y) = \prod_{i=2}^{n}P(X_1 \leq X_i)\]

			Como las $X_i$ son continuas entonces $X_i - X_j$ también lo es, por lo tanto \linebreak $P(X_i = X_j) = 0$. Por lo que 

			\[P(X_1 = Y) = \prod_{i=2}^{n}P(X_1 < X_i),\]
			
			por el inciso anterior

			\[P(X_1 = Y) = \prod_{i=2}^{n}\frac{\lambda_1}{\lambda_1 + \lambda_i},\]

			por lo que concluimos que

			\[P(X_1 = Y) = \frac{\lambda_1^{n-1}}{\prod_{i=2}^{n}\lambda_1 + \lambda_i}\]

		\end{enumerate}

		\item Sea $(X,Y)$ un vector con función de densidad $f(x,y) = \frac{x+y}{8}$ para $0 \leq x, y \leq 2$.
		
		Empezaremos por calcular la funciones de densidad marginales de $X$ y $Y$, ya que podrian sernos utiles mas adelante.

		\begin{align*}
			f_X(x) &= \int_{-\infty}^\infty f(x,y) dy\\
			&= \int_{0}^2 f(x,y) dy\\
			&= \int_{0}^2 \frac{x+y}{8} dy\\
			&= \frac{x+1}{4}.\\
		\end{align*}

		Analogamente obtenemos que

		\[f_Y(y) = \frac{y+1}{4}.\]

		\begin{enumerate}
			\item Como $0 \leq x, y$ entonces $f(x,y) \geq 0$. Ahora veamos que estas acumulan $1$,
			
			\begin{align*}
				\int_{-\infty}^\infty \int_{-\infty}^\infty f(x,y) dx dy &= \int_{0}^2 \int_{0}^2 f(x,y) dx dy\\
				&= \int_{0}^2 \int_{0}^2 \frac{x+y}{8} dx dy\\
				&= \int_{0}^2 \frac{y+1}{4}dy\\
				&= 1\\
			\end{align*}

			Por lo que concluimos que es una función de distribución.

			\item La función de densidad marginal de $Y$ es $f_Y(y) = \frac{y+1}{4}$ y la de $X$ es $f_X(x) = \frac{x+1}{4}$, por lo que
			
			\begin{align*}
				f_{X|Y}(x|y) &= \frac{f(x,y)}{f_Y(y)}		&f_{Y|X}(y|x) &= \frac{f(x,y)}{f_X(x)}\\
				&= \frac{\frac{x + y}{8}}{\frac{y + 1}{4}}	& &= \frac{\frac{x + y}{8}}{\frac{x + 1}{4}}\\
				&= \frac{x + y}{2y + 2}						& &= \frac{x + y}{2x + 2}
			\end{align*}

			\item La probabilidad de $P(X > Y)$ es $\frac{1}{2}$ al ser indenticamente distribuidas. Aquí dejo el calculo de no ser lo suficientemente convincente. Primero calcularemos la función de distribución marginal de $X$.
			
			\begin{align*}
				F_X(x) &= \int_{-\infty}^{x} f_X(u) du\\
				&= \int_{0}^{x} \frac{u+1}{4} du\\
				&= \frac{x(x+2)}{8}.
			\end{align*}

			Usando ley total obtenemos

			\begin{align*}
				P(X < Y) &= \int_{0}^2 F_X(x) f_Y(x) dx\\
				&= \int_{0}^2 \frac{x(x+2)}{8} \frac{x + 1}{4} dx\\
				&= \frac{1}{2}
			\end{align*}

		\end{enumerate}

    \end{enumerate}

	\end{document}
