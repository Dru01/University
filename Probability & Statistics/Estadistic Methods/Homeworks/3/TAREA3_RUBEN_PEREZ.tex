% Preámbulo
\documentclass[letterpaper]{article}
\usepackage[utf8]{inputenc}
\usepackage[spanish]{babel}
\decimalpoint

\usepackage{enumitem}
\usepackage{titling}

% Símbolos
	\usepackage{amsmath}
	\usepackage{amssymb}
	\usepackage{amsthm}
	\usepackage{amsfonts}
	\usepackage{mathtools}
	\usepackage{bbm}
	\usepackage[thinc]{esdiff}
	\allowdisplaybreaks

% Márgenes
	\usepackage
	[
		margin = 1.2in
	]
	{geometry}

% Imágenes
	\usepackage{float}
	\usepackage{graphicx}
	\graphicspath{{imagenes/}}
	\usepackage{subcaption}

% Ambientes
	\usepackage{amsthm}

	\theoremstyle{definition}
	\newtheorem{ejercicio}{Ejercicio}

	\newtheoremstyle{lemathm}{4pt}{0pt}{\itshape}{0pt}{\bfseries}{ --}{ }{\thmname{#1}\thmnumber{ #2}\thmnote{ (#3)}}
	\theoremstyle{lemathm}
	\newtheorem{lema}{Lema}

	\newtheoremstyle{lemathm}{4pt}{0pt}{\itshape}{0pt}{\bfseries}{ --}{ }{\thmname{#1}\thmnumber{ #2}\thmnote{ (#3)}}
	\theoremstyle{lemathm}
	\newtheorem{sol}{Solución}
	
	\newtheoremstyle{lemathm}{4pt}{0pt}{\itshape}{0pt}{\bfseries}{ --}{ }{\thmname{#1}\thmnumber{ #2}\thmnote{ (#3)}}
	\theoremstyle{lemathm}
	\newtheorem{theo}{Teorema}

	\newtheoremstyle{lemademthm}{0pt}{10pt}{\itshape}{ }{\mdseries}{ --}{ }{\thmname{#1}\thmnumber{ #2}\thmnote{ (#3)}}
	\theoremstyle{lemademthm}
	\newtheorem*{lemadem}{Demostración}

% Macros
	\newcommand{\sumi}[2]{\sum_{i=#1}^{#2}}
	\newcommand{\dint}[2]{\displaystyle\int_{#1}^{#2}}
	\newcommand{\inte}[2]{\int_{#1}^{#2}}
	\newcommand{\dlim}{\displaystyle\lim}
	\newcommand{\limxinf}{\lim_{x\to\infty}}
	\newcommand{\limninf}{\lim_{n\to\infty}}
	\newcommand{\dlimninf}{\displaystyle\lim_{n\to\infty}}
	\newcommand{\limh}{\lim_{h\to0}}
	\newcommand{\ddx}{\dfrac{d}{dx}}
	\newcommand{\txty}{\text{ y }}
	\newcommand{\txto}{\text{ o }}
	\newcommand{\Txty}{\quad\text{y}\quad}
	\newcommand{\Txto}{\quad\text{o}\quad}
	\newcommand{\si}{\text{si}\quad}

	\newcommand{\etiqueta}{\stepcounter{equation}\tag{\theequation}}
	\newcommand{\tq}{:}
	\renewcommand{\o}{\circ}
	\newcommand*{\QES}{\hfill\ensuremath{\blacksquare}}
	\newcommand*{\qes}{\hfill\ensuremath{\square}}
	\newcommand*{\QESHERE}{\tag*{$\blacksquare$}}
	\newcommand*{\qeshere}{\tag*{$\square$}}
	\newcommand*{\QED}{\hfill\ensuremath{\blacksquare}}
	\newcommand*{\QEDHERE}{\tag*{$\blacksquare$}}
	\newcommand*{\qel}{\hfill\ensuremath{\boxdot}}
	\newcommand*{\qelhere}{\tag*{$\boxdot$}}
	\renewcommand*{\qedhere}{\tag*{$\square$}}

	\newcommand{\suc}[1]{\left(#1_n\right)_{n\in\N}}
	\newcommand{\en}[2]{\binom{#1}{#2}}
	\newcommand{\upsum}[2]{U(#1,#2)}
	\newcommand{\lowsum}[2]{L(#1,#2)}
	\newcommand{\abs}[1]{\left| #1 \right| }
	\newcommand{\bars}[1]{\left \| #1 \right \| }
	\newcommand{\pars}[1]{\left( #1 \right) }
	\newcommand{\bracs}[1]{\left[ #1 \right] }
	\newcommand{\inprod}[1]{\left\langle #1 \right\rangle }
    \newcommand{\norm}[1]{\left\lVert#1\right\rVert}
	\newcommand{\floor}[1]{\left \lfloor #1 \right\rfloor }
	\newcommand{\ceil}[1]{\left \lceil #1 \right\rceil }
	\newcommand{\angles}[1]{\left \langle #1 \right\rangle }
	\newcommand{\set}[1]{\left \{ #1 \right\} }
	\newcommand{\norma}[2]{\left\| #1 \right\|_{#2} }


	\newcommand{\NN}{\mathbb{N}}
	\newcommand{\QQ}{\mathbb{Q}}
	\newcommand{\RR}{\mathbb{R}}
	\newcommand{\ZZ}{\mathbb{Z}}
	\newcommand{\PP}{\mathbb{P}}
    \newcommand{\EE}{\mathbb{E}}
	\newcommand{\1}{\mathbbm{1}}
	\newcommand{\eps}{\varepsilon}
	\newcommand{\ttF}{\mathtt{F}}
	\newcommand{\bfF}{\mathbf{F}}

	\newcommand{\To}{\longrightarrow}
	\newcommand{\mTo}{\longmapsto}
	\newcommand{\ssi}{\Longleftrightarrow}
	\newcommand{\sii}{\Leftrightarrow}
	\newcommand{\then}{\Rightarrow}

	\newcommand{\pTFC}{{\itshape 1er TFC\/}}
	\newcommand{\sTFC}{{\itshape 2do TFC\/}}


% Datos
    \title{Métodos Estadísticos \\ Tarea 2}
    \author{Rubén Pérez Palacios Lic. Computación Matemática\\Profesor: Dr. Rogelio Ramos Quiroga}
    \date{\today}

% DOCUMENTO
\begin{document}
	\maketitle
	En el \'area de microbiolog\'ia cuantitativa es de inter\'es determinar la densidad de microorganismos presentes en un determinado medio. Sea $\theta$ el n\'umero medio de microorganismos por unidad de volumen en cierto medio l\'iquido. Queremos estimar $\theta$.
		
	Pensemos en una unidad de volumen (digamos 1 cm$^3$), y supongamos que el n\'umero de microorganismos, $Y$, por unidad de volumen sigue una ley Poisson, 
	
	\[P(Y=y) = \frac{e^{-\theta}\theta^y}{y!}, \qquad y=0,1,2,\cdots,\]
	
	donde $\theta$ es el n\'umero medio por unidad de volumen. La forma obvia es tomar una muestra, digamos 10 cm$^3$, del l\'iquido original y contar el n\'umero de microorganismos, $N$, que hay en la muestra y entonces $\widehat{\theta} = N/10$ y listo $\cdots$ un peque\~no problema  es que, en general, esto \underline{no} se puede hacer en forma directa (en algunos casos, la tecnolog\'ia todav\'ia no puede hacer ese conteo), entonces hay que  recurrir a m\'etodos indirectos.

	Entonces la idea es contar el n\'umero de platos, de los $n$, en los que no hay actividad biol\'ogica (platos inertes), sea $X$ el n\'umero de platos inertes, entonces $X$ sigue una distribuci\'on Binomial 
	
	\[P(X=x) = \left( \begin{array}{c} n \\ x \end{array} \right) p^x (1-p)^{n-x}, \qquad x = 0,1,2,\cdots,n,\]
	
	de aqu\'i que el problema se reduce a estimar $p$ (f\'acil: $\widehat{p} = X/n$), pues si tenemos $\widehat{p}$ entonces como $p = e^{-\theta}$, tendremos que $\widehat{\theta} = -\text{log}(\widehat{p})$ y listo $\cdots$ 

	En principio la idea es buena pero hay un pero. Si, por ejemplo, tenemos 6 platos de Petri, probablemente vamos a observar:

	\begin{align*}
		\text{actividad en plato 1 = si,} & \qquad \qquad \text{actividad en plato 2 = si,}\\
		\text{actividad en plato 3 = si,} & \qquad \qquad \text{actividad en plato 4 = si,}\\
		\text{actividad en plato 5 = si,} & \qquad \qquad \text{actividad en plato 6 = si}
	\end{align*}
	
	pues es muy probable que en cada unidad de volumen se vaya al menos un bicho, lo cual es suficiente para que, al ponerlo con nutientes, observemos actividad biol\'ogica.

	En general, es dif\'icil estimar la probabilidad de \'exito (plato inerte) si s\'olo observamos fracasos!. En fin, para sacarle la vuelta a este problema tenemos dos caminos: Aumentar $n$ o tomar muestras dilu\'idas.

	Consideremos la segunda opci\'on. Aqu\'i aumentamos la probabilidad de tener una muestra inerte simplmente diluyendo la muestra original, por ejemplo, tomamos 1 cm$^3$ del l\'iquido original y lo mezclamos con 1 cm$^3$ de un l\'iquido limpio y ponemos en un plato la mitad de esa mezcla la cual tendr\'a una densidad media igual a $\theta/2$.
	Podemos continuar de esta forma tomando muestras m\'as y m\'as dilu\'idas, en realidad no se tienen que dilu\'ir las muestras a la mitad, sea $a$ el factor de disoluci\'on ($a=2$ en el caso anterior); tendremos entonces (despu\'es de $k$ disoluciones) muestras con densidades (n\'umero de microorganismos por unidad de volumen):
	
	\[\frac{\theta}{a^0}, \quad \frac{\theta}{a^1}, \quad \frac{\theta}{a^2}, \quad \cdots \quad \frac{\theta}{a^k}.\]

	Supongamos que al nivel 0 de disoluci\'on observamos $x_0$ platos inertes (de un total de $n_0$ platos); al nivel 1 de disoluci\'on tenemos $x_1$ platos inertes (de un total de $n_1$) y as\'i sucesivamente. En cada paso, el n\'umero de platos inertes tendr\'a una distribuci\'on Binomial con par\'ametros $n_i$ y $p_i = \text{exp}(-\theta /a^i)$.

	Entonces, la verosimilitud para nuestros datos es

	\[L(\theta) = \prod_{i=0}^k f(x_i | \theta) = \prod_{i=0}^k \left( \begin{array}{c} n_i \\ x_i \end{array} \right) p_i^x (1-p_i)^{n_i-x_i}, \; \text{donde} \; p_i = \text{exp}\left( \frac{-\theta}{a^i} \right).\]

	En un experimento de este tipo se us\'o $a=2$, $k=9$ y, a cada nivel de disoluci\'on se us\'o $n_i=5$. Los datos (n\'umero de platos inertes) observados son:

	\[\left\{ x_i \right\} = \left\{ 0,0,0,0,1,2,3,3,5,5 \right\}\]

	\begin{enumerate}
		\item Encuentre el estimador de m\'axima verosimilitud para $\theta$ usando el m\'etodo Newton-Raphson.
		
		El estimador de m\'axima verosimilitud para $\theta$ es:
		
		\[\hat{\theta} = 30.64979,\]
		
		y es máximo puesto que 
		
		\[-I_{O} = -0.0122.\]

		\item Encontrar el intervalo de verosimilitud del 25\% para $\theta$.
		
		El intervalo del 25\% para $\theta$ es:

		\[IV_{0.25}\pars{\theta} = \set{\theta : R\pars{\theta;x} \geq 0.10} = \bracs{12.940345430114384,48.35923539576445}.\]
		
		\item La varianza del estimador de m\'axima verosimilitud (resultado asint\'otico) est\'a dada por $I^{-1}(\theta)/k$ (equivalentemente $[kI(\theta)]^{-1}$, donde 
		
		\[I(\theta) = E\left( -\frac{\partial^2 \text{log} f(x|\theta)}{\partial \theta^2} \right),\]
		
		note que esta expresi\'on supone observaciones independientes e id\'enticamente distribuidas (el cual no es nuestro caso: tenemos $k$ diferentes binomiales en la verosimilitud). Una forma de aproximar la varianza del estimador m\'aximo veros\'imil es tomando el inverso de la ``matriz de informaci\'on observada'':

		\[I_o = -\frac{\partial^2 \text{log} L(\widehat{\theta)}}{\partial \theta^2} = -\sum_{i=1}^k \frac{\partial^2}{\partial \theta^2} \left. \text{log} f(x_i|\theta) \right |_{\theta=\widehat{\theta}}.\]

		As\'i, la varianza se puede aproximar como $V(\widehat{\theta}) = I_o^{-1}$. Calcule esta varianza.

		La varianza es

		\[V\pars{\theta} = 81.6390154101694.\]

		\item Usando el resultado obtenido en el inciso anterior, obtenga un intervalo del 95\% de confianza para $\theta$: 
		
		\[\widehat{\theta} \pm z_{\alpha/2} \sqrt{I_o^{-1}}.\]
		
		Compare y comente este intervalo con el del inciso 2.

		Un intervalo del 95\% de confianza para $\theta$ es:

		\[\bracs{12.940345430114384,48.35923539576445}.\]

		\item El experimento completo us\'o 9 disoluciones (10 contando la primera en donde no hay disoluci\'on), en cada etapa de usaron 5 platos de Petri, dando un total de 50 platos en total. Una pregunta natural es: ?`Vali\'o la pena el trabajo de hacer las disoluciones?, ?`no hubiera sido mejor simplemente preparar 50 platos de Petri con muestras sin dilu\'ir y en base a ellos estimar $p$? ... Explique c\'omo  responder\'ia a esta pregunta usando simulaci\'on (no hay que hacer la simulaci\'on, simplemente explicar como le podr\'ia hacer).
		
		Dada una simulación calcularía el máximo estimador de verosimilitud de $theta$ cuando las muestras no se disimulan, llamemoslo $\hat{\theta}_1$ y $\hat{\theta}$ de la misma simulación, después calcularía sus respectivos intervalos de confianza. Además generaría varias simulaciones como la obtenida en el problema $(n=5,k=9,\cdots)$ y varias con solo una disolución y aquel intervalo de confianza que tenga un valor mayor o más cercano a 95\% sería el criterio de desempate sería el tenga un longitud menor.
	\end{enumerate}
\end{document}

